\documentclass[10pt,twoside,a5paper,openright]{book}
\usepackage[utf8]{inputenc}
\usepackage{ngerman}
\usepackage{url}
\usepackage{fancyhdr}                % Leck're headers
\usepackage{latexsym}                % symbole!!
\usepackage{amsmath}                 % Mehr mathematische Befehle...
\usepackage{amssymb}                 % ... und Symbole
\usepackage[final,pdftex,colorlinks,urlcolor=blue]{hyperref}		% setzt Links und Verweise im PDF-Dokument
\usepackage{graphicx}               % zum Einf"ugen von eps-Bildern etc.
\usepackage{fullpagegraphic}
\usepackage{wasysym}
\usepackage{color,colortbl,titlesec, array}

% For Watermark
% \usepackage{draftwatermark}
% \SetWatermarkScale{2}
% \SetWatermarkText{\sf nach Graz}

\setlength{\hoffset}{-.5in}
\setlength{\voffset}{-1in}
%\addtolength{\textheight}{1.9in}
%\addtolength{\textwidth}{0.9in}
\setlength{\textheight}{15.5cm}
\setlength{\textwidth}{11.5cm}
\setlength{\oddsidemargin}{0.3cm}
% \setlength{\evensidemargin}{.2cm}
\setlength{\topmargin}{1.3cm}
\setlength{\footskip}{1.3cm}

%%% KOPF & FUSSZEILEN
\pagestyle{fancy}
\fancyhead{} % Aufraeumen
\fancyhead[ER]{\scshape \footnotesize \leftmark}
\fancyhead[OL]{\scshape \footnotesize \rightmark}
\fancyhead[EL,OR]{\thepage}
\fancyfoot{} % Aufraeumen
\fancyfoot[EL]{\scshape \footnotesize KoMa-Handbuch}
%\fancyfoot[EC,OC]{\scshape \footnotesize FH Regensburg}
\fancyfoot[OR]{\scshape \footnotesize Berufungskommissionen}

%---------------------------------------------------------------
%für Fragebogen
\definecolor{light}{gray}{0.80}
\newcommand{\greybox}[3]{\colorbox{light}{\parbox[c][#1][c]{#2}{\center{$#3$\vspace{3.7mm}}}}}

\newcommand{\checkboxen}
{
    \raisebox{1.5mm}
    {
        \begin{minipage}{50mm}
            \setlength{\tabcolsep}{0.43mm}
            \setlength{\fboxsep}{0mm}
            	   \begin{Form}
                	\CheckBox{ja}\hspace{4mm}\CheckBox{nein}
             \end{Form}
        \end{minipage}
    }
}

\newcommand{\bewertung}
{
     \raisebox{1.5mm}
    {
        \begin{minipage}{40mm}
            \setlength{\tabcolsep}{0.43mm}
            \setlength{\fboxsep}{0mm}
          \begin{tabular}{p{8mm}p{.5mm}p{8mm}p{.5mm}p{8mm}p{.5mm}p{8mm}p{.5mm}}
                \greybox{5mm}{6mm}{--}&&\greybox{5mm}{6mm}{-}&&\greybox{5mm}{6mm}{+}&&\greybox{5mm}{6mm}{++}
            \end{tabular}
        \end{minipage}
    }
}




\begin{document}
% aussen Umschlag
\frontmatter
\includegraphicsfullpage{titelseite}
\newpage
% leere Seite
\thispagestyle{empty}~
\newpage

%%% Schmutzumschlag

\begin{titlepage}
\begin{flushright}\sffamily
	\vspace*{1cm}
	{\Huge{Handbuch}}\\
	\vspace{2.0cm}
	{\large zur}\smallskip\par
	{\huge{Studentischen Mitwirkung in}\medskip \\
	{\huge Berufungskommissionen}} \\
	\vspace{2ex}
	\vfill
	{\large{Arbeitskreis Berufungskommissionen}\smallskip \\ 
	{\large{der Konferenz der deutschsprachigen Mathematikfachschaften}}\\
	\vspace{3cm}
	{\Large Sommersemester 2012}}\\
	\vspace{14ex}
\end{flushright}
\end{titlepage}


%%% IMPRESSUM
\newpage
\vspace*{\fill}
\subsection*{Impressum}

\begin{table}[h]
\footnotesize
%	\begin{center}
		\begin{tabular}{ll}
		Herausgeber:				& KoMa-B"uro \\
									& c/o Fachschaftsrat Mathematik \\
									& an der TU Chemnitz \\
									& \texttt{www.die-koma.org} \\
		Erschienen:					& August 2012 \\
		Auflage:					& 2.~Auflage\\
		Redaktion:					& Andreas Cord-Landwehr, Tim Haga \& Maria Meister \\
									& Arbeitskreis "`Berufungskommissionen"' (AK BK) \\ 
		Redaktionsschluss:			& 30. August 2012 \\
		Copyright:					& Das Copyright f"ur alle Texte liegt bei den jeweiligen Autoren. \\
		\end{tabular}
%	\end{center}
\end{table}

\newpage
\tableofcontents\thispagestyle{fancy}


%%%%%%%%%%%%%%%%%%%%%%%%%%%%%%%%
%%% KAPITEL 1: VORBEMERKUNGEN
%%%%%%%%%%%%%%%%%%%%%%%%%%%%%%%%

\chapter*{Vorbemerkungen}\thispagestyle{fancy}
\addcontentsline{toc}{chapter}{Vorbemerkungen}

Das vorliegende Handbuch zur studentischen Mitwirkung in Berufungskommissionen wurde vom Arbeitskreis "`Berufungskommissionen"' (AK BK) der Konferenz der deutschsprachigen Mathematikfachschaften (KoMa) erstellt. Die hierin enthaltenen Ratschläge und Erläuterungen wurden von Fachschaftsvertretern zahlreicher Universitäten und Fachhochschulen im gesamten deutschsprachigen Raum zusammengetragen. 

Trotz zahlreicher regionaler Unterschiede bei der Durchführung von Berufungsverfahren versucht dieses Handbuch einen in sich geschlossenen Gesamtablauf darzustellen, dessen Struktur an jeder deutschsprachigen Hochschule in dieser oder vielleicht in einer leicht abgeänderten Form wiedergefunden werden kann. Gleiches gilt für unterschiedliche Fachbereiche in unserer Hochschullandschaft. Zwar wurde dieses Handbuch speziell von Mathematikern für Mathematiker geschrieben, dennoch trifft der wesentliche Teil dieses Handbuches auch auf andere Fachkulturen zu.

Neben einer einheitlichen und klaren Darstellung des Gesamtablaufs halten wir es aber auch für wichtig, die regionalen Spezifika bzw. mögliche Ausprägungen von Berufungsverfahren zu erläutern. Dieses soll zum einen zur Abgrenzung der verschiedenen Landesgesetzgebungen dienen, zum anderen aber auch den Blick auf Verfahren an anderen Hochschulen weiten.

Der Aufbau dieses Handbuches ist in drei Kapitel unterteilt. Der erste Teil liefert eine Einführung in den Bereich der Berufungskommissionen. Im zweiten Teil wird der Ablauf einer Berufungskommission ausführlich dargestellt; insbesondere werden hier die verschiedenen Phasen erläutert und durch Hinweise und Empfehlungen kommentiert. Der dritte Teil geht auf den Kontakt mit anderen Fachschaften im Bereich der Berufungskommissionen und auf rechtliche Fragestellungen ein.

Zur Schreibweise dieses Dokumentes ist zu sagen, dass wir uns aufgrund der besseren Lesbarkeit für das generische Maskulinum entschieden haben. Stets sind jedoch Personen beiden Geschlechtes gemeint.

Für Anregungen, Kommentare, Ergänzungen und weitere Hinweise ist der Arbeitskreis Berufungskommissionen dankbar. Für Kontakt zur KoMa allgemein oder direkt zum Arbeitskreis verweisen wir auf die Webseite der KoMa: \url{www.die-koma.org}.

\vspace{1cm}\noindent
Die Teilnehmer der WAchKoMa Bremen, 2009


%%%%%%%%%%%%%%%%%%%%%%%%%%%%%%%%
%%% KAPITEL 2: EINLEITUNG
%%%%%%%%%%%%%%%%%%%%%%%%%%%%%%%%


\mainmatter
\chapter{Einleitung}\thispagestyle{fancy}

\section{Allgemeines}
Dieses Handbuch ist für \emph{DICH} gedacht. Sein alleiniger Zweck ist es, dich bei deiner Arbeit in deiner Berufungskommission (BK) zu unterstützen. Zunächst versuchen wir dir daher das notwendige Hintergrundwissen über den Ablauf und die rechtlichen Rahmenbedingungen zu vermitteln, denn dieses Wissen haben auch alle anderen Mitglieder in der Kommission. Daneben geben wir Empfehlungen und Erfahrungen aus unseren bisherigen selbst miterlebten Berufungskommissionen weiter. Dieser Schatz an Erfahrungen summiert sich innerhalb des Arbeitskreises auf weit über 50 Kommissionen von W1 bis W3 Professuren, über außerplanmäßige bis Honorarprofessuren, über Kampfabstimmungen bis Einigkeit. 

Viele dieser Empfehlungen und Hinweise sind jedoch situationsabhängig zu sehen. Nur allein du kannst entscheiden, was zu einem bestimmten Moment sinnvoll ist. Denn das Gelingen einer Berufungskommission liegt zum Teil auch in deiner Verantwortung. Du hast dafür zu sorgen -- wie auch alle anderen Mitglieder der Kommission -- einen fähigen Hochschullehrer auszuwählen, welcher eine exzellente Lehre macht und nebenbei auch in der Forschung die ein oder anderen Lorbeeren vorzuweisen hat. Denn wenn man sich bei einer Berufung für ein faules Ei entscheidet, dann wird man es bis zur Pensionierung nicht mehr los.

Bevor wir nun mitten in das Thema starten, möchten wir dir mit auf den Weg geben, dass wir uns sehr freuen, dass du dich bereit erklärt hast, an einer Berufungskommission teilzunehmen. Es ist zwar ein großer Teil Verantwortung und es kostet auch einiges an Zeit, aber die Erfahrungen und Einblicke die man durch solch eine Kommission gewinnen kann wiegen dieses bei weitem auf. Und zu guter Letzt kannst du dich später damit rühmen, den "`Super-Prof"' an deine Uni geholt zu haben ;-)

\section{Rechtliche Rahmenbedingungen}%TODO mit Österreich und Schweiz checken!
In Deutschland gibt es in jedem Bundesland ein Hochschulgesetz. Durch dieses Gesetz werden die rechtlichen Rahmenbedingungen für die Berufung neuer Professoren festgelegt. Ebenso sollte an deiner Hochschule eine Berufungsordnung existieren. In dieser sind dann weitere, feinere Regelungen zum Ablauf der jeweiligen Berufungsverfahren zu finden. Insbesondere sind dieses folgende Punkte:
\begin{itemize}
	\item Zusammensetzung der Kommission (insb. studentische Mitglieder und Vertretungsmöglichkeiten)
	\item Ablauf der Kommission, Fristen, Mitspracherecht und deine Aufgaben als Mitglied der Kommission
	\item Wirkung findende Gleichstellungsgesetze\footnote{Anmerkung: Da oft in den Professoren- und Mitarbeiterreihen nur wenige Frauen zu finden sind, können Studierende oftmals einen besonderen Einfluss auf die Kommission ausüben, da sie die Möglichkeit haben weibliche Mitglieder zu entsenden.}
\end{itemize}
In den weiteren Abschnitten möchten wir nun auf jeden Teil der BK kurz eingehen und einige "`Best Practices"' vorstellen:  

\section{Begriffserklärungen}
Wie in allen Hochschulgremien gibt es auch in Berufungskommissionen eine eigene Sprache. Dieses sind Bezeichnungen für bestimmte Gehaltsstufen, Lehrstulausstattungen, aber auch Verfahrensanweisungen. Die Wichtigsten stellen wir hier kurz vor:
\begin{description}
	\item [Gehaltsstufen]%TODO nur in Deutschland?
			Mittlerweile werden alle neu zu berufenden Professoren nach der \emph{Bundesbesoldungsordnung W}
			bezahlt. Ältere Professoren, die schon lange an der Hochschule sind werden ggf. noch nach der alten
			Ordnung C bezahlt. In der W-Besoldung gibt es drei Gehaltsstufen, W1, W2 und W3. Dabei werden
			\emph{Juniorprofessoren} mit W1 besoldet, alle anderen Professoren nach W2 oder W3. Abgesehen vom Grundgehalt
			unterscheiden sich die Stellen in der Ausstattung mit Mitarbeitern und Sachmitteln. Dabei sind
			W3-Professuren in der Regel am Besten ausgestattet. In manchen Bundesländern gibt es keine
			W2-Professuren, hier werden die W3-Professuren zum Teil mit \emph{Leitungsfunktion} ausgeschrieben.
			Diese werden dann besser bezahlt und haben mehr Personal- und Sachmittel zur Verfügung.

	\item [Hausberufung]
			Von einer Hausberufung spricht man, wenn der zu Berufende schon am ausschreibenden Institut arbeitet,
			etwa als Postdoc, oder Juniorprofessor. Hausberufungen sind in Deutschland eher unüblich und werden
			oft kritisch gesehen. Die Berufungskommission muss in solch einem Fall besonders begründen, warum der Bewerber
			der Beste aller Kandidaten ist, da sonst der Eindruck entstehen kann, dass dort eigene Leute
			bevorteilt werden.
			
	\item [Honorarprofessur]
			Dozenten, die keine Hochschullehrer sind, weil sie etwa in der Wirtschaft oder einem Forschungsinstitut
			arbeiten, können zum Honorarprofessor berufen werden. Eine Honorarprofessor wird hierbei als Auszeichnung für
			verdiente Gastdozenten verstanden. Honorarprofessoren haben meist keine oder nur eine sehr geringe
			Lehrverpflichtung und arbeiten hauptamtlich noch in ihrem eigentlichen Beruf.
			
	\item [Juniorprofessoren]
			Die sogenannten Juniorprofessuren wurden 2002 eingeführt. Sie stellen eine Möglichkeit dar, sich
			ohne Habilitation für eine ordentliche Professur zu qualifizieren. Juniorprofessoren werden in der Regel mit W1
			besoldet und haben eine geringere Lehrverpflichtung als W2- oder W3-Professoren. Sie sind, anders als die "`großen"'
			Professuren, auch keine Professuren auf Lebenszeit.

	\item [Lehrprofessur]
			In einigen Bundesländern wurden vor einigen Jahren sogenannte Lehrprofessuren eingeführt.
			(auch Hochschuldozenturen und Juniordozenturen genannt). Diese werden mit W1 oder W2 bezahlt, haben
			in der Regel aber keine Mitarbeiterstellen und eine stark er\-höh\-te Lehrverpflichtung. Daher
			sind diese Stellen eher unattraktiv und wurden bisher kaum besetzt. Ebenso ist es momentan noch
			politisch stark umstritten ob diese Stellen überhaupt mit dem Humboldtschen Prinzip der Einheit
			von Forschung und Lehre vereinbar sein. 
			
	\item [Mitarbeiterstellen]
			Wie das Gehalt wird bei der Berufung auch festgelegt, wie viele Mitarbeiter ein Professor bekommt.
			In der Regel kann er frei entscheiden, wie er die Mitarbeiterstellen auf Postdocs und Doktoranden
			aufteilt. Zum Beispiel kann in Bremen ein W2-Professor entscheiden, ob er einen Postdoc oder zwei
			Doktoranden einstellen möchte. Üblicherweise ist die genaue Anzahl der Mitarbeiter 
			Verhandlungssache.
			
	\item [Pari Passu]
			Kann sich die Berufungskommission bei der Erstellung der Berufungsliste über die Reihenfolge zweier
			Kandidaten nicht einig, kommt es manchmal vor, dass beide \emph{pari passu} gesetzt werden. Pari passu
			bedeutet soviel wie "`gleichrangig"'. Man hat dann z.\,B. zwei zweite Plätze. Oftmals wird dann ein
			\emph{Sperrvermerk} gesetzt.
			
	\item [Sperrvermerk]
			Ein Sperrvermerk wird auf die Berufungsliste gesetzt, wenn die Berufungskommission noch einmal
			über die Liste befinden möchte. Dies kann zum Beispiel der Fall sein, wenn die Berufungskommission zwei Kandidaten
			pari passu gesetzt hat und erst bei Absage des Erstplatzierten über die Reihung entscheiden
			möchte.

	\item [Tenure Track]
	 		Oft hört man die Professoren in der Berufungskommission Begriffe sagen, wie \emph{tenure} oder \emph{tenure track}.
	 		Diese Begriffe stammen aus dem angelsächsischen System. Tenure bezeichnet eine unbefristete Stelle,
	 		vergleichbar mit "`Beamter auf Lebenszeit"'. Tenure Track bezeichnet eine (auf fünf oder sechs) Jahre
	 		befristete Stelle, die anschließend bei positiver Evaluation in Tenure umgewandelt wird.
\end{description}


%%%%%%%%%%%%%%%%%%%%%%%%%%%%%%%%
%%% KAPITEL 3: KOMMISSIONSARBEIT
%%%%%%%%%%%%%%%%%%%%%%%%%%%%%%%%


\chapter{Kommissionsarbeit}\thispagestyle{fancy}
Anders als man erwarten könnte, beginnt die Arbeit für eine Berufungskommission bereits lange vor der ersten Sitzung. Denn schon vorher muss die Stelle genehmigt werden und auch in allen an der Ausschreibung beteiligten Statusgruppen wird schon im Vorhinein gemauschelt, wer sich denn an der Kommission beteiligen sollte. Dieses trifft natürlich auch für die Studierenden zu.

Dieses Kapitel führt im Folgenden chronologisch durch den Ablauf einer Berufungskommission und zeigt die einzelnen Etappen auf, an denen ihr besonders gefragt seid.


\section[Zusammensetzung und Ausschreibung]{Zusammensetzung der Kommission und Ausschreibung der Stelle}
\sectionmark{Zusammensetzung und Ausschreibung}
Die Ausschreibung und Beantragung bzw. Einrichtung einer neuen Stelle ist üblicherweise ein langwieriger Prozess. Wir gehen hier davon aus, dass bereits eine entsprechende Instanz deiner Hochschule die Ausschreibung einer Professur angeordnet bzw. genehmigt hat (z.\,B. Dekanat, Präsidium, Rektorat).

Für die Besetzung dieser Stelle werden an öffentlichen Hochschulen Kommissionen eingesetzt, welche jeweils eine Liste mit Vorschlägen und eine entsprechende Reihung erarbeitet. In Deutschland -- je nach Bundesland und geltendem Hochschulgesetz kann die Zusammensetzung der Kommission und Gestaltung der Arbeit innerhalb der Kommission recht unterschiedlich ausfallen -- befindet sich stets mindestens ein studentischer Vertreter in solch einer Kommission\footnote{Wir gehen davon aus, dass sich an dieser Tatsache auch in Zukunft nichts ändern wird, verweisen aber auf das Publikationsdatum dieses Handbuches.}. In diesem Abschnitt soll nun der Prozess der Kommissionszusammensetzung, aber auch der Auswahl eines geeigneten studentischen Mitgliedes genauer beleuchtet werden.

\subsection{Mitglieder der Kommission} 
Die Zusammensetzung der Kommission wird üblicherweise im Fakultätsrat oder Fachbereichsrat vorgenommen. Dort nominieren die jeweiligen Statusgruppen ihre Vertreter -- also jeweils Professoren, Mitarbeiter und Studierende -- und wählen diese nach Statusgruppen getrennt. Die Anzahl der Vertreter der jeweiligen Statusgruppen regelt dabei eure \emph{Berufungsordnung}. 

Neben den Vertretern der einzelnen Statusgruppen kann eure Berufungsordnung noch weitere Mitglieder vorschreiben. Eventuell wünscht sich aber auch der Fakultätsrat oder Fachbereichrats weitere externe oder beratende Mitglieder, welche sich an der Kommission beteiligen sollen. Solche Mitglieder können etwa folgende sein (die Bezeichnungen variieren von Hochschule zu Hochschule):

\begin{description}
	\item [Beratendes Mitglied] 
	Ein beratendes Mitglied kann eine Person aus der eigenen Hochschule, also z.\,B. ein weiterer Professor oder aber auch ein weiterer Student sein.
		
	\item [Fakultätsübergreifende Mitglieder] 
	Diese Mitglieder aus einer anderen Fakultät --- gele\-gent\-lich auch "`Wachhunde genannt"' --- sollen im Auftrag der Hochschulleitung dafür sorgen, dass das Berufungsverfahren ordnungskonform und zügig durchgeführt wird. Diese sind jeder nicht an jeder Hochschule zu finden.
		
	\item [Auswärtiger Professor] 
	Dieses bezeichnet einen Professor einer anderen Hochschule. Gerade bei kleinen Fachbereichen wird gerne auf diese Möglichkeit zurückgegriffen, um eine hinreichende Fachkompetenz in der Kommission zu haben. Auswärtige Professoren haben dabei in der Regel Stimmrecht.
		 
	\item [Auswärtiger Experte]
	Experten sind meist Personen außerhalb der Hochschullandschaft, welche mit oder ohne Stimmrecht in die Kommission aufgenommen werden. Beispielsweise bei Stiftungsprofessuren wird gerne ein Vertreter der entsprechenden Stiftung mit in die Kommission aufgenommen.
	
	\item [Frauenbeauftragte]
	Die Frauenbeauftragte (oder Gleichstellungsbeauftragte, je nach Hochschule)	soll die Rechte der Frauen vertreten. Sie hat in der Regel kein Stimmrecht, darf aber ein Votum zur Frage der Berücksichtigung von
	Frauen abgeben. In vielen Bundesländern darf Sie auf Einladung von Frauen bestehen.
	
	\item [Schwerbehindertenbeauftragte]
	Der Schwerbehindertenbeauftragte soll die Rechte schwerbehinderter Bewerber vertreten. Er nimmt in der
	Regel nur dann an der Berufungskommission teil, wenn und solange schwerbehinderte Bewerber im Verfahren
	sind. Der Schwerbehindertenbeauftragte hat kein Stimmrecht.

\end{description}

Ob die o.\,g. Mitglieder Stimmrecht haben oder nicht, hängt von der jeweiligen Berufungsordnung ab. Bei der Zusammensetzung der Kommission gilt jedoch immer: die Hälfte der Mitglieder müssen Professoren sein. Daher kann sich durch das Hinzuziehen von weiteren Mitgliedern ggf. die Stimmengewichtung der professuralen Mitglieder innerhalb der Kommission ändern. Hier müssen wir aber auf deine jeweilige Berufungsordnung verweisen.

\subsection{Das studentische Mitglied}
Es ist nicht immer leicht sich für ein studentisches Mitglied für die Kommission zu entscheiden bzw. es überhaupt zu finden. Folgendes sollte bei der Auswahl eines studentischen Mitgliedes aber immer beachtet werden:
\begin{itemize}
	\item Eine Berufungskommission benötigt punktuell sehr viel Zeit. Es ist insbesondere nur schwer möglich, neben dem Studium sämtliche Kommissionssitzungen, Bewerbungsvorträge und Bewerbungsgespräche zu besuchen. Daher solltest du schauen, ob die Möglichkeit eines Stellvertreters in deiner Berufungsordnung gegeben ist. Falls dieses nicht möglich ist, so kann man oft ein weiteres beratendes studentisches Mitglied in die Kommission wählen lassen.
	
	\item Es ist ggf. möglich Vergünstigungen und Erleichterungen für die Arbeit in einer Berufungskommission zu erhalten. So kann man mit den Dozenten absprechen, dass man aufgrund der hohen Arbeitsbelastung einen Übungszettel nicht bearbeiten oder später abgeben darf. Insbesondere in der Woche, in der die Vorträge und Interviews stattfinden, kommt man kaum zum Studieren. Wenn man Kurse bei Professoren besucht, die selber in der Berufungskommission sitzen, hat man gute Chancen etwas Arbeitserleichterung zu bekommen. Sollte es in einigen deiner Kurse Anwesenheitspflicht geben, kannst du für Sitzungen der Berufungskommission davon befreit werden. 
	
	\item Ihr solltet euch immer \emph{selbst} -- ohne Beeinflussung von außen -- überlegen, wen ihr in diese Kommission schicken möchtet. Diese Person sollte aber das Grundstudium, bzw. die erste Hälfte des Bachelors abgeschlossen haben. Dieses wird teilweise in den Berufungsordnungen auch gefordert. In jedem Falle ist dieses aber sinnvoll, da so sicher gestellt wird, dass bereits eine genügende Fachkompetenz und Erfahrung beim studentischen Mitglied vorhanden ist.
	
	\item Ein wichtiger Punkt ist die Frauenquote, welche stets eine Kardinalsfrage bei der Zusammenstellung der Kommission ist. Meist haben die Professoren oder Mitarbeiter nur einen sehr geringen Frauenanteil. Somit können Studierende allein über das Einsetzen von weiblichen Kommissionsmitgliedern ggf. ein besonderes Gewicht in dieser Kommission einnehmen.
\end{itemize}

\subsection{Ausschreibung der Stelle}
Die Stellenausschreibung wird üblicherweise in der ersten Kommissionssitzung gemeinsam mit einem \emph{Kriterienkatalog} verabschiedet. Bei beidem solltet ihr darauf achten, dass die pädagogische Eignung explizit genannt wird. Ebenso sollte der Lehrbedarf im Service-Bereich\footnote{Service-Bereich sind in diesem Sinne Veranstaltungen für andere Studiengänge, wie z.B. Mathe für Wirtschaftswissenschaftler.} auch als Aufgabe in der Ausschreibung stehen.

Neben der pädagogischen Eignung für die abzudeckenen Veranstaltungen solltet ihr zudem darauf achten, dass die Stelle breit genug ausgeschrieben wird. Ein ab und an auftretendes Problem ist es nämlich, dass teilweise schon sehr genaue Vorstellungen existieren, wer die Stelle annehmen könnte. Wenn nun auch die Ausschreibung explizit auf diese Person zugeschnitten ist verhindert das die Möglichkeit einer Auswahl. 

\subsubsection*{Kriterienkatalog}\label{kriterienkatalog}
Der Kriterienkatalog ist die wesentliche Entscheidungsgrundlage für die Auswahl und Reihung der Listenkandidaten. Die Festlegung der Kriterien muss in der Regel in der ersten Kommissionssitzung -- insbesondere vor Beschluss der Ausschreibung -- erfolgen und dient zur Bewertung der Bewerber. Die darin enthaltenen Kriterien sind die alleinigen Bewertungskriterien, welche die Kommission im Weiteren anlegen darf. Kriterienkataloge gibt es sowohl mit als auch ohne Gewichtung und Reihung der einzelnen Punkte. 

\begin{center}
\fbox{
\begin{minipage}{0.8\textwidth}
\begin{itemize}
	\item Wissenschaftliche Qualifikation
	\item Passendes Arbeitsgebiet gemäß Ausschreibung
	\item {\bf Lehr- und Vortragserfahrung sowie pädagogische Eignung}
	\item Fachübergreifende Bezüge zu anderen an der Hochschule vertretenden Arbeitsgebieten (Potential zur Bildung von Kooperationen)
	\item Erfahrung in der Einwerbung von Drittmitteln
	\item Internationale Erfahrung
\end{itemize}
\end{minipage}
}
\begin{center}
Ein Beispiel für einen Kriterienkatalog.
\end{center}
\end{center}

\section{Bewerbungsunterlagen}\thispagestyle{fancy}
Nachdem die Stelle ausgeschrieben worden ist kommen ca. zwei Monate später die Bewerbungsunterlagen im entsprechenden Sekretariat an. Oft erstellt eine Schreibkraft direkt aus diesen Daten Übersichtstabellen, um Bewerber leichter vergleichen zu können. Dieses ist jedoch nicht immer der Fall und auch die Vergleichsinformationen zur Lehre der einzelnen Dozenten schwanken stark. Wir empfehlen euch daher aus den folgenden Punkten die für euch wichtigsten Vergleichspunkte zu wählen und eine eigene Vergleichstabelle zur Lehre der einzelnen Bewerber zu erstellen.

Folgende Merkmale können hierzu sinnvoll verwendet werden:
\begin{itemize}
	\item Anzahl der bereits gehaltenen Lehrveranstaltungen? (je Grund- und Hauptstudium)
	\item Anzahl der bisher betreuten Abschlussarbeiten? (je Bachelor und Master)
	\item Größe der bisher größten Veranstaltung?
	\item Hat der Bewerber Erfahrung mit Lehramtsvorlesungen?
	\item Hat der Bewerber Erfahrung im Service-Bereich?
	\item Welche Vorlesungen hat der Bewerber bereits angeboten?
	\item Wie viele (fachspezifische) Seminare hat der Bewerber schon angeboten?
	\item Hat der Bewerber schon einen ganzen Vorlesungszyklus gelesen?
	\item Wenn es an seiner Heim-Hochschule Evaluationen gibt: Wie hat der Bewerber abgeschnitten? Hat er Preise erhalten?
	\item Steht in der Bewerbung überhaupt etwas zu seiner Lehre?
\end{itemize}

Diese Vergleichstabelle sollte zur zweiten Sitzung der Kommission fertig sein. Ebenso ist eine Gruppierung der Bewerber anhand dieser Liste in Gruppe A, B und C (von gut über mittel nach schlecht) sinnvoll. Zwar wird in der zweiten Sitzung vor allem die Qualität und Quantität der bisherigen Publikationen der Bewerber bewertet, jedoch sollten Zweifel an der Lehrbefähigung einzelner Kandidaten auch schon an dieser Stelle deutlich angebracht werden.

Üblicherweise gibt es am Ende der Sitzung immer die Frage, ob man einige Kandidaten noch aus dem Grund einladen sollte, um etwa die Gleichstellungsbeauftragte oder andere Menschen "`zu beruhigen"'. Um diesen Diskussionen mit sachlichen Argumenten entgegentreten zu können, solltest du dich im Vorfeld bereits genauer mit den hierfür in Frage kommenden Bewerbern beschäftigen. Denn auch hier sollte die Qualität der Bewerber an erster Stelle stehen und schlechte bis mittelmäßige Bewerber sollten nicht in die zweite Runde gelangen.

Nach dieser Sitzung wird es dann (meistens) mindestens vier Wochen Pause geben, bis Termine mit allen Bewerbern vereinbart wurden, an denen sie sich persönlich bei euch vorstellen. Diese Zeit sollte mit dem Einholen von Informationen sinnvoll gefüllt werden.

\section{Informationen einholen}
Es ist üblich, dass die studentischen Mitglieder der Berufungskommission Informationen über die Lehrqualität der Bewerber an den jeweiligen Heimuniversitäten einholen. -- Meist wird das sogar von den Professoren erwartet. -- Dieses sollte auch ruhig vor dem Bewerbungsgespräch geschehen, damit du entsprechend dieser Informationen Fragen stellen kannst.

Im deutschsprachigen Raum findet man an fast jeder Hochschule eine Fachschaft, Studiengangsausschuss oder Fachschaftsverein. Dieses sollte deine erste Anlaufstelle sein. Am einfachsten sind diese per E-Mail zu erreichen.  Wenn ihr per E-Mail keine Antwort erhaltet solltet ihr aber spätestens in der Woche vor den Vorträgen versuchen, telefonisch Informationen zu bekommen.

Daneben existieren aber noch einige weitere Möglichkeiten Informationen über die Bewerber zu erhalten, die nicht direkt in den Bewerbungsunterlagen stehen.

\subsection{Wege um Informationen zu bekommen}
Je nach Ergiebigkeit der einzelnen Quellen bietet es sich an mehrere der folgenden Möglichkeiten zu kombinieren:
\begin{description}
	\item [Google]
		  Dieses kann der erste Anlaufpunkt sein, um zunächst einmal ein grobes Bild über den Bewerber zu erhalten.
	\item [Webseite des Bewerbers]
		  Sofern die Bewerbungsunterlagen nicht besonders ergiebig sind lassen sich hier meist die bisher gehaltenen Veranstaltungen finden. Ebenso kann man z.\,B. nach Skripten, Übungsblättern oder Klausuren des Bewerbers suchen.
	\item [\url{www.ratemyprofessors.com}]
		  Gerade bei ausländischen Bewerbern ist es es extrem schwierig an objektive Lehrgutachten zu gelangen. Zwar werden meist Ergebnisse aus Lehrevaluationen mit zu den Bewerbungsunterlagen gelegt, jedoch zeigen diese naturgemäß nur Positives. Ein Einstieg für die Recherche zu Bewerbern aus dem nicht deutschsprachigen Raum kann diese Webseite sein. Wir betonen aber, dass die Webseite keinesfalls repräsentativ ist und höchstens für den Einstieg Sinn macht.
	\item [\url{www.facebook.com}]
		  Wenn ihr direkt Kontakt zu Studierenden aufnehmen möchtet, welche gerade im Moment eine Vorlesung bei einem Bewerber belegen, so eignet sich Facebook oder ähnliches exzellent, um Kontakte zu solchen Teilnehmern zu erhalten. Bei diesen Anfragen solltet ihr aber bedenken, dass den Studierenden die grundsätzlichen Probleme beim Einholen solcher Informationen in der Regel nicht bekannt sind.
	\item [Fachschaft kontaktieren]
		  Hier sind E-Mail oder das Telefon das Medium der Wahl. Vorteil bei solchen Nachfragen ist, dass man gleich Informationen über das Engagement in der universitären Selbstverwaltung erhalten kann. Eine Muster-E-Mail für eine solche Anfrage findest du im Abschnitt~\ref{musteranfrage_fachschaft}.
	\item [Besuchen]
		  Falls ein Bewerber momentan an eurer Nachbarhochschule unterrichtet, so kann man auch einfach vorbeifahren und seine Lehre aus erster Hand erfahren. Dieses sollte möglichst nicht angekündigt werden.
	\item [Studentische Gutachten]
		  Bei ausländischen Bewerbern kann man darum bitten, studentische Gutachten über die Lehrqualität mit einzureichen. Im angelsächsischen Raum ist dieses Vorgehen üblich. Jedoch ist hier zu beachten, dass der Bewerber meist selber den Studierenden aussucht, der das Gutachten schreibt.
\end{description}

Für alle auftretenden rechtlichen Fragen verweisen wir auf Abschnitt~\ref{sec:rechtliches}.


\section{Die Vorträge an sich}
An jeder Hochschule gibt es Fachvorträge, in denen die Bewerber ihre aktuellen Forschungsgebiete präsentieren, ihre Erfolge loben und auch Ausblicke für ihre nächsten Ziele geben. Inhalt dieser Vorträge ist oft ein auch für Studenten nachvollziehbarer Teil. Dieser soll es ermöglichen, einen kleinen Einblick in die pädagogische Qualität des Bewerbers zu erhalten. Daneben gibt es an einigen Hochschulen zusätzlich noch Lehrproben, die also explizit die pädagogische Eignung der einzelnen Bewerber überprüfen. Leider ist dieses kein Standard und es tritt oft das Problem auf, dass die pädagogische Eignung allein durch den Fachvortrag überprüft werden muss. Wir gehen daher im Folgenden auf beide Fälle getrennt ein. 

Für jeder Art dieser Vorträge ist aber zu sagen, dass es für euch immer hilfreich ist, wenn weitere Studierende während der Vorträge anwesend sind und dich bei der Bewertung dieser Vorträge unterstützen\footnote{Beispielsweise kann man Fragebögen (siehe \ref{fragebogen}) austeilen, um Notizen bitten oder einfach nach dem Vortrag in die Runde fragen.}. Daher solltest du dich auch bemühen, die Vorträge möglichst gut bei den Studierenden in deinem Fachbereich publik zu machen. Wunderbar eignen sich dafür etwa Mailverteiler, Aushänge oder Ankündigungen in Vorlesungen.

\subsection{Lehrprobe}
Eine Lehrprobe ist das Halten einer Vorlesung vor einer genügend großen Anzahl an Studierenden im Rahmen eines Berufungsverfahrens. Insbesondere sind dieses Veranstaltungen aus dem Grundstudium, welche es auch dem studentischen Kommissionsmitglied ermöglichen, sich voll und ganz auf die Darstellung zu konzentrieren und weniger auf den mathematischen Gehalt. Eine Lehrprobe kann etwa als Vertretung einer Analysis oder Lineare Algebra Vorlesung gehalten werden aber auch als eigenständiger Lehrvortrag. Wir haben hier zwei Beispiele:

\subsubsection{Uni Flensburg}
\begin{itemize}
	\item Lehrprobe wird bei allen Bewerbern gefordert
	\item findet nach dem Fachvortrag statt
	\item Bewerber dürfen sich selber das Thema aussuchen, es muss aber mit der Ausrichtung der Stelle verbunden sein
	\item Zielgruppe sind Bachelor-Studenten im Hauptstudium
	\item Dauer: 45\,min Vortrag zzgl. 15\,min für Fragen
	\item im Durchschnitt sind 30 -- 40 Studierende anwesend
	\item Angekündigt wird von der Fachschaft in einzelnen Veranstaltungen und durch Aushang
\end{itemize}

\subsubsection{Uni Bremen}
\begin{itemize}
	\item pro Bewerber wird sich ein halber Tag Zeit genommen
	\item Dauer: 45\,min Probevorlesung 
	\item Inhalt des Vortrags
		\begin{itemize} 
			\item Früher: entsprechend der ausgeschriebenen Professur
			\item Derzeit: es werden zwei Themen zur Auswahl angeboten: "`Satz über impl. Funktion"' oder "`Satz von Picard-Lindelöff"'
			\item Zukünftig: Bewerber halten Ersatzvorlesung für Grundstudiumsvorlesung (Analysis 1/2 oder Lineare Algebra 1/2)
		\end{itemize}
	 \item am Ende werden zur Bewertung Fragebögen an zuhörende Studenten verteilt
\end{itemize} 


\subsection{Fachvortrag}
Sofern keine Lehrprobe gehalten wird, solltest du darauf bestehen, dass sich ein signifikanter Teil des Fachvortrages an Studierende richten muss. Hierbei ist ein \emph{signifikanter Teil} eines 45 Minuten Vortrages mindestens 15 Minuten lang\footnote{Und mit mindestens meinen wir wirklich, dass man keinesfalls darunter gehen darf.}. Zielgruppe sollten auch in diesem Bereich Studierende mit abgeschlossenem Grundstudium sein.

Du solltest außerdem darauf achten, dass der Vorsitzende der Kommission daran denkt, den Vortragenden diese Zeiteinteilung mitzuteilen. Dieses wird ab und an schon mal vergessen.

\subsection{Bewertung der Vorträge}
Der Vortrag/die Probevorlesung sind üblicherweise hochschulöffentlich, jeder Student kann sich
diese also anhören. Um bewerten zu können, wie gut der Vortrag ankommt, kann man die zuhörenden Studenten
im Anschluss befragen. Einen Musterfragebogen findest du in Anhang~\ref{fragebogen}.

Grundsätzlich gilt, dass du dir immer Notizen machen solltest. Du wirst im Folgenden noch öfters darauf angewiesen sein. Diese Notizen sollen aber nicht den Vortrag wiedergeben, sondern deine Eindrücke, wie der Vortragende seine Arbeit macht. Da dieses etwas ungewohnt ist, haben wir ein paar Punkte zusammengestellt, auf die man dazu achten kann:
\begin{itemize}
	\item Welche Medien verwendet der Bewerber und setzt er sie sinnvoll ein?
	\item Spricht der Bewerber dem Publikum zugewandt?
	\item Übersichtlichkeit der Darstellung (z.\,B. Farben, Handschrift, Folien)
	\item Wie geht der Bewerber auf Fragen ein?
	\item Wird abgelesen oder frei gesprochen?
	\item Wie ist die Motivation des Bewerbers und kann er motivieren?
	\item Artikulationsvermögen (insb. bei nicht Deutsch-Muttersprachlern)
	\item Geschwindigkeit des Vortrages
	\item Ist ein roter Faden erkennbar?
	\item Ist der Vortrag gut strukturiert?
\end{itemize}
Erfahrungsgemäß übersieht man inhaltliche Fehler, wenn man auf alle genannten Punkte achtet. Deswegen ist es sinnvoll, sich die Arbeit aufzuteilen, wenn man mit mehreren Studenten in der Kommission sitzt. Ansonsten werden inhaltliche Fehler aber in der Regel von den Professoren festgestellt.

\subsection{Fragen zum Vortrag}
Bereits während und auch direkt im Anschluss an die Fachvorträge werden üblicherweise Fragen gestellt. Auch ihr solltet diese Gelegenheit nutzen. Nützliche Fragen sind etwa:
\begin{itemize}
	\item Können sie \dots\ noch einmal erklären. Vielleicht für einen Studierenden im 3. Semester?
	\item Was sollte ich von diesem Vortrag mitnehmen? In zwei Sätzen bitte.
	\item Wie ist die Relevanz des Vortrages in den entsprechenden Themenbereich einzuordnen?
	\item Ggf. könnt ihr eine Frage auf Englisch stellen um das Englisch des Bewerbers zu testen.
\end{itemize}


\section{Bewerbungsgespräch}
Im Anschluss an die Vorträge finden üblicherweise die Bewerbungsgespräche zwischen den Bewerbern und der Kommission statt. Im Regelfall sind diese Gespräche vertraulich und hinter verschlossener Tür. D.\,h. ihr seid dann zur Verschwiegenheit verpflichtet und dürft nur mit (studentischen) Vertretern in höher rangigen Gremien über die Inhalte dieser Gespräche sprechen, welche auch Einblick in die internen Unterlagen der Berufungskommission haben.\footnote{An einzelnen Hochschulen sind diese Gespräche dahingegen in die Hälften Forschung und Persönliches geteilt. Hierbei ist dann der erste Teil hochschulöffentlich.}

Der Vorsitzende der Kommission leitet üblicherweise das Gespräch und wechselt dabei auch zwischen den Themenbereichen. Dieses sorgt dafür, dass der Bewerber nicht 90\% der Zeit über sein Lieblingsgebiet sprechen kann. Irgendwann wird der Vorsitzende dann zur Lehre überführen, spätestens hier solltest du deine Fragen stellen. Zur besseren Vergleichbarkeit solltet ihr darauf achten, jedem Bewerber die gleichen Fragen zu stellen. 

Sollte die Zeit knapp werden und wurde immer noch nicht über die Lehre gesprochen, solltest du die Initiative ergreifen und versuchen in diese Richtung zu lenken.

Wir haben im Folgenden einige Fragen gesammelt, aus welchen du gerne auswählen darst oder von denen du dich bei deinen eigenen Fragen inspirieren lassen kannst.

\subsection{Fragenkatalog: allgemeine Fragen}
\begin{itemize}
	\item Was bedeutet für Sie gute Lehre?
	\item Haben Sie sich unsere Bachelor- und Masterprüfungsordnung angeschaut? Kennen Sie sich damit aus?\footnote{Da Professoren erwarten, dass sich Bewerber mit den Forschungsschwerpunkten des Institutes beschäftigt haben, können wir dieses auch erwarten.}
	\item Wie würden Sie Studiengebühren verwenden? Wie ist Ihre Meinung zu dieser Thematik?
	\item Wie stellen Sie sich ihren Übungs-/Vorlesungsbetrieb vor?
	\item Was sollte ein Bachelor können, wenn er fertig ist? (und Vergleich zu Lehrämtlern)
	\item Wie würden Sie persönlich die Bereiche Verwaltung, Lehre und Forschung gewichten?
	\item Haben Sie Erfahrungen in der akademischen Selbstverwaltung?
	\item Wie stellen Sie sich die Zusammenarbeit mit der Fachschaft vor?
	\item Differenzieren Sie zwischen Lehramts- und Vollfachstudenten? Wie gehen Sie auf die besonderen Bedürfnisse der Lehramtsstudenten ein?
	\item Haben Sie schon an hochschuldidaktischen Fortbildungen teilgenommen?
	\item Warum wollen Sie genau an diese Hochschule?
	\item In einer Veranstaltung fallen 80\% der Studenten durch die Klausur. Wie gehen Sie damit um?
	\item Wie stellen Sie sich den Übungsbetrieb vor? (Präsenzübungen vs. Vorrechnen)
	\item Wie können Studenten die Fragen haben Sie erreichen? Wie viele Tage pro Woche sind Sie an der Hochschule? Können Studenten auch außerhalb der Sprechzeiten zu Ihnen kommen?
	\item Was sind ihre Stärken und Schwächen?
	\item Wie unterscheidet sich bei Ihnen eine Vorlesung mit 12 Studierenden von einer mit 400 Studierenden? Wie gehen Sie mit dem höheren Lärmpegel um?
	\item Nennen Sie spontan je ein Thema für eine Bachelor-, Master- und Staatsexamensarbeit.
	\item Welche Vorlesungen wollen/werden Sie anbieten? Was ist ihr Kanon? Werden Sie eine Spezialisierungssequenz anbieten?
	\item Wie bereiten Sie ihre Vorlesungen vor? Gibt es bei Ihnen ein Skript?
	\item Können Sie interessante Themen für Vorlesungen anbieten, die sich in erster Linie an Lehrämtler richten?\footnote{Dieses sollten interessante und lebensnahe mathematische Themen sein, mit denen man auch mal für ein paar Schüler beeindrucken kann. Jeder Matheprof sollte so etwas in petto haben (denn wer will sich schon mit Numerik VI rumschlagen?)}
\end{itemize}

\subsection{Fragenkatalog: bei Bedarf}
\begin{itemize}
	\item Wie viele/welche Vorlesungen haben Sie schon gehalten? (sofern nicht in Unterlagen angegeben)
	\item Gab es bei Ihnen eine Vorlesungsevaluation und können Sie uns diese zuschicken? Wie haben Sie dort abgeschnitten und was halten Sie von solchen Befragungen? (wenn nicht bereits offiziell vorhanden)
	\item Auf spezielle Tätigkeiten eingehen, wenn diese z.\,B. auf der Webseite des Bewerbers gefunden werden konnten (z.\,B. Prüfungsausschuss).
\end{itemize}


\section{Entscheidung vs. Gutachten}
Nach der Vorstellung der Bewerber entscheidet die Kommission, welche Bewerber weiter berücksichtigt werden sollen. Über diese werden bei verschiedenen namhaften Persönlichkeiten aus dem jeweiligen Bereich Gutachten über die Qualität der Bewerber eingeholt. Das Einholen dieser Gutachten dauert bis zu zwei Monate. 

Sobald alle Gutachten eingetroffen sind, wird auf Grundlage dieser Gutachten eine Berufungsliste erstellt. Diese Liste hat in der Regel drei Plätze, wobei einzelne Plätze aber auch (wenn auch sehr selten) doppelt besetzt werden können. Sofern diese Liste durch die weiteren Gremien bestätigt wird, erteilt irgendwann der Präsidenten bzw. Rektor einen Ruf an den Erstplatzierten. Wenn dieser nicht annimmt, dann geht ein Ruf an den Zweitplatzierten usw.

Im Idealfall sollte diese Entscheidung aufgrund der Gutachten sehr einfach sein, indem optimalerweise alle Gutachter eine gleiche Reihung vorschlagen. In der Praxis erhält man aber teils widersprüchliche Aussagen, merkt dass Gutachter den Ausschreibungstext falsch verstanden haben oder kann klar aus dem Gutachten herauslesen, dass der Gutachter mit genau einem Kandidaten persönlich gut befreundet ist. Alles das muss in dieser Sitzung abgewogen werden. Die rechtlichen Rahmenbestimmungen werden bereits durch die Professoren sichergestellt. Anhand der Gutachten und deiner Eindrücke (die durch Notizen belegt sind) in den Probevorträgen bist du aber herzlich dazu eingeladen mitzudiskutieren wen du als passende Besetzung für die ausgeschriebene Stelle siehst.

Die fachliche Eignung der Kandidaten und die Passung in das Institut kannst Du als Student meist wenig bis kaum beurteilen. Daher solltest Du diese Kriterien den Professoren überlassen. Deine Stärke liegt in der Beurteilung der Lehrqualität.

Eines solltest du hierbei immer im Hinterkopf behalten: es kann/sollte/darf natürlich nur jemand berufen werden, der auch vernünftige Lehre macht. Dieses ist deine Hauptaufgabe und du bist gefragt um dieses sicherzustellen und ggf. dafür zu sorgen, dass die pädagogisch schwächeren Kandidaten auf die hinteren Listenplätze wandern oder ganz von der Liste fallen. Im Zweifelsfall ist aber oft noch ein Sperrvermerk herauszuhandeln. Du solltest übrigens auch darauf vorbereitet sein, dass du vielleicht schon zu Beginn der Sitzung gefragt wirst: "`Wie würden denn Sie aus studentischer Sicht die Bewerber gewichten?"'


\subsection{Begutachtung der Gutachten}
Dieser Abschnitt soll dir ein paar Hilfestellungen geben, wie man pro oder contra bestimmter Bewerber argumentieren kann.

Auch wenn in machen Gutachten steht, dass die Lehrqualität des Bewerbers hervorragend ist, so sind diese Gutachten immer mit Vorsicht zu genießen. Oft haben die Gutachter die einzelnen Bewerber nur auf Tagungen vortragen gesehen und kennen daher nicht die Qualität ihrer Vorlesungen.

\paragraph{Über die Gutachten}
\begin{itemize}
	\item Es gibt \emph{Einzelgutachten}, welche nur einen Bewerber begutachten. Diese fallen üblicherweise sehr gut bis hervorragend aus. Einzig ein neutrales bis negatives Einzelgutachten kann dir zu Entscheidungsfindung dienen.
	\item \emph{Vergleichende Gutachten} sind Gutachten, die mehrere der Bewerber vergleichen. Hierbei gibt der Gutachter meist auch seine persönliche Reihung an. Dieses sind mit die wichtigsten Entscheidungshilfen.
	\item Wenn ein Bewerber \emph{nicht habilitiert} ist muss im Gutachten stehen, dass er zu eigenständiger Lehre befähigt ist bzw. alle Voraussetzungen für eine Professur erfüllt.
	\item Die didaktische Eignung eines Bewerbers kann nur in den seltensten Fällen mit einem Standardgutachten begründet werden.
	\item Von den Bewerbern selbst vorgeschlagene Gutachter sollte mit Vorsicht genossen werden. Dieses sind meist Kandidaten für das Schreiben von Einzelgutachten. Einzelne Berufungsordnungen schließen von den Bewerbern vorgeschlagene Gutachter grundsätzlich aus.
	\item Jeder Bewerber sollte von wenigstens zwei Gutachtern begutachtet werden, davon mindestens bei einem Gutachter vergleichend.
\end{itemize}

\paragraph{Argumentation gegen einen Bewerber}
\begin{itemize}
	\item Anhand des Kriterienkatalogs (hier muss pädagogische Eignung drin stehen, vgl. Seite~\pageref{kriterienkatalog}).
	\item Ggf. hat der Bewerber keine Habilitation.
	\item Notizen aus Vortrag und Gespräch.
	\item Bewerber hat nur geringe Lehrerfahrung.
	\item Je nachdem was im Gutachten steht: Gutachten kann die Lehrqualität nicht beurteilen bzw. Gutachter bescheinigt sehr gute Lehre.
	\item Schlechter Eindruck bei Studierenden (z.\,B. durch Umfrage beim Vortrag).
	\item Evaluationsergebnisse.
	\item Auf formale Fehler achten.
	\item Mit anderen Kommissionsmitgliedern und Sonderbeauftragen (Gleichstellungsbeauftragte etc.) zusammenschließen.
\end{itemize}

\paragraph{Argumentation für einen Bewerber}
\begin{itemize}
	\item Kriterienkatalog: wenn eine gute Lehrerfahrung vorliegt.
	\item Ggf. haben andere Bewerber keine Habilitation.
	\item Notizen aus Vortrag und Gespräch.
	\item Gute Lehrerfahrung, insbesondere auch mit großen Vorlesungen.
	\item Eindruck von Studierenden (z.\,B. durch Umfrage bei Vortrag).
	\item Evaluationsergebnisse.
	\item Gegen die anderen Bewerber diskutieren.
	\item Je nachdem was im Gutachten steht: Gutachten kann die Lehrqualität nicht beurteilen bzw. Gutachter bescheinigt sehr gute Lehre.
\end{itemize}

Jedoch ist es so, dass auch ein hervorragender Hochschullehrer nur dann auf eine Stelle gesetzt werden kann, wenn er in das jeweilige Forschungsprofil passt. Wenn du merkst, dass ein Bewerber nicht auf die Stellenausschreibung passt, dann ist es nicht sinnvoll weiter für ihn zu kämpfen.

Weiter ist noch zu sagen, dass die Pausen hervorragend für Mauscheln und Kontakte knüpfen geeignet sind. Dort kann man manchmal in einem netten Plausch schneller zum Ziel kommen als direkt während der Sitzung. Je nachdem wie gut du mit den Professoren klar kommst, kannst du diese auch in die ein oder andere Richtung lotsen. Insbesondere kannst du in festgefahrenen Situation versuchen einen Kompromiss vorzuschlagen. 

\subsection{Die Abstimmung}
Der Ablauf der Abstimmung ist üblicherweise sehr genau durch die Berufungsordnung geregelt. Beispielsweise wird meist ein doppeltes Quorum (Mehrheit der Professoren und Mehrheit der Mitglieder) gefordert. 

Fast immer ist die endgültige Abstimmung über die Bewerber geheim. Dazu solltest du dich auch auf jeden Fall mit deiner Berufungsordnung auseinander setzen. Gerade wenn du überlegt gegen den von der Kommission favorisierten Kandidaten zu stimmen, solltest du dir das genau überlegen. Es ist meistens auch sinnvoll, schon während der Beratung deutlich die Vorbehalte vorzubringen. Wie du letztendlich über die Bewerber abstimmst ist aber natürlich eine Gewissensangelegenheit. Ob du dagegen, dafür oder mit Enthaltung stimmst, solltest du dir situationsabhängig überlegen.

Die Professoren möchten üblicherweise ein einstimmiges Ergebnis, um den weiteren Weg des Berufungsverfahrens durch die einzelnen Instanzen einfacher zu gestalten. Daher hast du als Student ein gewisses "`Druckmittel"'. Vermutlich wirst du den ersten Listenplatz damit nicht ändern können, aber weiter hinten auf der Liste kannst du so evtl. ein pari passu oder einen Sperrvermerk erreichen.


\subsection{Das Sondervotum}
Wenn du gegen einen Kandidaten stimmst, er aber trotzdem auf die Liste kommt, so hast du üblicherweise das Recht ein Sondervotum abzugeben. Das bedeutet, du kündigst (je nach Berufungsordnung) unverzüglich nach der Abstimmung an, dass du ein Sondervotum einreichen möchtest und reichst dieses in der gegebenen Frist ein. 

Der Vorteil eines Sondervotums ist der, dass dieses durch alle weiteren Gremien wandert und du so die Möglichkeit hast, deine Bedenken begründet weiter zu geben. Ebenso solltest du in einem solchen Fall auch direkt die studentischen Vertreter in den nächst höheren Gremien (Fachbereichs-/Fakultätsrat, Senat) informieren, damit diese mit einem gewissen Hintergrundwissen diskutieren können.

Jedoch solltest du dir vor einem Sondervotum stets darüber klar sein, dass dieses eine sehr deutliche bzw. extreme Meinungsäußerung ist und bei einzelnen Mitgliedern der Kommission für Unmut sorgen könnte.

\section{Studentisches Votum}
In einigen Bundesländern hast du die Möglichkeit bzw. die Aufgabe ein \emph{Studentisches Votum} einzureichen. In diesem Votum sollen die Listenplätze aus studentischer Sicht und vor allem unter der Frage der Lehrqualität verglichen und diskutiert werden.

Wenn du ein solches Votum schreiben musst, sollte das Folgende darin vorkommen:
\begin{itemize}
	\item Diskussion der einzelnen Bewerber aus studentischer Sicht (Qualität des Vortrags).
	\item Inhalt sollte ein "`Votum über die Lehrleistung der Kandidaten"' sein.
	\item Jeder Bewerber sollte in ein paar Sätzen einzeln diskutiert werden.
	\item Abschließend sollte ein Vergleich aller Bewerber erfolgen.
\end{itemize}

Es ist sinnvoll zuerst zu beschreiben, was Grundlage deiner Bewertung ist. Wenn Du für den Vortrag/Probevorlesung Fragebögen verteilt hast, dann beschreibe kurz, wie dieser aussah. Erläutere auch, zu welchen Themen du Fragen in den Interviews gestellt hast. Bei der Bewertung der einzelnen Bewerber braucht man nicht in "`Arbeitszeugnisdeutsch"' zu verfallen. Statt "`Er hat sich bemüht eine gute Vorlesung zu halten."' kannst Du einfach schreiben, dass die Zuhörer die Vorlesung nicht gut fanden. Natürlich sollte kein Bewerber zerrissen werden, aber wenn jemand eine schlechte Vorlesung gehalten hat, solltest du das auch zum Ausdruck bringen. 

Zum Abschluss des Votums solltest du dich dann dazu äußern, ob du mit der Reihenfolge der Berufungsliste
einverstanden bist, oder Vorbehalte gegen diese hast. Vorbehalte solltest du dann ausführlich begründen.

\section{Der weitere Gang}
Nach der Abschlusssitzung wandert die Vorschlagsliste weiter an die höhere Gremien. Deine Arbeit endet aber noch nicht ganz. Falls es in der Kommission etwa sehr kontroverse Abstimmungen gab oder gar ein Kandidat auf die Liste genommen wurde, welchen du für nicht listenfähig hältst, so solltest du dich als nächstes an deine studentischen Vertreter in den nächst höheren Gremien wenden. Nur durch deine Informationen können diese in den folgenden Diskussionen nämlich die Probleme korrekt wiedergeben, die du bei den Kandidaten gesehen hast.

Sollte die Liste dann irgendwann beschlossen sein und ein Kandidat tatsächlich den Ruf annehmen, dann ist die Kommissionsarbeit schließlich beendet. Es kann jedoch noch der Fall eintreten, dass eine \emph{pari passu} besetzte Stelle noch einmal zur Stellungnahme an die Kommission zurück gegeben wird.

Sobald nun aber ein Kandidat zugesagt hat oder die Liste erschöpft ist und kein Bewerber gefunden wurde, endet tatsächlich deine Arbeit in der Kommission. Zu guter Letzt wäre jetzt der Zeitpunkt gekommen, zu dem du deine Erfahrungen (soweit möglich) für die Fachschaft dokumentierst.


%%%%%%%%%%%%%%%%%%%%%%%%%%%%%%%%
%%% KAPITEL 4: KONTAKT
%%%%%%%%%%%%%%%%%%%%%%%%%%%%%%%%

\chapter{Kontakt mit anderen Fachschaften}\thispagestyle{fancy}
Es liegt in der Natur einer Berufungskommission, dass die Bewerber fast immer von einer anderen Hochschule kommen. Aus diesem Grunde ist der Kontakt zu anderen Fachschaften wichtig. Denn meist hat die Heimfachschaft eines Bewerbers einen guten Überblick über diese Person oder kennt zumindest jemanden, der schon einige Vorlesung bei diesem Bewerber gehört hat.

Hierbei sind neben rechtlichen Fragen auch die Möglichkeiten des Austausches und die Form der Anfragen und Antworten zu diskutieren. Dieses wollen wir in diesem Kapitel versuchen.


\section{Rechtliches}\label{sec:rechtliches}
Gerade bei Anfragen an andere Fachschaften herrscht oft große Rechtsunsicherheit. Folgendes können wir zu den häufigen Fragen sagen:
\begin{itemize}
	\item Eine Anfrage bei anderen Fachschaften zu der Lehrqualität eines Bewerbers ist i.\,d.\,R. rechtlich unbedenklich\footnote{Dabei sollte man vorsichtshalber nicht erwähnen, dass es sich bei der Person um einen Bewerber handelt (wobei sich das der anderen Fachschaft aus dem Kontext ergeben wird). Mittlerweile
	kann man im Internet auch Listen finden, auf denen aktuelle Berufungsverfahren inklusive Bewerber und Listenplatzierte aufgeführt werden.}. Auch das Einholen der Evaluationen eines Bewerbers ist hier möglich. Die Herausgabe muss aber unter Hinblick des Datenschutzes erfolgen und wird daher bei internen Evaluationen nur mit Zustimmung des Bewerbers möglich sein.
	
	\item Der Höflichkeit halber wäre es angebracht, den Bewerber zu fragen, ob er etwas dagegen habe, dir die Evaluationen zusenden zu lassen.
	
	\item Wenn du Informationen oder Evaluationsergebnisse einer anderen Fachschaft erhalten hast, so sind das zunächst einmal deine persönlichen Informationen. Wenn du diese Informationen in das Berufungsverfahren einbringen möchtest, so ist der rechtlich unbedenklichste Weg, wenn du dem Bewerber während des Interviews passende Fragen stellst.

	\item In die andere Richtung sind Informationen aus der Berufungskommission heraus dann unbedenklich, wenn sie während eines öffentlichen Teils bekannt geworden sind. So sind alle Informationen, welche die Berufungskommission während der öffentlichen Probevorlesungen/Fachvorträge gewinnt unproblematisch. Auch Informationen aus den öffentlichen Sitzungen der Berufungskommission sind hier nicht als bedenklich einzustufen. Nicht zulässig wäre aber eine Weitergabe aus nichtöffentlichen Teilen, insbesondere aus Interviews zur außerfachlichen Qualifikation, Zitaten aus Gutachten, etc.
	
	\item Generell darf aber jeder seinen persönlichen Eindruck von Bewerbern wiedergeben.
\end{itemize}


\section{Du möchtest eine Anfrage schreiben}
\label{musteranfrage_fachschaft}
Eine Anfrage kannst du im deutschsprachigen Raum am einfachsten per E-Mail stellen. Ein entsprechendes Beispiel findest du im Anhang auf Seite~\pageref{musteranfrage_txt}. Im Regelfall solltest du die entsprechende Adresse ganz einfach mit der Internetsuchmaschine deiner Wahl finden. Falls dieses nicht möglich ist könntest du aber Folgendes versuchen:
\begin{itemize}
	\item Frag im Büro deiner Bundesfachschaftenkonferenz nach der Adresse. Wir von der KoMa haben beispielsweise ein Adressverzeichnis aller deutschsprachiger Mathematikfachschaften.
	\item Ansonsten kannst du auch im Adressreader des fzs nachschauen: \url{http://www.adressreader.de/}. Dieser wird vom \emph{freier zusammenschluss von studentInnenschaften e.\,V.} angeboten und ist online kostenlos verfügbar.
\end{itemize}
Sollte ein Kontakt per E-Mail nicht möglich sein kann man übrigens auch die Post-Methode versuchen. Diese kann in solchen Fällen gegebenenfalls zum Erfolg führen.


\section{Deine Fachschaft erhält eine Anfrage}
Auch wenn du momentan in keiner Berufungskommission sitzt, so kann es sein, dass die Anfrage einer anderen Fachschaft zu einem deiner Dozenten bei euch ankommt. Wenn dieses geschieht, so bitten wir dich bei der Antwort folgendes zu berücksichtigen:
\begin{itemize}
	\item Beantworte alles so, wie du es selbst verantworten kannst und es selbst von jeder anderen Fachschaft erwarten würdest.
	\item Entweder \emph{antworte ehrlich} oder antworte, dass du \emph{dich nicht äußern möchtest}.
	\item Wenn du nicht schriftlich antworten möchtet, aber mündlich Fragen beantwortet würdest, so teil dieses bitte der fragenden Fachschaft mit, ggf. auch eine Kontaktperson.
	\item Wenn du Bedenken bei deiner Antwort hast, so schreibt dieses ruhig in die Antwort hinein -- Es sollte sowieso klar sein, dass diese Antwort ausschließlich für das studentische Mitglied in der Kommission bestimmt ist. Es ist auch kein Problem, wenn du anonym bleiben möchtest.
	\item Wenn rechtlich möglich, schick bitte Evaluationen mit.
	\item Bitte suche jemanden, der bei dem Bewerber bereits Vorlesungen/ Seminare besucht hat und konkret etwas zu
	\begin{enumerate}
		\item Vorbereitung des Dozenten,
		\item Tafelanschrieb,
		\item Struktur der Vorlesung und
		\item Existenz bzw. Qualität des Skripts
	\end{enumerate}
	schreibt.
	\item Schreib bitte über deine Erfahrungen mit dem Dozenten/das Klima in der Vorlesung/ die Erreichbarkeit außerhalb der Vorlesungen und etwas zu den Sprechzeiten.
	\item Wenn möglich, schreib bitte auch etwas zur Erfahrung mit dem Dozenten in der universitären Selbstverwaltung.
\end{itemize}



%%%%%%%%%%%%%%%%%%%%%%%%%%%%%%%%
%%% ANHANG
%%%%%%%%%%%%%%%%%%%%%%%%%%%%%%%%


\appendix
\chapter{Anhang}\thispagestyle{fancy}

\section{Checkliste}
Hier sind ein paar Punkte, die man bei einer Berufungskommission abhaken sollte:
\begin{itemize}
	\item [\Square] Unterlagen gesichtet
	\item [\Square] Anfragen an Fachschaften herausgeschickt
	\item [\Square] bei fehlenden Antworten nachgehakt
	\item [\Square] Fragen für das Bewerbungsgespräch vorbereitet
	\item [\Square] Bewerbungsvortrag bei Studierenden angekündigt
	\item [\Square] ggf. Fragebögen für den Bewerbungsvortrag vorbereitet
	\item [\Square] Gutachten gelesen
	\item [\Square] Deine Kandidatenreihung erstellt
	\item [\Square] Votum geschrieben
	\item [\Square] ggf. Studentisches Votum pünktlich eingereicht
\end{itemize}
Es gibt zwar noch viel mehr zu tun, aber dieses sind oft die zeitkritischen Punkte, die man leicht vergessen kann.


\section{Eine Beispielausschreibung}
Dieses ist eine Ausschreibung für eine W2-Professur Zahlentheorie an einer NRW-Universität. Wenn du weitere Ausschreibungen sehen möchtest, dann schau doch einfach mal in \emph{Die Zeit}.  In dieser Wochenzeitung werden eigentlich alle in Deutschland ausgeschriebenen Professuren veröffentlicht.\bigskip 

\noindent\fbox{\begin{minipage}{.95\textwidth}\vspace{0.4cm}
Im Institut für Mathematik der Fakultät \dots ist eine W2-Professur für Zahlentheorie zum \dots zu besetzen. Die zu berufende Persönlichkeit soll ein aktuelles Forschungsgebiet im Bereich Zahlentheorie vertreten. Erwartet wird aktive Kooperationsbereitschaft bei Aufbau und Fortführung von Forschungsschwerpunkten und Graduiertenkollegs des Instituts. Erwünscht ist außerdem eine Zusammenarbeit mit der Theoretischen Informatik.

Der Stelleninhaber/ die Stelleninhaberin soll sich maßgeblich an den Lehraufgaben des Faches Mathematik beteiligen und soll insbesondere auch im Service (vor allem für Informatik, Ingenieur- und Naturwissenschaften) mitwirken.

Einstellungsvoraussetzungen: \S~46 Abs.\,1 Ziff.~4a HG NW (Habilitation oder habilitations-adäquate Leistungen) und Ziff. 2 HG NW (pädagogische Eignung).

Die Universität \dots strebt eine Erhöhung des Anteils der Frauen als Hochschullehrerinnen an und fordert daher qualifizierte Wissenschaftlerinnen nachdrücklich zur Bewerbung auf. Frauen werden nach \S~7 LGG bei gleicher Eignung, Befähigung und fachlicher Leistung bevorzugt berücksichtigt.
Schwerbehinderte erhalten bei gleicher Eignung den Vorrang.

Bewerbungen mit den üblichen unterlagen werden innerhalb von vier Wochen nach Veröffentlichung
unter Angabe der Kennziffer \dots erbeten an den Leiter des Instituts für Mathematik, \dots
\vspace{0.4cm}
\end{minipage}
}
\vfill
\newpage
\section{Muster-Fragebogen für Probevorlesungen}\label{fragebogen}

\noindent\begin{minipage}{\textwidth}
\begin{center}
	\Large 
	Probevorlesung\smallskip\par
	\large
	Professur \underline{\hspace{6cm}}
\end{center}

\parbox{\textwidth}{\large Ich habe folgende Probevorlesungen besucht:}\smallskip\par
\begin{tabular}{|p{4.5cm}p{4cm}|p{1.5cm}|}
\hline
%\multicolumn{2}{|l|}{\cellcolor{light}}\\
\rowcolor{light}
Kandidat & Thema & Besucht? \\
\hline
  & & ja/nein\\\hline
  & & ja/nein\\\hline
  & & ja/nein\\\hline
  & & ja/nein\\\hline
\end{tabular}
\bigskip

\parbox{\textwidth}{\large Deine Bewertung:}\smallskip\par
\begin{tabular}{|m{61mm}|m{20mm}|m{20mm}|}\hline
				\rowcolor{light}\hline
								
        \raisebox{6mm}[9mm]{\footnotesize{\tiny Name des Vortragenden}} {}&
				\raisebox{6mm}[9mm]{\footnotesize{\tiny Datum}}{}&
        \raisebox{6mm}[9mm]{\footnotesize{\tiny Dein Fachsemester}}\\
        \hline
\hline
\raisebox{1.5mm}[6mm]{War das Thema interessant?} &\multicolumn{2}{|m{40mm}|}{\raisebox{0mm}[6mm]{\bewertung}}\\\hline
\raisebox{1.5mm}[6mm]{War der Vortrag verständlich?} & \multicolumn{2}{|m{40mm}|}{\raisebox{0mm}[6mm]{\bewertung}}\\\hline
\raisebox{3mm}[9mm]{\parbox{60mm}{Ist der Vortragende gut auf die gestellten Fragen eingegangen?}} & \multicolumn{2}{|m{40mm}|}{\raisebox{0mm}[6mm]{\bewertung}}\\\hline
\raisebox{4mm}[12mm]{\parbox{60mm}{Hat sich der Vortragende angemessen auf den Kenntnisstand der Zuhörer eingestellt?}} & \multicolumn{2}{|m{40mm}|}{\raisebox{0mm}[6mm]{\bewertung}}\\\hline
\raisebox{3mm}[9mm]{\parbox{60mm}{Würdest Du gerne eine Vorlesung bei dem Kandidaten hören?}} & \multicolumn{2}{|m{40mm}|}{{\raisebox{0mm}[6mm]\bewertung}}\\\hline
\multicolumn{3}{|l|}{\raisebox{26mm}[30mm]{Mögliche weitere Kommentare zum Vortrag oder zum Kandidaten:}}\\\hline

\end{tabular}
\end{minipage}

\section{Eine negative Stellungnahme}
\begin{center}
\fbox{\begin{minipage}{.95\textwidth}\vspace{0.4cm}
\begin{center}{\large  Studentisches Votum zur Besetzung der Professur}\smallskip\\
{\Large W3 Nachfolge Prof. Guru}
\end{center}\bigskip

Sehr geehrte Damen und Herren,\medskip\\

Als studentisches Mitglied der Berufungskommission zur
Besetzung einer W3-Professur (Kennziffer xyz, Nachfolge Guru) im Institut
für Mathematik, möchte ich die von der Kommission verabschiedete Liste nicht unterstützen.\smallskip\par

Meine folgenden Einschätzungen zu den einzelnen Bewerbern beziehen sich
ausschließlich auf deren didaktische und methodische Fähigkeiten.
Der Vortrag von Herrn Fermat machte auf mich einen gut strukturierten und
durchdachten Eindruck, allerdings hätte ich mir zusätzlich Beispiele und/oder
Anwendungsbezüge von ihm gewünscht. Ich habe Zweifel, dass er in Vorlesungen
besser auf seine Zuhörer eingehen kann, insbesondere da er bislang noch keine
Erfahrungen mit Anfängerveranstaltungen, sog. Serviceveranstaltungen oder mit speziellen
Veranstaltungen für Lehramtsstudierende, welche einen großen Anteil der
Lehrtätigkeit dieser Professur darstellen, gesammelt hat.
Herr Cauchy bot meiner Meinung nach einen gut strukturierten Vortrag mit
Einbindung von Beispielen. Auch der Vortrag von Herrn Weierstraß enthielt
Beispiele und Erklärungen, was auf mich einen positiven Eindruck machte.\smallskip\par

Generell habe ich den Eindruck, dass im Entscheidungsprozess der Kommission die
Lehrqualifikation der Bewerber im Vergleich zum Forschungsgebiet und den
Kooperationsmöglichkeiten in der Mathematik an der Universität ABC zu wenig
berücksichtigt wurde. In der konstituierenden Sitzung der Kommission wurde allerdings die Lehrqualifikation neben der Forschungsqualifikation als höchstes Bewertungskriterium beschlossen.\smallskip

Aus diesem Grund möchte ich die von der Kommission verabschiedete Liste nicht unterstützen.
\vspace{0.4 cm}
\end{minipage}
}
\end{center}

\section{Eine positive Stellungnahme}
\begin{center}
\fbox{\begin{minipage}{.95\textwidth}\vspace{0.4cm}
\begin{center}{\large Studentisches Votum zur Besetzung der Juniorprofessur}\smallskip\\
{\Large Programmieren für Anfänger (Kennz. 123)}
\end{center}\bigskip

Sehr geehrte Damen und Herren,\medskip\\

Als studentisches Mitglied unterstütze ich die von der Berufungskommission aufgestellte
Vorschlagsliste ausdrücklich. Dies begründe ich wie folgt:\smallskip

Dr. Java hielt einen gut verständlichen und (soweit ich das beurteilen kann) auch tiefgehenden
Vortrag. Im Gespräch zeigte er sich an der mit der Juniorprofessur verbundenen Lehre
außerordentlich interessiert. Dieser Eindruck wird durch seine bisherige Erfahrung in der Lehre an
der University of Somewhere unterstützt. Ich erwarte, dass seine Veranstaltungen eine Bereicherung
des Lehrangebotes am Institut Mathematik der Universität ABC sein werden.\smallskip

Dr. C überzeugte ebenfalls durch einen gut verständlichen, exzellent strukturierten Vortrag. Zwar ist
seine Lehrerfahrung nicht mit der von Dr. Java zu vergleichen, er zeigte sich jedoch während seiner
Promotionszeit überdurchschnittlich engagiert, was die Lehre betrifft, und konnte diesen Eindruck
auch im Gespräch bestätigen. Zudem ist zu erwarten, dass er seine Erfahrungen in der Wirtschaft in
positiver Weise in die Lehre einbringen kann.\smallskip

Ich bin der Überzeugung, dass beide Kandidaten die mit der zu besetzenden Juniorprofessur
einhergehenden Aufgaben in Lehre und Forschung in vorbildlicher Weise erfüllen werden.


\vspace{0.5 cm}

Mit freundlichen Gr"u"sen\par
\hspace*{2cm}Klaus Mustermann\\
\vspace{0.4cm}
\end{minipage}
}
\end{center}

\section{Musteranfrage}\label{musteranfrage_txt}
\begin{center}
\parbox{0.9\textwidth}{\ttfamily
Liebe Kommilitonen,\medskip\\
wir hätten mal eine inoffizielle (!) Anfrage zu einem Dozenten an eurer Uni! Und zwar handelt es sich um XXX.\medskip\\
Was wir gerne von euch wüssten, wären folgende Punkte (in der Hoffnung, dass einer von euch schon mal eine Vorlesung bei ihm gehört hat):\smallskip\\
- Geht er auf Fragen/Wünsche von Studenten ein?\\
- Kann man ihm in der Vorlesung folgen?\\
- Ist das Tafelbild gut strukturiert?\\
- Wie ist das Klima/die Stimmung in der Vorlesung?\\
- Was haltet ihr insgesamt von seiner Lehre?\\
- Wie ist sein Verhältnis zu den Studierenden?\\
- Steht seine Tür jederzeit offen (wenn man Probleme/Fragen hat)?\\
- Engagiert er sich bei Veranstaltungen, z.B. für Schüler, Tag der Mathematik o.\,ä.?\\
- Gibt es irgendetwas, was man sonst von ihm wissen sollte, positiver sowie negativer Natur?\smallskip\\
Es würde uns freuen, wenn wir dazu ein paar Infos von euch haben könnte! Alles wird natürlich vertraulich behandelt, so wie hoffentlich auch diese Anfrage! Wenn ihr irgendwelche Evaluationen von ihm habt, wäre das natürlich auch sehr hilfreich.\medskip\\
Danke schon mal!\medskip\\
Liebe Grüße,\\
X Y\ \ (Mathematik Fachschaft ABC)
}
\end{center}
\newpage
\pagestyle{empty}~
\newpage
\pagestyle{empty}~
\vfill
\centering\url{http://www.die-koma.org}


\end{document}