\section{Studentisches Votum}
In einigen Bundesländern hast du die Möglichkeit bzw. die Aufgabe ein \emph{Studentisches Votum} einzureichen. In diesem Votum sollen die Listenplätze aus studentischer Sicht und vor allem unter der Frage der Lehrqualität verglichen und diskutiert werden.

Wenn du ein solches Votum schreiben musst, sollte das Folgende darin vorkommen:
\begin{itemize}
    \item Diskussion der einzelnen Bewerber aus studentischer Sicht. Dies kann beinhalten:
    \begin{itemize}
        \item Qualität des Vortrags,
        \item Lehrkonzepte aus der Bewerbung,
        \item und Eindrücke aus dem studentischen Gespräch mit dem Bewerber.
    \end{itemize}
    \item Jeder Bewerber sollte in ein paar Sätzen einzeln diskutiert werden.
    \item Abschließend sollte ein Vergleich aller Bewerber erfolgen.
\end{itemize}

Es ist sinnvoll zuerst zu beschreiben, was Grundlage deiner Bewertung ist. Wenn Du für den Vortrag/Probevorlesung Fragebögen verteilt hast, dann beschreibe kurz, wie dieser aussah. Erläutere auch, zu welchen Themen du Fragen in den Interviews gestellt hast. Bei der Bewertung der einzelnen Bewerber braucht man nicht in "`Arbeitszeugnisdeutsch"' zu verfallen. Statt "`Er hat sich bemüht eine gute Vorlesung zu halten."' kannst Du einfach schreiben, dass die Zuhörer die Vorlesung nicht gut fanden. Natürlich sollte kein Bewerber zerrissen werden, aber wenn jemand eine schlechte Vorlesung gehalten hat, solltest du das auch zum Ausdruck bringen.

Zum Abschluss des Votums solltest du dich dann dazu äußern, ob du mit der Reihenfolge der Berufungsliste
einverstanden bist, oder Vorbehalte gegen diese hast. Vorbehalte solltest du dann ausführlich begründen.

%%% Local Variables:
%%% mode: LaTeX
%%% TeX-master: "../bkhandbuch"
%%% End:
