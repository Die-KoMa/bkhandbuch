\section{Bewerbungsunterlagen}\thispagestyle{fancy}
Nachdem die Stelle ausgeschrieben worden ist kommen ca. zwei Monate später die Bewerbungsunterlagen im entsprechenden Sekretariat an. Oft erstellt eine Schreibkraft direkt aus diesen Daten Übersichtstabellen, um Bewerber leichter vergleichen zu können. Dieses ist jedoch nicht immer der Fall und auch die Vergleichsinformationen zur Lehre der einzelnen Dozenten schwanken stark. Wir empfehlen euch daher aus den folgenden Punkten die für euch wichtigsten Vergleichspunkte zu wählen und eine eigene Vergleichstabelle zur Lehre der einzelnen Bewerber zu erstellen.

Folgende Merkmale können hierzu sinnvoll verwendet werden:
\begin{itemize}
    \item Anzahl der bereits gehaltenen Lehrveranstaltungen? (je Grund- und Hauptstudium)
    \item Anzahl der bisher betreuten Abschlussarbeiten? (je Bachelor und Master)
    \item Größe der bisher größten Veranstaltung?
    \item Hat der Bewerber Erfahrung mit Lehramtsvorlesungen?
    \item Hat der Bewerber Erfahrung im Service-Bereich?
    \item Welche Vorlesungen hat der Bewerber bereits angeboten?
    \item Wie viele (fachspezifische) Seminare hat der Bewerber schon angeboten?
    \item Hat der Bewerber schon einen ganzen Vorlesungszyklus gelesen?
    \item Wenn es an seiner Heim-Hochschule Evaluationen gibt: Wie hat der Bewerber abgeschnitten? Hat er Preise erhalten?
    \item Steht in der Bewerbung überhaupt etwas zu seiner Lehre?
\end{itemize}

Diese Vergleichstabelle sollte zur zweiten Sitzung der Kommission fertig sein. Ebenso ist eine Gruppierung der Bewerber anhand dieser Liste in Gruppe A, B und C (von gut über mittel nach schlecht) sinnvoll. Zwar wird in der zweiten Sitzung vor allem die Qualität und Quantität der bisherigen Publikationen der Bewerber bewertet, jedoch sollten Zweifel an der Lehrbefähigung einzelner Kandidaten auch schon an dieser Stelle deutlich angebracht werden.

Üblicherweise gibt es am Ende der Sitzung immer die Frage, ob man einige Kandidaten noch aus dem Grund einladen sollte, um etwa die Gleichstellungsbeauftragte oder andere Menschen "`zu beruhigen"'. Um diesen Diskussionen mit sachlichen Argumenten entgegentreten zu können, solltest du dich im Vorfeld bereits genauer mit den hierfür in Frage kommenden Bewerbern beschäftigen. Denn auch hier sollte die Qualität der Bewerber an erster Stelle stehen und schlechte bis mittelmäßige Bewerber sollten nicht in die zweite Runde gelangen.

Nach dieser Sitzung wird es dann (meistens) mindestens vier Wochen Pause geben, bis Termine mit allen Bewerbern vereinbart wurden, an denen sie sich persönlich bei euch vorstellen. Diese Zeit sollte mit dem Einholen von Informationen sinnvoll gefüllt werden.

%%% Local Variables:
%%% mode: LaTeX
%%% TeX-master: "../bkhandbuch"
%%% End:
