\section{Bewerbungsunterlagen}\thispagestyle{fancy}
Nachdem die Stelle ausgeschrieben worden ist, kommen ca. zwei Monate später die Bewerbungsunterlagen im entsprechenden Sekretariat an. Oft erstellt eine Schreibkraft direkt aus diesen Daten Übersichtstabellen, um Bewerber\gstar{innen} leichter vergleichen zu können. Dies ist jedoch nicht immer der Fall und auch die Vergleichsinformationen zur Lehre der einzelnen Dozenten\gstar{innen} schwanken stark. Wir empfehlen euch daher, aus den folgenden Punkten die für euch wichtigsten Vergleichspunkte zu wählen und eine eigene Vergleichstabelle zur Lehre der einzelnen Bewerber\gstar{innen} zu erstellen.

Folgende Merkmale können hierzu sinnvoll verwendet werden:
\begin{itemize}
    \item Bisherige Lehrerfahrung
    \begin{itemize}
        \item Anzahl der bereits gehaltenen Lehrveranstaltungen? (je Grund- und Hauptstudium)
        \item Größe, der bisher größten Veranstaltung?
        \item Welche Vorlesungen hat die Bewerber\gstar{in} bereits angeboten?
        \item Wie viele (fachspezifische) Seminare hat die Bewerber\gstar{in} schon angeboten?
        \item Hat die Bewerber\gstar{in} schon einen ganzen Vorlesungszyklus gelesen?
        \item Hat die Bewerber\gstar{in} Erfahrung mit Lehramtsvorlesungen?
        \item Hat die Bewerber\gstar{in} Erfahrung im Service-Bereich (Lehrexporte für andere Fächer)?    
    \end{itemize}
    \item Anzahl der bisher betreuten Abschlussarbeiten? (je Bachelor und Master)
    \item Haben Abschlussarbeiten bereits zu Publikationen geführt?
    \item Hat die Bewerber\gstar{in} Lehrevaluationen angehängt? Sind diese aktuell und was steht darin? 
    \item Hat die Bewerber\gstar{in} Lehrpreise erhalten oder Drittmittel für Lehre eingeworben?
    \item Beinhaltet die Bewerbung einen Abschnitt zur Lehre\footnote{Ein solches "Teaching Statement" kann man auch explizit bei der Bewerbung fordern.}? Ist dieser überzeugend?
\end{itemize}

Diese Vergleichstabelle sollte zur zweiten Sitzung der Kommission fertig sein. Ebenso ist eine Gruppierung der Bewerber\gstar{innen} anhand dieser Liste in Gruppen A, B und C (von gut über mittel nach schlecht) sinnvoll und erfolgt oft in der Kommission.
Deine Aufgabe ist es nun, die bestmöglichen Kandidat\gstar{innen} zu finden, wobei dein Hauptaugenmerk auf der Lehre liegen sollte. 
In der zweiten Sitzung wird zwar vor allem die Qualität und Quantität der bisherigen Publikationen der Bewerber\gstar{innen} bewertet, allerdings muss auch die Lehrbefähigung gegeben sein. Zweifel daran solltest du schon an dieser Stelle deutlich anbringen.
Ein Abschnitt zur Lehre kann zunehmend von den Bewerber\gstar{innen} erwartet werden und sollte ebenfalls in die Bewertung einfließen.

Aus Gründen der Gleichstellung wird manchmal eingebracht, dass weitere Personen eingeladen werden sollten\footnote{Sollten etwaige Quoten mit der bisherigen Bewerbendenlage nicht erfüllbar sein, so deutet das auf einen Mangel der bisherigen Bewerbendensuche hin und eine Neueröffnung des Verfahrens könnte notwendig sein. Teilweise können auch gezielt geeignete Personen eingeladen werden, sich zu bewerben.}.
Ihr solltet euch dabei dafür einsetzten, dass nur Personen eingeladen werden, die auch tatsächlich eine Chance haben, d.h. für die Stelle geeignet sind.
Insgesamt muss die Qualität der Bewerbungen an erster Stelle stehen und schlechte bis mittelmäßige Bewerber\gstar{innen} sollten nicht in die zweite Runde gelangen.

Nach dieser Sitzung wird es dann (meistens) mindestens vier Wochen Pause geben, bis Termine mit allen Bewerber\gstar{innen} vereinbart wurden, an denen sie sich persönlich bei euch vorstellen. Diese Zeit sollte mit dem Einholen von Informationen sinnvoll gefüllt werden.

%%% Local Variables:
%%% mode: LaTeX
%%% TeX-master: "../bkhandbuch"
%%% End:
