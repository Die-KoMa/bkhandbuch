\section{Informationen einholen}
Es ist üblich, dass die studentischen Mitglieder der Berufungskommission Informationen über die Lehrqualität der Bewerber an den jeweiligen Heimuniversitäten einholen.
Das sollte vor dem Bewerbungsgespräch geschehen, damit du auf Basis dieser Informationen Fragen stellen kannst.

Je nach Ergiebigkeit der einzelnen Quellen bietet es sich an mehrere der folgenden Möglichkeiten zu kombinieren:
\begin{description}
      \item [Bewerbungsunterlagen]
            Vor den Gesprächen solltest du die Bewerbungsunterlagen der Bewerber nocheinmal gründlich lesen.
      \item [Uniwebsiten]
            Viele Lehrende stellen ihre Lehrmaterialien im Internet zur Verfügung. Diese können eine gute Grundlage für die Beurteilung der Lehrqualität sein. Hierzu zählen Vorlesungsskripte, Übungsblätter, Klausuren und andere Materialien.
      \item [Youtube]
            Einige Lehrende stellen auch Vorlesungsaufzeichnungen und Videos bereit. Diese sind eine gute Möglichkeit, um einen Eindruck von der Lehrweise des Bewerbers zu bekommen.
      \item [Suchmaschinen]
            Dieses kann der erste Anlaufpunkt sein, um zunächst einmal ein grobes Bild über den Bewerber zu erhalten.
      \item [Forschungsportale]
            Profile auf Portalen wie Google Scholar können dir helfen, die wissenschaftliche Qualifikation des Bewerbers zu beurteilen, dabei wird besonders der $h$-Index oft herangezogen.
      \item [Webseite des Bewerbers]
            Sofern die Bewerbungsunterlagen nicht besonders ergiebig sind lassen sich hier meist die bisher gehaltenen Veranstaltungen finden. Auch findet sich oft ein Überblick über bisherige akademische Leistungen.
      \item [Fachschaft kontaktieren]
            Hier ist die E-Mail das Medium der Wahl. Vorteil bei solchen Nachfragen ist, dass man gleich Informationen über das Engagement in der universitären Selbstverwaltung erhalten kann. Eine Muster-E-Mail für eine solche Anfrage findest du im Abschnitt~\ref{sec:anfrage}.
      \item [Studentische Gutachten]
            Bei ausländischen Bewerbern kann man darum bitten, studentische Gutachten über die Lehrqualität mit einzureichen. Im angelsächsischen Raum ist dieses Vorgehen üblich. Jedoch ist hier zu beachten, dass der Bewerber meist selber den Studierenden aussucht, der das Gutachten schreibt.
\end{description}

Für alle auftretenden rechtlichen Fragen verweisen wir auf Abschnitt~\ref{sec:rechtliches}.

Im deutschsprachigen Raum findet man an fast jeder Hochschule eine Fachschaft, Studiengangsausschuss oder Fachschaftsverein. 
Am einfachsten sind diese meist per E-Mail zu erreichen. 
Sei dabei umsichtig, denn Bewerbungsverfahren sind in der Regel vertraulich.
Nicht alle Fachschaften sind bereit, Informationen zu geben.
Sollte deine Fachschaft in so einer Sache angesprochen werden, müsst ihr diese nicht beantworten, solltet aber unter keinen Umständen falsche Informationen geben.
Eine Auskunft von der Fachschaft kann dir Informationen zum Umgang mit Studierenden und über die Lehrqualität des Bewerbers geben, die in den Bewerbungsunterlagen nicht enthalten sind.
In der Regel darfst du solche Informationen im Berufungsverfahren selbst nicht verwenden, sie können aber helfen die Bewerbenden besser zu verstehen.

Daneben existieren aber noch einige weitere Möglichkeiten Informationen über die Bewerber zu erhalten, die nicht direkt in den Bewerbungsunterlagen stehen.
Beispielsweise kann man auch die Bewerbenden direkt fragen, allerdings sollte ein solches Vorgehen mindestens mit der Kommissionsleitung abgesprochen sein.


%%% Local Variables:
%%% mode: LaTeX
%%% TeX-master: "../bkhandbuch"
%%% End:
