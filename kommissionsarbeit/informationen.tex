\section{Informationen einholen}
Es ist üblich, dass die studentischen Mitglieder der Berufungskommission Informationen über die Lehrqualität der Bewerber an den jeweiligen Heimuniversitäten einholen. -- Meist wird das sogar von den Professoren erwartet. -- Dieses sollte auch ruhig vor dem Bewerbungsgespräch geschehen, damit du entsprechend dieser Informationen Fragen stellen kannst.

Im deutschsprachigen Raum findet man an fast jeder Hochschule eine Fachschaft, Studiengangsausschuss oder Fachschaftsverein. Dieses sollte deine erste Anlaufstelle sein. Am einfachsten sind diese per E-Mail zu erreichen.  Wenn ihr per E-Mail keine Antwort erhaltet solltet ihr aber spätestens in der Woche vor den Vorträgen versuchen, telefonisch Informationen zu bekommen.

Daneben existieren aber noch einige weitere Möglichkeiten Informationen über die Bewerber zu erhalten, die nicht direkt in den Bewerbungsunterlagen stehen.

\subsection{Wege um Informationen zu bekommen}
Je nach Ergiebigkeit der einzelnen Quellen bietet es sich an mehrere der folgenden Möglichkeiten zu kombinieren:
\begin{description}
    \item [Google]
          Dieses kann der erste Anlaufpunkt sein, um zunächst einmal ein grobes Bild über den Bewerber zu erhalten.
    \item [Webseite des Bewerbers]
          Sofern die Bewerbungsunterlagen nicht besonders ergiebig sind lassen sich hier meist die bisher gehaltenen Veranstaltungen finden. Ebenso kann man z.\,B. nach Skripten, Übungsblättern oder Klausuren des Bewerbers suchen.
    \item [\url{www.ratemyprofessors.com}]
          Gerade bei ausländischen Bewerbern ist es es extrem schwierig an objektive Lehrgutachten zu gelangen. Zwar werden meist Ergebnisse aus Lehrevaluationen mit zu den Bewerbungsunterlagen gelegt, jedoch zeigen diese naturgemäß nur Positives. Ein Einstieg für die Recherche zu Bewerbern aus dem nicht deutschsprachigen Raum kann diese Webseite sein. Wir betonen aber, dass die Webseite keinesfalls repräsentativ ist und höchstens für den Einstieg Sinn macht.
    \item [\url{www.facebook.com}]
          Wenn ihr direkt Kontakt zu Studierenden aufnehmen möchtet, welche gerade im Moment eine Vorlesung bei einem Bewerber belegen, so eignet sich Facebook oder ähnliches exzellent, um Kontakte zu solchen Teilnehmern zu erhalten. Bei diesen Anfragen solltet ihr aber bedenken, dass den Studierenden die grundsätzlichen Probleme beim Einholen solcher Informationen in der Regel nicht bekannt sind.
    \item [Fachschaft kontaktieren]
          Hier sind E-Mail oder das Telefon das Medium der Wahl. Vorteil bei solchen Nachfragen ist, dass man gleich Informationen über das Engagement in der universitären Selbstverwaltung erhalten kann. Eine Muster-E-Mail für eine solche Anfrage findest du im Abschnitt~\ref{sec:anfrage}.
    \item [Besuchen]
          Falls ein Bewerber momentan an eurer Nachbarhochschule unterrichtet, so kann man auch einfach vorbeifahren und seine Lehre aus erster Hand erfahren. Dieses sollte möglichst nicht angekündigt werden.
    \item [Studentische Gutachten]
          Bei ausländischen Bewerbern kann man darum bitten, studentische Gutachten über die Lehrqualität mit einzureichen. Im angelsächsischen Raum ist dieses Vorgehen üblich. Jedoch ist hier zu beachten, dass der Bewerber meist selber den Studierenden aussucht, der das Gutachten schreibt.
\end{description}

Für alle auftretenden rechtlichen Fragen verweisen wir auf Abschnitt~\ref{sec:rechtliches}.

%%% Local Variables:
%%% mode: LaTeX
%%% TeX-master: "../bkhandbuch"
%%% End:
