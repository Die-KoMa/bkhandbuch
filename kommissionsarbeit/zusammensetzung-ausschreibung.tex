\section[Zusammensetzung und Ausschreibung]{Zusammensetzung der Kommission und Ausschreibung der Stelle}
\sectionmark{Zusammensetzung und Ausschreibung}
Die Ausschreibung und Beantragung bzw. Einrichtung einer neuen Stelle ist üblicherweise ein langwieriger Prozess. Wir gehen hier davon aus, dass bereits eine entsprechende Instanz deiner Hochschule die Ausschreibung einer Professur angeordnet bzw. genehmigt hat (z.\,B. Dekanat, Präsidium, Rektorat).

Für die Besetzung dieser Stelle werden an öffentlichen Hochschulen Kommissionen eingesetzt, welche jeweils eine Liste mit Vorschlägen und eine entsprechende Reihung erarbeitet. In Deutschland -- je nach Bundesland und geltendem Hochschulgesetz kann die Zusammensetzung der Kommission und Gestaltung der Arbeit innerhalb der Kommission recht unterschiedlich ausfallen -- befindet sich stets mindestens ein studentischer Vertreter in solch einer Kommission\footnote{Wir gehen davon aus, dass sich an dieser Tatsache auch in Zukunft nichts ändern wird, verweisen aber auf das Publikationsdatum dieses Handbuches.}. In diesem Abschnitt soll nun der Prozess der Kommissionszusammensetzung, aber auch der Auswahl eines geeigneten studentischen Mitgliedes genauer beleuchtet werden.

\subsection{Mitglieder der Kommission}
Die Zusammensetzung der Kommission wird üblicherweise im Fakultätsrat oder Fachbereichsrat vorgenommen. Dort nominieren die jeweiligen Statusgruppen ihre Vertreter -- also jeweils Professoren, Mitarbeiter und Studierende -- und wählen diese nach Statusgruppen getrennt. Die Anzahl der Vertreter der jeweiligen Statusgruppen regelt dabei eure \emph{Berufungsordnung}.

Neben den Vertretern der einzelnen Statusgruppen kann eure Berufungsordnung noch weitere Mitglieder vorschreiben. Eventuell wünscht sich aber auch der Fakultätsrat oder Fachbereichrats weitere externe oder beratende Mitglieder, welche sich an der Kommission beteiligen sollen. Solche Mitglieder können etwa folgende sein (die Bezeichnungen variieren von Hochschule zu Hochschule):

\begin{description}
    \item [Beratendes Mitglied]
          Ein beratendes Mitglied kann eine Person aus der eigenen Hochschule, also z.\,B. ein weiterer Professor oder aber auch ein weiterer Student sein.

    \item [Fakultätsübergreifende Mitglieder]
          Diese Mitglieder aus einer anderen Fakultät --- gele\-gent\-lich auch "`Wachhunde genannt"' --- sollen im Auftrag der Hochschulleitung dafür sorgen, dass das Berufungsverfahren ordnungskonform und zügig durchgeführt wird. Diese sind jeder nicht an jeder Hochschule zu finden.

    \item [Auswärtiger Professor]
          Dieses bezeichnet einen Professor einer anderen Hochschule. Gerade bei kleinen Fachbereichen wird gerne auf diese Möglichkeit zurückgegriffen, um eine hinreichende Fachkompetenz in der Kommission zu haben. Auswärtige Professoren haben dabei in der Regel Stimmrecht.

    \item [Auswärtiger Experte]
          Experten sind meist Personen außerhalb der Hochschullandschaft, welche mit oder ohne Stimmrecht in die Kommission aufgenommen werden. Beispielsweise bei Stiftungsprofessuren wird gerne ein Vertreter der entsprechenden Stiftung mit in die Kommission aufgenommen.

    \item [Frauenbeauftragte]
          Die Frauenbeauftragte (oder Gleichstellungsbeauftragte, je nach Hochschule) soll die Rechte der Frauen vertreten. Sie hat in der Regel kein Stimmrecht, darf aber ein Votum zur Frage der Berücksichtigung von
          Frauen abgeben. In vielen Bundesländern darf Sie auf Einladung von Frauen bestehen.

    \item [Schwerbehindertenbeauftragte]
          Der Schwerbehindertenbeauftragte soll die Rechte schwerbehinderter Bewerber vertreten. Er nimmt in der
          Regel nur dann an der Berufungskommission teil, wenn und solange schwerbehinderte Bewerber im Verfahren
          sind. Der Schwerbehindertenbeauftragte hat kein Stimmrecht.

\end{description}

Ob die o.\,g. Mitglieder Stimmrecht haben oder nicht, hängt von der jeweiligen Berufungsordnung ab. Bei der Zusammensetzung der Kommission gilt jedoch immer: die Hälfte der Mitglieder müssen Professoren sein. Daher kann sich durch das Hinzuziehen von weiteren Mitgliedern ggf. die Stimmengewichtung der professuralen Mitglieder innerhalb der Kommission ändern. Hier müssen wir aber auf deine jeweilige Berufungsordnung verweisen.

\subsection{Das studentische Mitglied}
Es ist nicht immer leicht sich für ein studentisches Mitglied für die Kommission zu entscheiden bzw. es überhaupt zu finden. Folgendes sollte bei der Auswahl eines studentischen Mitgliedes aber immer beachtet werden:
\begin{itemize}
    \item Eine Berufungskommission benötigt punktuell sehr viel Zeit. Es ist insbesondere nur schwer möglich, neben dem Studium sämtliche Kommissionssitzungen, Bewerbungsvorträge und Bewerbungsgespräche zu besuchen. Daher solltest du schauen, ob die Möglichkeit eines Stellvertreters in deiner Berufungsordnung gegeben ist. Falls dieses nicht möglich ist, so kann man oft ein weiteres beratendes studentisches Mitglied in die Kommission wählen lassen.

    \item Es ist ggf. möglich Vergünstigungen und Erleichterungen für die Arbeit in einer Berufungskommission zu erhalten. So kann man mit den Dozenten absprechen, dass man aufgrund der hohen Arbeitsbelastung einen Übungszettel nicht bearbeiten oder später abgeben darf. Insbesondere in der Woche, in der die Vorträge und Interviews stattfinden, kommt man kaum zum Studieren. Wenn man Kurse bei Professoren besucht, die selber in der Berufungskommission sitzen, hat man gute Chancen etwas Arbeitserleichterung zu bekommen. Sollte es in einigen deiner Kurse Anwesenheitspflicht geben, kannst du für Sitzungen der Berufungskommission davon befreit werden.

    \item Ihr solltet euch immer \emph{selbst} -- ohne Beeinflussung von außen -- überlegen, wen ihr in diese Kommission schicken möchtet. Diese Person sollte aber das Grundstudium, bzw. die erste Hälfte des Bachelors abgeschlossen haben. Dieses wird teilweise in den Berufungsordnungen auch gefordert. In jedem Falle ist dieses aber sinnvoll, da so sicher gestellt wird, dass bereits eine genügende Fachkompetenz und Erfahrung beim studentischen Mitglied vorhanden ist.

    \item Ein wichtiger Punkt ist die Frauenquote, welche stets eine Kardinalsfrage bei der Zusammenstellung der Kommission ist. Meist haben die Professoren oder Mitarbeiter nur einen sehr geringen Frauenanteil. Somit können Studierende allein über das Einsetzen von weiblichen Kommissionsmitgliedern ggf. ein besonderes Gewicht in dieser Kommission einnehmen.
\end{itemize}

\subsection{Ausschreibung der Stelle}
Die Stellenausschreibung wird üblicherweise in der ersten Kommissionssitzung gemeinsam mit einem \emph{Kriterienkatalog} verabschiedet. Bei beidem solltet ihr darauf achten, dass die pädagogische Eignung explizit genannt wird. Ebenso sollte der Lehrbedarf im Service-Bereich\footnote{Service-Bereich sind in diesem Sinne Veranstaltungen für andere Studiengänge, wie z.B. Mathe für Wirtschaftswissenschaftler.} auch als Aufgabe in der Ausschreibung stehen.

Neben der pädagogischen Eignung für die abzudeckenen Veranstaltungen solltet ihr zudem darauf achten, dass die Stelle breit genug ausgeschrieben wird. Ein ab und an auftretendes Problem ist es nämlich, dass teilweise schon sehr genaue Vorstellungen existieren, wer die Stelle annehmen könnte. Wenn nun auch die Ausschreibung explizit auf diese Person zugeschnitten ist verhindert das die Möglichkeit einer Auswahl.

\subsubsection*{Kriterienkatalog}\label{kriterienkatalog}
Der Kriterienkatalog ist die wesentliche Entscheidungsgrundlage für die Auswahl und Reihung der Listenkandidaten. Die Festlegung der Kriterien muss in der Regel in der ersten Kommissionssitzung -- insbesondere vor Beschluss der Ausschreibung -- erfolgen und dient zur Bewertung der Bewerber. Die darin enthaltenen Kriterien sind die alleinigen Bewertungskriterien, welche die Kommission im Weiteren anlegen darf. Kriterienkataloge gibt es sowohl mit als auch ohne Gewichtung und Reihung der einzelnen Punkte.

\begin{center}
    \fbox{
        \begin{minipage}{0.8\textwidth}
            \begin{itemize}
                \item Wissenschaftliche Qualifikation
                \item Passendes Arbeitsgebiet gemäß Ausschreibung
                \item {\bf Lehr- und Vortragserfahrung sowie pädagogische Eignung}
                \item Fachübergreifende Bezüge zu anderen an der Hochschule vertretenden Arbeitsgebieten (Potential zur Bildung von Kooperationen)
                \item Erfahrung in der Einwerbung von Drittmitteln
                \item Internationale Erfahrung
            \end{itemize}
        \end{minipage}
    }
    \begin{center}
        Ein Beispiel für einen Kriterienkatalog.
    \end{center}
\end{center}

%%% Local Variables:
%%% mode: LaTeX
%%% TeX-master: "../bkhandbuch"
%%% End:
