\section{Entscheidung vs.\ Gutachten}
Nach der Vorstellung der Bewerber entscheidet die Kommission, welche Bewerber\gstar{innen} weiter berücksichtigt werden sollen. Über diese werden bei verschiedenen namhaften Persönlichkeiten aus dem jeweiligen Bereich Gutachten über die Qualität der Bewerber\gstar{innen} eingeholt. Das Einholen dieser Gutachten kann bis zu zwei Monate dauern.

Aufgabe der Kommission ist es, auf Grundlage dieser Gutachten eine Berufungsliste zu erstellen. Üblicherweise hat eine Berufungsliste drei Plätze, sehr selten können auch Plätze doppelt besetzt werden. Diese Liste wandert dann durch die weiteren Gremien (z.B.\ den Fakultätsrat), wo sie jeweils bestätigt oder an die Kommission zurück verwiesen wird. Kommt die Liste schließlich bei der Präsident\gstar{in} oder Rektor\gstar{in} an, so verhandelt diese zunächst mit der erstplatzierten Bewerber\gstar{in}. Kommt es zu einer Einigung, so wird der Ruf erteilt, ansonsten wird mit der Zweitplatzierten verhandelt usw.

Im Idealfall fällt die Entscheidung sehr leicht, weil alle Gutachter\gstar{innen} dieselbe Reihung vorschlagen. Oft widersprechen sich aber manche der Gutachten (weil etwa eine Gutachter\gstar{in} den Ausschreibungstext falsch verstanden hat), oder es ist klar aus dem Gutachten herauszulesen, dass eine Gutachter\gstar{in} mit einer Kandidat\gstar{in} persönlich gut befreundet ist. Alles dies muss in der Kommission abgewogen werden. Oft nimmt eine Person aus der zentralen Universitätsverwaltung an den Sitzungen mit Teil, um sicherzustellen, dass die rechtlichen Rahmenbedingungen eingehalten werden, ansonsten liegt dies in der Verantwortung der Professor\gstar{innen}. Deine Aufgabe ist es, anhand der Gutachten und deiner (mit Notizen belegten) Eindrücke aus den Probevorträgen und den Gesprächen mitzudiskutieren, wen du aus studentischer Sicht als passende Besetzung für die ausgeschriebene Stelle siehst.

Deine Stärke liegt hierbei darin, die Lehrqualität zu beurteilen. Fachliche Eignung und Passung in das Institut kannst du als Student\gstar{in} meist kaum beurteilen -- überlasse die Bewertung dieser Kriterien daher ruhig den Professor\gstar{innen}.

Du solltest aber immer im Hinterkopf behalten, dass natürlich nur berufen werden kann, wer auch vernüftige Lehre macht. Dies sicherzustellen, ist deine Hauptaufgabe; ggf.\ musst du dafür sorgen, dass pädagogisch schwächere Kandidat\gstar{innen} auf die hinteren Listenplätze wandern oder sogar ganz von der Liste fallen. Im Zweifelsfall kannst du aber oft noch einen Sperrvermerk heraushandeln. Du solltest übrigens auch darauf vorbereitet sein, dass du vielleicht schon zu Beginn der Sitzung gefragt wirst: \enquote{Wie würden denn Sie aus studentischer Sicht die Bewerber\gstar{innen} gewichten?}


\subsection{Begutachtung der Gutachten}
Dieser Abschnitt soll dir ein paar Hilfestellungen geben, wie man pro oder contra bestimmter Bewerber argumentieren kann.

Auch wenn in machen Gutachten steht, dass die Lehrqualität des Bewerbers hervorragend ist, so sind diese Gutachten immer mit Vorsicht zu genießen. Oft haben die Gutachter die einzelnen Bewerber nur auf Tagungen vortragen gesehen und kennen daher nicht die Qualität ihrer Vorlesungen.

\paragraph{Über die Gutachten}
\begin{itemize}
    \item Es gibt \emph{Einzelgutachten}, welche nur einen Bewerber begutachten. Diese fallen üblicherweise sehr gut bis hervorragend aus. Einzig ein neutrales bis negatives Einzelgutachten kann dir zu Entscheidungsfindung dienen.
    \item \emph{Vergleichende Gutachten} sind Gutachten, die mehrere der Bewerber vergleichen. Hierbei gibt der Gutachter meist auch seine persönliche Reihung an. Dieses sind mit die wichtigsten Entscheidungshilfen.
    \item Wenn ein Bewerber \emph{nicht habilitiert} ist muss im Gutachten stehen, dass er zu eigenständiger Lehre befähigt ist bzw. alle Voraussetzungen für eine Professur erfüllt.
    \item Die didaktische Eignung eines Bewerbers kann nur in den seltensten Fällen mit einem Standardgutachten begründet werden.
    \item Von den Bewerbern selbst vorgeschlagene Gutachter sollte mit Vorsicht genossen werden. Dieses sind meist Kandidaten für das Schreiben von Einzelgutachten. Einzelne Berufungsordnungen schließen von den Bewerbern vorgeschlagene Gutachter grundsätzlich aus.
    \item Jeder Bewerber sollte von wenigstens zwei Gutachtern begutachtet werden, davon mindestens bei einem Gutachter vergleichend.
\end{itemize}

\paragraph{Argumentation gegen einen Bewerber}
\begin{itemize}
    \item Anhand des Kriterienkatalogs (hier muss pädagogische Eignung drin stehen, vgl. Seite~\pageref{kriterienkatalog}).
    \item Ggf. hat der Bewerber keine Habilitation.
    \item Notizen aus Vortrag und Gespräch.
    \item Bewerber hat nur geringe Lehrerfahrung.
    \item Je nachdem was im Gutachten steht: Gutachten kann die Lehrqualität nicht beurteilen bzw. Gutachter bescheinigt sehr gute Lehre.
    \item Schlechter Eindruck bei Studierenden (z.\,B. durch Umfrage beim Vortrag).
    \item Evaluationsergebnisse.
    \item Auf formale Fehler achten.
    \item Mit anderen Kommissionsmitgliedern und Sonderbeauftragen (Gleichstellungsbeauftragte etc.) zusammenschließen.
\end{itemize}

\paragraph{Argumentation für einen Bewerber}
\begin{itemize}
    \item Kriterienkatalog: wenn eine gute Lehrerfahrung vorliegt.
    \item Ggf. haben andere Bewerber keine Habilitation.
    \item Notizen aus Vortrag und Gespräch.
    \item Gute Lehrerfahrung, insbesondere auch mit großen Vorlesungen.
    \item Eindruck von Studierenden (z.\,B. durch Umfrage bei Vortrag).
    \item Evaluationsergebnisse.
    \item Gegen die anderen Bewerber diskutieren.
    \item Je nachdem was im Gutachten steht: Gutachten kann die Lehrqualität nicht beurteilen bzw. Gutachter bescheinigt sehr gute Lehre.
\end{itemize}

Jedoch ist es so, dass auch ein hervorragender Hochschullehrer nur dann auf eine Stelle gesetzt werden kann, wenn er in das jeweilige Forschungsprofil passt. Wenn du merkst, dass ein Bewerber nicht auf die Stellenausschreibung passt, dann ist es nicht sinnvoll weiter für ihn zu kämpfen.

Weiter ist noch zu sagen, dass die Pausen hervorragend für Mauscheln und Kontakte knüpfen geeignet sind. Dort kann man manchmal in einem netten Plausch schneller zum Ziel kommen als direkt während der Sitzung. Je nachdem wie gut du mit den Professoren klar kommst, kannst du diese auch in die ein oder andere Richtung lotsen. Insbesondere kannst du in festgefahrenen Situation versuchen einen Kompromiss vorzuschlagen.

\subsection{Die Abstimmung}
Der Ablauf der Abstimmung ist üblicherweise sehr genau durch die Berufungsordnung geregelt. Beispielsweise wird meist ein doppeltes Quorum (Mehrheit der Professoren und Mehrheit der Mitglieder) gefordert.

Fast immer ist die endgültige Abstimmung über die Bewerber geheim. Dazu solltest du dich auch auf jeden Fall mit deiner Berufungsordnung auseinander setzen. Gerade wenn du überlegt gegen den von der Kommission favorisierten Kandidaten zu stimmen, solltest du dir das genau überlegen. Es ist meistens auch sinnvoll, schon während der Beratung deutlich die Vorbehalte vorzubringen. Wie du letztendlich über die Bewerber abstimmst ist aber natürlich eine Gewissensangelegenheit. Ob du dagegen, dafür oder mit Enthaltung stimmst, solltest du dir situationsabhängig überlegen.

Die Professoren möchten üblicherweise ein einstimmiges Ergebnis, um den weiteren Weg des Berufungsverfahrens durch die einzelnen Instanzen einfacher zu gestalten. Daher hast du als Student ein gewisses "`Druckmittel"'. Vermutlich wirst du den ersten Listenplatz damit nicht ändern können, aber weiter hinten auf der Liste kannst du so evtl. ein pari passu oder einen Sperrvermerk erreichen.


\subsection{Das Sondervotum}
Wenn du gegen einen Kandidaten stimmst, er aber trotzdem auf die Liste kommt, so hast du üblicherweise das Recht ein Sondervotum abzugeben. Das bedeutet, du kündigst (je nach Berufungsordnung) unverzüglich nach der Abstimmung an, dass du ein Sondervotum einreichen möchtest und reichst dieses in der gegebenen Frist ein.

Der Vorteil eines Sondervotums ist der, dass dieses durch alle weiteren Gremien wandert und du so die Möglichkeit hast, deine Bedenken begründet weiter zu geben. Ebenso solltest du in einem solchen Fall auch direkt die studentischen Vertreter in den nächst höheren Gremien (Fachbereichs-/Fakultätsrat, Senat) informieren, damit diese mit einem gewissen Hintergrundwissen diskutieren können.

Jedoch solltest du dir vor einem Sondervotum stets darüber klar sein, dass dieses eine sehr deutliche bzw. extreme Meinungsäußerung ist und bei einzelnen Mitgliedern der Kommission für Unmut sorgen könnte.

%%% Local Variables:
%%% mode: LaTeX
%%% TeX-master: "../bkhandbuch"
%%% End:
