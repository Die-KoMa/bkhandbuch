\section{Bewerbungsgespräch}
Im Anschluss an die Vorträge finden üblicherweise die Bewerbungsgespräche zwischen den Bewerbern und der Kommission statt.
Im Regelfall sind diese Gespräche vertraulich und hinter verschlossener Tür.
Ihr seid dann zur Verschwiegenheit verpflichtet und dürft nur mit (studentischen) Vertretern in höherrangigen Gremien über die Inhalte dieser Gespräche sprechen, welche auch Einblick in die internen Unterlagen der Berufungskommission haben.\footnote{%
    An einzelnen Hochschulen sind diese Gespräche dahingegen in die Hälften Forschung und Persönliches geteilt.
    Hierbei ist dann der erste Teil hochschulöffentlich.}

Der Vorsitzende der Kommission leitet üblicherweise das Gespräch und wechselt dabei auch zwischen den Themenbereichen.
Dies sorgt dafür, dass der Bewerber nicht 90\% der Zeit über sein Lieblingsgebiet sprechen kann.
Irgendwann wird der Vorsitzende dann zur Lehre überführen, spätestens hier solltet ihre eure Fragen stellen.
Zur besseren Vergleichbarkeit solltet ihr darauf achten, jedem Bewerber die gleichen Fragen zu stellen.

Sollte die Zeit knapp werden und wurde immer noch nicht über die Lehre gesprochen, solltest du die Initiative ergreifen und versuchen in diese Richtung zu lenken.

Wir haben im Folgenden einige Fragen gesammelt, aus welchen du gerne auswählen darfst und von denen du dich bei deinen eigenen Fragen inspirieren lassen kannst.
Da die Zeit meist knapp bemessen ist, solltet ihr bevorzugt Fragen stellen, aus deren Antworten ihr euch die beste Einsicht über den Bewerber erwartet.

\subsection{Fragenkatalog: allgemeine Fragen}
\begin{itemize}
    \item Was bedeutet für Sie gute Lehre?
    \item Haben Sie sich unsere Bachelor- und Masterprüfungsordnung angeschaut?
          Kennen Sie sich damit aus?\footnote{%
              Da Professoren erwarten, dass sich Bewerber mit den Forschungsschwerpunkten des Institutes beschäftigt haben, können wir dieses auch erwarten.%
          }
    \item Wie stehen Sie zu Studiengebühren?
    \item Wie stellen Sie sich ihren Übungs-/Vorlesungsbetrieb vor?
          (Präsenzübungen, Vorrechnen, Hausaufgaben, …)
    \item Was sollte ein Bachelor können, wenn er fertig ist?
          Was sollte ein Lehramtsstudent können, wenn er fertig ist?
    \item Welche Erfahrungen in der akademischen Selbstverwaltung haben Sie?
    \item Wie stellen Sie sich die Zusammenarbeit mit der Fachschaft vor?
    \item Differenzieren Sie zwischen Lehramts- und Vollfachstudenten?
          Wie gehen Sie auf die besonderen Bedürfnisse der Lehramtsstudenten ein?
    \item Haben Sie schon an hochschuldidaktischen Fortbildungen teilgenommen?
    \item In einer Veranstaltung fallen 80\% der Studenten durch die Klausur.
          Wie gehen Sie damit um?
    \item Wie können Studenten Sie mit Fragen erreichen?
          Wie viele Tage pro Woche sind Sie an der Hochschule?
          Können Studenten auch außerhalb der Sprechzeiten zu Ihnen kommen?
    \item Wie unterscheidet sich bei Ihnen eine Vorlesung mit 12 Studierenden von einer mit 400 Studierenden?
          Wie gehen Sie mit dem höheren Lärmpegel um?
    \item Nennen Sie spontan je ein Thema für eine Bachelor-, Master- und Staatsexamensarbeit.
    \item Welche Vorlesungen wollen/werden Sie anbieten?
          Was ist ihr Kanon?
          Werden Sie eine Spezialisierungssequenz anbieten?
    \item Wie bereiten Sie ihre Vorlesungen vor?
          Welche Materialien (z.\,B. Skript) gibt es zur Vorlesung?
    \item Können Sie interessante Themen für Vorlesungen anbieten, die sich in erster Linie an Lehrämtler richten?\footnote{%
              Dieses sollten interessante und lebensnahe mathematische Themen sein, mit denen man auch mal ein paar Schüler beeindrucken kann;
              jeder Matheprof sollte so etwas in petto haben}
    \item Welche Kompetenzen haben Sie im Bereich der digitalen Lehre?
    \item Wie gehen Sie mit den Ergebnissen der Lehrevaluation um?
          Wie werten Sie diese aus, welche Konsequenzen ziehen Sie daraus?
          Was halten Sie von derartigen Befragungen?
\end{itemize}

\subsection{Fragenkatalog: bei Bedarf}
\begin{itemize}
    \item Wie viele/welche Vorlesungen haben Sie schon gehalten?
          (sofern nicht in Unterlagen angegeben)
    \item Gab es bei Ihnen eine Lehrevaluation und können Sie uns diese zuschicken?
          Wie haben Sie dort abgeschnitten und was halten Sie von solchen Befragungen? (wenn nicht bereits offiziell vorhanden)
    \item Auf spezielle Tätigkeiten eingehen, wenn diese z.\,B. auf der Webseite des Bewerbers gefunden werden konnten (z.\,B. Prüfungsausschuss).
\end{itemize}

%%% Local Variables:
%%% mode: LaTeX
%%% TeX-master: "../bkhandbuch"
%%% End:
