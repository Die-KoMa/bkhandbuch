\section{Die Vorträge an sich}
An jeder Hochschule gibt es Fachvorträge, in denen die Bewerber ihre aktuellen Forschungsgebiete präsentieren, ihre Erfolge loben und auch Ausblicke für ihre nächsten Ziele geben. Inhalt dieser Vorträge ist oft ein auch für Studenten nachvollziehbarer Teil. Dieser soll es ermöglichen, einen kleinen Einblick in die pädagogische Qualität des Bewerbers zu erhalten. Daneben gibt es an einigen Hochschulen zusätzlich noch Lehrproben, die also explizit die pädagogische Eignung der einzelnen Bewerber überprüfen. Leider ist dieses kein Standard und es tritt oft das Problem auf, dass die pädagogische Eignung allein durch den Fachvortrag überprüft werden muss. Wir gehen daher im Folgenden auf beide Fälle getrennt ein.

Für jeder Art dieser Vorträge ist aber zu sagen, dass es für euch immer hilfreich ist, wenn weitere Studierende während der Vorträge anwesend sind und dich bei der Bewertung dieser Vorträge unterstützen\footnote{Beispielsweise kann man Fragebögen (siehe~\ref{sec:fragebogen}) austeilen, um Notizen bitten oder einfach nach dem Vortrag in die Runde fragen.}. Daher solltest du dich auch bemühen, die Vorträge möglichst gut bei den Studierenden in deinem Fachbereich publik zu machen. Wunderbar eignen sich dafür etwa Mailverteiler, Aushänge oder Ankündigungen in Vorlesungen.

\subsection{Lehrprobe}
Eine Lehrprobe ist das Halten einer Vorlesung vor einer genügend großen Anzahl an Studierenden im Rahmen eines Berufungsverfahrens. Insbesondere sind dieses Veranstaltungen aus dem Grundstudium, welche es auch dem studentischen Kommissionsmitglied ermöglichen, sich voll und ganz auf die Darstellung zu konzentrieren und weniger auf den mathematischen Gehalt. Eine Lehrprobe kann etwa als Vertretung einer Analysis oder Lineare Algebra Vorlesung gehalten werden aber auch als eigenständiger Lehrvortrag. Wir haben hier zwei Beispiele:

\subsubsection{Uni Flensburg}
\begin{itemize}
    \item Lehrprobe wird bei allen Bewerbern gefordert
    \item findet nach dem Fachvortrag statt
    \item Bewerber dürfen sich selber das Thema aussuchen, es muss aber mit der Ausrichtung der Stelle verbunden sein
    \item Zielgruppe sind Bachelor-Studenten im Hauptstudium
    \item Dauer: 45\,min Vortrag zzgl. 15\,min für Fragen
    \item im Durchschnitt sind 30 -- 40 Studierende anwesend
    \item Angekündigt wird von der Fachschaft in einzelnen Veranstaltungen und durch Aushang
\end{itemize}

\subsubsection{Uni Bremen}
\begin{itemize}
    \item pro Bewerber wird sich ein halber Tag Zeit genommen
    \item Dauer: 45\,min Probevorlesung
    \item Inhalt des Vortrags
          \begin{itemize}
              \item Früher: entsprechend der ausgeschriebenen Professur
              \item Derzeit: es werden zwei Themen zur Auswahl angeboten: "`Satz über impl. Funktion"' oder "`Satz von Picard-Lindelöff"'
              \item Zukünftig: Bewerber halten Ersatzvorlesung für Grundstudiumsvorlesung (Analysis 1/2 oder Lineare Algebra 1/2)
          \end{itemize}
    \item am Ende werden zur Bewertung Fragebögen an zuhörende Studenten verteilt
\end{itemize}


\subsection{Fachvortrag}
Sofern keine Lehrprobe gehalten wird, solltest du darauf bestehen, dass sich ein signifikanter Teil des Fachvortrages an Studierende richten muss. Hierbei ist ein \emph{signifikanter Teil} eines 45 Minuten Vortrages mindestens 15 Minuten lang\footnote{Und mit mindestens meinen wir wirklich, dass man keinesfalls darunter gehen darf.}. Zielgruppe sollten auch in diesem Bereich Studierende mit abgeschlossenem Grundstudium sein.

Du solltest außerdem darauf achten, dass der Vorsitzende der Kommission daran denkt, den Vortragenden diese Zeiteinteilung mitzuteilen. Dieses wird ab und an schon mal vergessen.

\subsection{Bewertung der Vorträge}
Der Vortrag/die Probevorlesung sind üblicherweise hochschulöffentlich, jeder Student kann sich
diese also anhören. Um bewerten zu können, wie gut der Vortrag ankommt, kann man die zuhörenden Studenten
im Anschluss befragen. Einen Musterfragebogen findest du in Anhang~\ref{sec:fragebogen}.

Grundsätzlich gilt, dass du dir immer Notizen machen solltest. Du wirst im Folgenden noch öfters darauf angewiesen sein. Diese Notizen sollen aber nicht den Vortrag wiedergeben, sondern deine Eindrücke, wie der Vortragende seine Arbeit macht. Da dieses etwas ungewohnt ist, haben wir ein paar Punkte zusammengestellt, auf die man dazu achten kann:
\begin{itemize}
    \item Welche Medien verwendet der Bewerber und setzt er sie sinnvoll ein?
    \item Spricht der Bewerber dem Publikum zugewandt?
    \item Übersichtlichkeit der Darstellung (z.\,B. Farben, Handschrift, Folien)
    \item Wie geht der Bewerber auf Fragen ein?
    \item Wird abgelesen oder frei gesprochen?
    \item Wie ist die Motivation des Bewerbers und kann er motivieren?
    \item Artikulationsvermögen (insb. bei nicht Deutsch-Muttersprachlern)
    \item Geschwindigkeit des Vortrages
    \item Ist ein roter Faden erkennbar?
    \item Ist der Vortrag gut strukturiert?
\end{itemize}
Erfahrungsgemäß übersieht man inhaltliche Fehler, wenn man auf alle genannten Punkte achtet. Deswegen ist es sinnvoll, sich die Arbeit aufzuteilen, wenn man mit mehreren Studenten in der Kommission sitzt. Ansonsten werden inhaltliche Fehler aber in der Regel von den Professoren festgestellt.

\subsection{Fragen zum Vortrag}
Bereits während und auch direkt im Anschluss an die Fachvorträge werden üblicherweise Fragen gestellt. Auch ihr solltet diese Gelegenheit nutzen. Nützliche Fragen sind etwa:
\begin{itemize}
    \item Können sie \dots\ noch einmal erklären. Vielleicht für einen Studierenden im 3. Semester?
    \item Was sollte ich von diesem Vortrag mitnehmen? In zwei Sätzen bitte.
    \item Wie ist die Relevanz des Vortrages in den entsprechenden Themenbereich einzuordnen?
    \item Ggf. könnt ihr eine Frage auf Englisch stellen um das Englisch des Bewerbers zu testen.
\end{itemize}

%%% Local Variables:
%%% mode: LaTeX
%%% TeX-master: "../bkhandbuch"
%%% End:
