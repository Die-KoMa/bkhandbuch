\section{Die Vorträge an sich}
An jeder Hochschule gibt es Fachvorträge, in denen die Bewerber\gstar{innen} ihre aktuellen Forschungsgebiete präsentieren, ihre Erfolge loben und auch Ausblicke für ihre nächsten Ziele geben. Neben diesem Fachvortrag müssen die Bewerber\gstar{innen} auch an fast allen Hochschulen eine Lehrprobe halten, welche dazu dient, ihre pädagogischen Fähigkeiten unter Beweis zu stellen. Leider existieren immer noch Hochschulen, an denen zu diesem Zweck nur ein Teil des Fachvortrages herhalten muss. Dieser Teil muss sich dann spezifisch an Studierende richten. In einem solchen Fall solltet ihr euch unbedingt für eine Lehrprobe einsetzten und darlegen, dass diese essenziell für eine informierte Bewertung der pädagogischen Eignung ist. Wir gehen im Folgenden auf beide Fälle getrennt ein.

Für jede Art dieser Vorträge ist zu sagen, dass es für euch immer hilfreich ist, wenn weitere Studierende während der Vorträge anwesend sind und euch bei der Bewertung dieser Vorträge unterstützen\footnote{Beispielsweise kann man Fragebögen (siehe~\ref{sec:fragebogen}) austeilen, um Notizen bitten oder einfach nach dem Vortrag in die Runde fragen.}. Daher solltest du dich auch bemühen, die Vorträge möglichst gut bei den Studierenden in deinem Fachbereich publik zu machen. Wunderbar eignen sich dafür etwa Mailverteiler, Aushänge oder Ankündigungen in Vorlesungen.

\subsection{Lehrprobe}
Eine Lehrprobe ist das Halten einer Vorlesung vor einer ausreichend großen Anzahl an Studierenden im Rahmen eines Berufungsverfahrens. Die Themen der Lehrprobe sollten aus dem Grundstudium kommen um es allen anwesenden Studierenden zu ermöglichen, sich auf die Qualität des Vortrags zu konzentrieren. Eine Lehrprobe kann etwa als Vertretung einer Analysis oder Lineare Algebra Vorlesung gehalten werden, aber auch als eigenständiger Lehrvortrag. Bei der Auswahl der Themen muss eine Abwägung getroffen werden. Spezifische Themen führen zu einer besseren Vergleichbarkeit. Auf der anderen Seite können die Bewerber\gstar{innen} durch weniger konkrete Anweisungen zeigen, dass sie in der Lage sind, eine sinnvolle Themenauswahl zu treffen. Sollte eine Vorlesung zu einem bestimmten Thema gefordert werden, müssen die Bewerber\gstar{innen} im Vorherein unbedingt durch den Vorsitzende\gstar{n} darüber informiert werden, welches Vorwissen vorausgesetzt werden kann, bzw. was für Themen vor der geforderten Vorlesung behandelt wurden. Dies ermöglicht es, herauszufinden, ob die Bewerber\gstar{innen} in der Lage sind, eine Vorlesungsreihe sinnvoll zu strukturieren.

Wir haben hier zwei Beispiele:

\subsubsection{Uni Flensburg}
\begin{itemize}
    \item Lehrprobe wird bei allen Bewerber\gstar{innen} gefordert
    \item findet nach dem Fachvortrag statt
    \item Bewerber\gstar{innen} dürfen sich selber das Thema aussuchen, es muss aber mit der Ausrichtung der Stelle verbunden sein
    \item Zielgruppe sind Bachelor-Student\gstar{innen} im Hauptstudium
    \item Dauer: 45\,min Vortrag zzgl. 15\,min für Fragen
    \item im Durchschnitt sind 30 - 40 Studierende anwesend
    \item Angekündigt wird von der Fachschaft in einzelnen Veranstaltungen und durch Aushang
\end{itemize}

\subsubsection{Uni Bremen}
\begin{itemize}
    \item pro Bewerber\gstar{in} wird sich ein halber Tag Zeit genommen
    \item Dauer: 45\,min Probevorlesung
    \item Inhalt des Vortrags (im laufe der Zeit)
          \begin{itemize}
              \item Es werden zwei Themen zur Auswahl angeboten: "`Satz über impl. Funktion"' oder "`Satz von Picard-Lindelöff"'
              \item Bewerber\gstar{innen} halten Ersatzvorlesung für Grundstudiumsvorlesung (Analysis 1/2 oder Lineare Algebra 1/2)
          \end{itemize}
    \item am Ende werden zur Bewertung Fragebögen an zuhörende Student\gstar{innen} verteilt
\end{itemize}


\subsection{Fachvortrag}
Sofern keine Lehrprobe gehalten wird, solltest du darauf bestehen, dass sich ein signifikanter Teil des Fachvortrages an Studierende richten muss. Hierbei ist ein \emph{signifikanter Teil} eines 45 Minuten Vortrages mindestens 15 Minuten lang\footnote{Und mit mindestens meinen wir wirklich, dass man keinesfalls darunter gehen darf.}. Zielgruppe sollten auch in diesem Bereich Studierende mit abgeschlossenem Grundstudium sein.

Du solltest außerdem darauf achten, dass der/die Vorsitzende der Kommission daran denkt, den Vortragenden diese Zeiteinteilung mitzuteilen. Dies wird ab und an schon mal vergessen.

\subsection{Bewertung der Vorträge}
Der Vortrag/die Lehrprobe sind üblicherweise hochschulöffentlich, jede Student\gstar{in} kann sich
diese also anhören. Um bewerten zu können, wie gut der Vortrag ankommt, kann man die zuhörenden Studierenden
im Anschluss befragen. Einen Musterfragebogen findest du in Anhang~\ref{sec:fragebogen}.

Grundsätzlich gilt, dass du dir immer Notizen machen solltest. Zwischen den Vorträgen und den späteren Treffen der Kommission, in denen ebenfalls argumentiert wird, können Monate vergehen.
Da es später notwendig ist, begründet über die Bewerber\gstar{innen} zu sprechen, sollten diese Notizen aber nicht den Inhalt wiedergeben, sondern deine Eindrücke über die Qualität des Vortrags. Dies ist etwas ungewohnt. Deshalb haben wir ein paar Punkte zusammengestellt, auf die geachtet werden kann:
\begin{itemize} % eventuell erweitern/überprüfen, dass es sinnvoll ist.
    \item Welche Medien verwendet die Bewerber\gstar{in} und setzt er/sie sie sinnvoll ein?
    \item Spricht die Bewerber\gstar{in} dem Publikum zugewandt?
    \item Übersichtlichkeit der Darstellung (z.\,B. Farben, Handschrift, Folien)
    \item Wie geht die Bewerber\gstar{in} auf Fragen ein?
    \item Wird abgelesen oder frei gesprochen?
    \item Wie ist die Motivation der Bewerber\gstar{in} und kann er/sie motivieren?
    \item Artikulationsvermögen (insb. bei nicht Deutsch-Muttersprachlern)
    \item Geschwindigkeit des Vortrages
    \item Ist ein roter Faden erkennbar?
    \item Ist der Vortrag gut strukturiert?
\end{itemize}
Erfahrungsgemäß übersieht man inhaltliche Fehler, wenn man auf alle genannten Punkte achtet. Deswegen ist es sinnvoll, sich die Arbeit aufzuteilen, wenn man mit mehreren Studierenden in der Kommission sitzt. Ansonsten werden inhaltliche Fehler aber in der Regel von den Professor\gstar{in} festgestellt.

\subsection{Fragen zum Vortrag}
Bereits während und auch direkt im Anschluss an die Fachvorträge werden üblicherweise Fragen gestellt. Auch ihr solltet diese Gelegenheit nutzen. Nützliche Fragen sind etwa:
\begin{itemize}
    \item Können Sie \dots\ noch einmal erklären. Vielleicht für einen Studierenden im 3. Semester?
    \item Was sollte ich von diesem Vortrag mitnehmen? In zwei Sätzen bitte.
    \item Wie ist die Relevanz des Vortrages in den entsprechenden Themenbereich einzuordnen?
    \item Ggf. könnt ihr eine Frage auf Englisch stellen um das Englisch der Bewerber\gstar{in} zu testen.
\end{itemize}

%%% Local Variables:
%%% mode: LaTeX
%%% TeX-master: "../bkhandbuch"
%%% End:
