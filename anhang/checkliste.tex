\chapter{Checkliste}\thispagestyle{fancy}\label{chp:checkliste}
Hier sind ein paar Punkte, die man bei einer Berufungskommission abhaken sollte:
\begin{itemize}
    \item [\Square] Unterlagen gesichtet
    \item [\Square] Anfragen an Fachschaften herausgeschickt
    \item [\Square] bei fehlenden Antworten nachgehakt
    \item [\Square] Fragen für das Bewerbungsgespräch vorbereitet
    \item [\Square] Bewerbungsvortrag bei Studierenden angekündigt
    \item [\Square] ggf. Fragebögen für den Bewerbungsvortrag vorbereitet
    \item [\Square] Gutachten gelesen
    \item [\Square] Deine Kandidatenreihung erstellt
    \item [\Square] Votum geschrieben
    \item [\Square] ggf. Studentisches Votum pünktlich eingereicht
\end{itemize}
Es gibt zwar noch viel mehr zu tun, aber dieses sind oft die zeitkritischen Punkte, die man leicht vergessen kann.

%%% Local Variables:
%%% mode: LaTeX
%%% TeX-master: "../bkhandbuch"
%%% End:
