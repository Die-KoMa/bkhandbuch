\chapter{Eine Beispielausschreibung}\thispagestyle{fancy}\label{chp:beispielausschreibung}
Dieses ist eine Ausschreibung für eine W2-Professur Zahlentheorie an einer NRW-Universität. Wenn du weitere Ausschreibungen sehen möchtest, dann schau doch einfach mal in \emph{Die Zeit}.  In dieser Wochenzeitung werden eigentlich alle in Deutschland ausgeschriebenen Professuren veröffentlicht.
\clearpage{}

\noindent\fbox{\begin{minipage}{.95\textwidth}\vspace{0.4cm}
        Im Institut für Mathematik der Fakultät \dots ist eine W2-Professur für Zahlentheorie zum \dots zu besetzen. Die zu berufende Persönlichkeit soll ein aktuelles Forschungsgebiet im Bereich Zahlentheorie vertreten. Erwartet wird aktive Kooperationsbereitschaft bei Aufbau und Fortführung von Forschungsschwerpunkten und Graduiertenkollegs des Instituts. Erwünscht ist außerdem eine Zusammenarbeit mit der Theoretischen Informatik.

        Der Stelleninhaber/ die Stelleninhaberin soll sich maßgeblich an den Lehraufgaben des Faches Mathematik beteiligen und soll insbesondere auch im Service (vor allem für Informatik, Ingenieur- und Naturwissenschaften) mitwirken.

        Einstellungsvoraussetzungen: \S~46 Abs.\,1 Ziff.~4a HG NW (Habilitation oder habilitations-adäquate Leistungen) und Ziff. 2 HG NW (pädagogische Eignung).

        Die Universität \dots strebt eine Erhöhung des Anteils der Frauen als Hochschullehrerinnen an und fordert daher qualifizierte Wissenschaftlerinnen nachdrücklich zur Bewerbung auf. Frauen werden nach \S~7 LGG bei gleicher Eignung, Befähigung und fachlicher Leistung bevorzugt berücksichtigt.
        Schwerbehinderte erhalten bei gleicher Eignung den Vorrang.

        Bewerbungen mit den üblichen unterlagen werden innerhalb von vier Wochen nach Veröffentlichung
        unter Angabe der Kennziffer \dots erbeten an den Leiter des Instituts für Mathematik, \dots
        \vspace{0.4cm}
    \end{minipage}
}

%%% Local Variables:
%%% mode: LaTeX
%%% TeX-master: "../bkhandbuch"
%%% End:
