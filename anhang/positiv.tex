\chapter{Eine positive Stellungnahme}\thispagestyle{fancy}
\begin{center}
    \fbox{\begin{minipage}{.95\textwidth}\vspace{0.4cm}
            \begin{center}{\large Studentisches Votum zur Besetzung der Juniorprofessur}\smallskip\\
                {\Large Programmieren für Anfänger (Kennz. 123)}
            \end{center}\bigskip

            Sehr geehrte Damen und Herren,\medskip\\

            Als studentisches Mitglied unterstütze ich die von der Berufungskommission aufgestellte
            Vorschlagsliste ausdrücklich. Dies begründe ich wie folgt:\smallskip

            Dr. Java hielt einen gut verständlichen und (soweit ich das beurteilen kann) auch tiefgehenden
            Vortrag. Im Gespräch zeigte er sich an der mit der Juniorprofessur verbundenen Lehre
            außerordentlich interessiert. Dieser Eindruck wird durch seine bisherige Erfahrung in der Lehre an
            der University of Somewhere unterstützt. Ich erwarte, dass seine Veranstaltungen eine Bereicherung
            des Lehrangebotes am Institut Mathematik der Universität ABC sein werden.\end{minipage}}

    \fbox{\begin{minipage}{.95\textwidth}\vspace{0.4cm}Dr. C überzeugte ebenfalls durch einen gut verständlichen, exzellent strukturierten Vortrag. Zwar ist
            seine Lehrerfahrung nicht mit der von Dr. Java zu vergleichen, er zeigte sich jedoch während seiner
            Promotionszeit überdurchschnittlich engagiert, was die Lehre betrifft, und konnte diesen Eindruck
            auch im Gespräch bestätigen. Zudem ist zu erwarten, dass er seine Erfahrungen in der Wirtschaft in
            positiver Weise in die Lehre einbringen kann.\smallskip

            Ich bin der Überzeugung, dass beide Kandidaten die mit der zu besetzenden Juniorprofessur
            einhergehenden Aufgaben in Lehre und Forschung in vorbildlicher Weise erfüllen werden.


            \vspace{0.5 cm}

            Mit freundlichen Gr"u"sen\par
            \hspace*{2cm}Klaus Mustermann\\
            \vspace{0.4cm}
        \end{minipage}
    }
\end{center}

%%% Local Variables:
%%% mode: LaTeX
%%% TeX-master: "../bkhandbuch"
%%% End:
