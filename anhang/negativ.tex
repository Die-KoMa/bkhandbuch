\chapter{Eine negative Stellungnahme}\thispagestyle{fancy}\label{chp:negativ}
\begin{center}
    \fbox{\begin{minipage}{.95\textwidth}\vspace{0.4cm}
            \begin{center}{\large  Studentisches Votum zur Besetzung der Professur}\smallskip\\
                {\Large W3 Nachfolge Prof. Guru}
            \end{center}\bigskip

            Sehr geehrte Damen und Herren,\medskip\\

            Als studentisches Mitglied der Berufungskommission zur
            Besetzung einer W3-Professur (Kennziffer xyz, Nachfolge Guru) im Institut
            für Mathematik, möchte ich die von der Kommission verabschiedete Liste nicht unterstützen.\smallskip\par

            Meine folgenden Einschätzungen zu den einzelnen Bewerbern beziehen sich
            ausschließlich auf deren didaktische und methodische Fähigkeiten.
            Der Vortrag von Herrn Fermat machte auf mich einen gut strukturierten und
            durchdachten Eindruck, allerdings hätte ich mir zusätzlich Beispiele und/oder
            Anwendungsbezüge von ihm gewünscht.
        \end{minipage}}

      \fbox{\begin{minipage}{.95\textwidth}\vspace{0.4cm}
          Ich habe Zweifel, dass er in Vorlesungen
            besser auf seine Zuhörer eingehen kann, insbesondere da er bislang noch keine
            Erfahrungen mit Anfängerveranstaltungen, sog. Serviceveranstaltungen oder mit speziellen
            Veranstaltungen für Lehramtsstudierende, welche einen großen Anteil der
            Lehrtätigkeit dieser Professur darstellen, gesammelt hat.
            Herr Cauchy bot meiner Meinung nach einen gut strukturierten Vortrag mit
            Einbindung von Beispielen. Auch der Vortrag von Herrn Weierstraß enthielt
            Beispiele und Erklärungen, was auf mich einen positiven Eindruck machte.

            Generell habe ich den Eindruck, dass im Entscheidungsprozess der Kommission die
            Lehrqualifikation der Bewerber im Vergleich zum Forschungsgebiet und den
            Kooperationsmöglichkeiten in der Mathematik an der Universität ABC zu wenig
            berücksichtigt wurde. In der konstituierenden Sitzung der Kommission wurde allerdings die Lehrqualifikation neben der Forschungsqualifikation als höchstes Bewertungskriterium beschlossen.\smallskip

            Aus diesem Grund möchte ich die von der Kommission verabschiedete Liste nicht unterstützen.
            \vspace{0.4 cm}
        \end{minipage}
    }
\end{center}

%%% Local Variables:
%%% mode: LaTeX
%%% TeX-master: "../bkhandbuch"
%%% End:
