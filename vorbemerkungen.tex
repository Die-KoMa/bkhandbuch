\chapter*{Vorbemerkungen}\thispagestyle{fancy}
\addcontentsline{toc}{chapter}{Vorbemerkungen}

Das vorliegende Handbuch zur studentischen Mitwirkung in Berufungskommissionen wurde vom Arbeitskreis "`Berufungskommissionen"' (AK BK) der Konferenz der deutschsprachigen Mathematikfachschaften (KoMa) erstellt. Die hierin enthaltenen Ratschläge und Erläuterungen wurden von Fachschaftsvertretern zahlreicher Universitäten und Fachhochschulen im gesamten deutschsprachigen Raum zusammengetragen.

Trotz zahlreicher regionaler Unterschiede bei der Durchführung von Berufungsverfahren versucht dieses Handbuch einen in sich geschlossenen Gesamtablauf darzustellen, dessen Struktur an jeder deutschsprachigen Hochschule in dieser oder vielleicht in einer leicht abgeänderten Form wiedergefunden werden kann. Gleiches gilt für unterschiedliche Fachbereiche in unserer Hochschullandschaft. Zwar wurde dieses Handbuch speziell von Mathematikern für Mathematiker geschrieben, dennoch trifft der wesentliche Teil dieses Handbuches auch auf andere Fachkulturen zu.

Neben einer einheitlichen und klaren Darstellung des Gesamtablaufs halten wir es aber auch für wichtig, die regionalen Spezifika bzw. mögliche Ausprägungen von Berufungsverfahren zu erläutern. Dieses soll zum einen zur Abgrenzung der verschiedenen Landesgesetzgebungen dienen, zum anderen aber auch den Blick auf Verfahren an anderen Hochschulen weiten.

Der Aufbau dieses Handbuches ist in drei Kapitel unterteilt. Der erste Teil liefert eine Einführung in den Bereich der Berufungskommissionen. Im zweiten Teil wird der Ablauf einer Berufungskommission ausführlich dargestellt; insbesondere werden hier die verschiedenen Phasen erläutert und durch Hinweise und Empfehlungen kommentiert. Der dritte Teil geht auf den Kontakt mit anderen Fachschaften im Bereich der Berufungskommissionen und auf rechtliche Fragestellungen ein.

Zur Schreibweise dieses Dokumentes ist zu sagen, dass wir uns aufgrund der besseren Lesbarkeit für das generische Maskulinum entschieden haben. Stets sind jedoch Personen beiden Geschlechtes gemeint.

Für Anregungen, Kommentare, Ergänzungen und weitere Hinweise ist der Arbeitskreis Berufungskommissionen dankbar. Für Kontakt zur KoMa allgemein oder direkt zum Arbeitskreis verweisen wir auf die Webseite der KoMa: \url{www.die-koma.org}.

\vspace{1cm}\noindent
Die Teilnehmer der WAchKoMa Bremen, 2009

%%% Local Variables:
%%% mode: LaTeX
%%% TeX-master: "bkhandbuch"
%%% End:
