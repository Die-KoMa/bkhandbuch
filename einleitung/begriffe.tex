\section{Begriffserklärungen}
Wie in allen Hochschulgremien gibt es auch in Berufungskommissionen eine eigene Sprache. Dieses sind Bezeichnungen für bestimmte Gehaltsstufen, Lehrstulausstattungen, aber auch Verfahrensanweisungen. Die Wichtigsten stellen wir hier kurz vor:
\begin{description}
    \item [Gehaltsstufen]%TODO nur in Deutschland?
          Mittlerweile werden alle neu zu berufenden Professoren nach der \emph{Bundesbesoldungsordnung W}
          bezahlt. Ältere Professoren, die schon lange an der Hochschule sind werden ggf. noch nach der alten
          Ordnung C bezahlt. In der W-Besoldung gibt es drei Gehaltsstufen, W1, W2 und W3. Dabei werden
          \emph{Juniorprofessoren} mit W1 besoldet, alle anderen Professoren nach W2 oder W3. Abgesehen vom Grundgehalt
          unterscheiden sich die Stellen in der Ausstattung mit Mitarbeitern und Sachmitteln. Dabei sind
          W3-Professuren in der Regel am Besten ausgestattet. In manchen Bundesländern gibt es keine
          W2-Professuren, hier werden die W3-Professuren zum Teil mit \emph{Leitungsfunktion} ausgeschrieben.
          Diese werden dann besser bezahlt und haben mehr Personal- und Sachmittel zur Verfügung.

    \item [Hausberufung]
          Von einer Hausberufung spricht man, wenn der zu Berufende schon am ausschreibenden Institut arbeitet,
          etwa als Postdoc, oder Juniorprofessor. Hausberufungen sind in Deutschland eher unüblich und werden
          oft kritisch gesehen. Die Berufungskommission muss in solch einem Fall besonders begründen, warum der Bewerber
          der Beste aller Kandidaten ist, da sonst der Eindruck entstehen kann, dass dort eigene Leute
          bevorteilt werden.

    \item [Honorarprofessur]
          Dozenten, die keine Hochschullehrer sind, weil sie etwa in der Wirtschaft oder einem Forschungsinstitut
          arbeiten, können zum Honorarprofessor berufen werden. Eine Honorarprofessor wird hierbei als Auszeichnung für
          verdiente Gastdozenten verstanden. Honorarprofessoren haben meist keine oder nur eine sehr geringe
          Lehrverpflichtung und arbeiten hauptamtlich noch in ihrem eigentlichen Beruf.

    \item [Juniorprofessoren]
          Die sogenannten Juniorprofessuren wurden 2002 eingeführt. Sie stellen eine Möglichkeit dar, sich
          ohne Habilitation für eine ordentliche Professur zu qualifizieren. Juniorprofessoren werden in der Regel mit W1
          besoldet und haben eine geringere Lehrverpflichtung als W2- oder W3-Professoren. Sie sind, anders als die "`großen"'
          Professuren, auch keine Professuren auf Lebenszeit.

    \item [Lehrprofessur]
          In einigen Bundesländern wurden vor einigen Jahren sogenannte Lehrprofessuren eingeführt.
          (auch Hochschuldozenturen und Juniordozenturen genannt). Diese werden mit W1 oder W2 bezahlt, haben
          in der Regel aber keine Mitarbeiterstellen und eine stark er\-höh\-te Lehrverpflichtung. Daher
          sind diese Stellen eher unattraktiv und wurden bisher kaum besetzt. Ebenso ist es momentan noch
          politisch stark umstritten ob diese Stellen überhaupt mit dem Humboldtschen Prinzip der Einheit
          von Forschung und Lehre vereinbar sein.

    \item [Mitarbeiterstellen]
          Wie das Gehalt wird bei der Berufung auch festgelegt, wie viele Mitarbeiter ein Professor bekommt.
          In der Regel kann er frei entscheiden, wie er die Mitarbeiterstellen auf Postdocs und Doktoranden
          aufteilt. Zum Beispiel kann in Bremen ein W2-Professor entscheiden, ob er einen Postdoc oder zwei
          Doktoranden einstellen möchte. Üblicherweise ist die genaue Anzahl der Mitarbeiter
          Verhandlungssache.

    \item [Pari Passu]
          Kann sich die Berufungskommission bei der Erstellung der Berufungsliste über die Reihenfolge zweier
          Kandidaten nicht einig, kommt es manchmal vor, dass beide \emph{pari passu} gesetzt werden. Pari passu
          bedeutet soviel wie "`gleichrangig"'. Man hat dann z.\,B. zwei zweite Plätze. Oftmals wird dann ein
          \emph{Sperrvermerk} gesetzt.

    \item [Sperrvermerk]
          Ein Sperrvermerk wird auf die Berufungsliste gesetzt, wenn die Berufungskommission noch einmal
          über die Liste befinden möchte. Dies kann zum Beispiel der Fall sein, wenn die Berufungskommission zwei Kandidaten
          pari passu gesetzt hat und erst bei Absage des Erstplatzierten über die Reihung entscheiden
          möchte.

    \item [Tenure Track]
          Oft hört man die Professoren in der Berufungskommission Begriffe sagen, wie \emph{tenure} oder \emph{tenure track}.
          Diese Begriffe stammen aus dem angelsächsischen System. Tenure bezeichnet eine unbefristete Stelle,
          vergleichbar mit "`Beamter auf Lebenszeit"'. Tenure Track bezeichnet eine (auf fünf oder sechs) Jahre
          befristete Stelle, die anschließend bei positiver Evaluation in Tenure umgewandelt wird.
\end{description}

%%% Local Variables:
%%% mode: LaTeX
%%% TeX-master: "../bkhandbuch"
%%% End:
