\section{Allgemeines}
Dieses Handbuch ist für \emph{DICH} gedacht. Sein alleiniger Zweck ist es, dich bei deiner Arbeit in deiner Berufungskommission (BK) zu unterstützen. Zunächst versuchen wir dir daher das notwendige Hintergrundwissen über den Ablauf und die rechtlichen Rahmenbedingungen zu vermitteln, denn dieses Wissen haben auch alle anderen Mitglieder in der Kommission. Daneben geben wir Empfehlungen und Erfahrungen aus unseren bisherigen selbst miterlebten Berufungskommissionen weiter. Dieser Schatz an Erfahrungen summiert sich innerhalb des Arbeitskreises auf weit über 50 Kommissionen von W1 bis W3 Professuren, über außerplanmäßige bis Honorarprofessuren, über Kampfabstimmungen bis Einigkeit.

Viele dieser Empfehlungen und Hinweise sind jedoch situationsabhängig zu sehen. Nur allein du kannst entscheiden, was zu einem bestimmten Moment sinnvoll ist. Denn das Gelingen einer Berufungskommission liegt zum Teil auch in deiner Verantwortung. Du hast dafür zu sorgen -- wie auch alle anderen Mitglieder der Kommission -- einen fähigen Hochschullehrer auszuwählen, welcher eine exzellente Lehre macht und nebenbei auch in der Forschung die ein oder anderen Lorbeeren vorzuweisen hat. Denn wenn man sich bei einer Berufung für ein faules Ei entscheidet, dann wird man es bis zur Pensionierung nicht mehr los.

Bevor wir nun mitten in das Thema starten, möchten wir dir mit auf den Weg geben, dass wir uns sehr freuen, dass du dich bereit erklärt hast, an einer Berufungskommission teilzunehmen. Es ist zwar ein großer Teil Verantwortung und es kostet auch einiges an Zeit, aber die Erfahrungen und Einblicke die man durch solch eine Kommission gewinnen kann wiegen dieses bei weitem auf. Und zu guter Letzt kannst du dich später damit rühmen, den "`Super-Prof"' an deine Uni geholt zu haben ;-)

%%% Local Variables:
%%% mode: LaTeX
%%% TeX-master: "../bkhandbuch"
%%% End:
