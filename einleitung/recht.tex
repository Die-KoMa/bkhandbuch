\section{Rechtliche Rahmenbedingungen}
% TODO mit Österreich und Schweiz checken!
In Deutschland gibt es in jedem Bundesland ein Hochschulgesetz. Durch dieses Gesetz werden die rechtlichen Rahmenbedingungen für die Berufung neuer Professoren festgelegt. Ebenso sollte an deiner Hochschule eine Berufungsordnung existieren. In dieser sind dann weitere, feinere Regelungen zum Ablauf der jeweiligen Berufungsverfahren zu finden. Insbesondere sind dieses folgende Punkte:
\begin{itemize}
    \item Zusammensetzung der Kommission (insb. studentische Mitglieder und Vertretungsmöglichkeiten)
    \item Ablauf der Kommission, Fristen, Mitspracherecht und deine Aufgaben als Mitglied der Kommission
    \item Wirkung findende Gleichstellungsgesetze\footnote{Anmerkung: Da oft in den Professoren- und Mitarbeiterreihen nur wenige Frauen zu finden sind, können Studierende oftmals einen besonderen Einfluss auf die Kommission ausüben, da sie die Möglichkeit haben weibliche Mitglieder zu entsenden.}
\end{itemize}
In den weiteren Abschnitten möchten wir nun auf jeden Teil der BK kurz eingehen und einige "`Best Practices"' vorstellen:

%%% Local Variables:
%%% mode: LaTeX
%%% TeX-master: "../bkhandbuch"
%%% End:
