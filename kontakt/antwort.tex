\section{Deine Fachschaft erhält eine Anfrage}
Auch wenn du momentan in keiner Berufungskommission sitzt, so kann es sein, dass die Anfrage einer anderen Fachschaft zu einem deiner Dozenten bei euch ankommt. Wenn dieses geschieht, so bitten wir dich bei der Antwort folgendes zu berücksichtigen:
\begin{itemize}
    \item Beantworte alles so, wie du es selbst verantworten kannst und es selbst von jeder anderen Fachschaft erwarten würdest.
    \item Entweder \emph{antworte ehrlich} oder antworte, dass du \emph{dich nicht äußern möchtest}.
    \item Wenn du nicht schriftlich antworten möchtet, aber mündlich Fragen beantwortet würdest, so teil dieses bitte der fragenden Fachschaft mit, ggf. auch eine Kontaktperson.
    \item Wenn du Bedenken bei deiner Antwort hast, so schreibt dieses ruhig in die Antwort hinein -- Es sollte sowieso klar sein, dass diese Antwort ausschließlich für das studentische Mitglied in der Kommission bestimmt ist. Es ist auch kein Problem, wenn du anonym bleiben möchtest.
    \item Wenn rechtlich möglich, schick bitte Evaluationen mit.
    \item Bitte suche jemanden, der bei dem Bewerber bereits Vorlesungen/ Seminare besucht hat und konkret etwas zu
          \begin{enumerate}
              \item Vorbereitung des Dozenten,
              \item Tafelanschrieb,
              \item Struktur der Vorlesung und
              \item Existenz bzw. Qualität des Skripts
          \end{enumerate}
          schreibt.
    \item Schreib bitte über deine Erfahrungen mit dem Dozenten/das Klima in der Vorlesung/ die Erreichbarkeit außerhalb der Vorlesungen und etwas zu den Sprechzeiten.
    \item Wenn möglich, schreib bitte auch etwas zur Erfahrung mit dem Dozenten in der universitären Selbstverwaltung.
\end{itemize}

%%% Local Variables:
%%% mode: LaTeX
%%% TeX-master: "../bkhandbuch"
%%% End:
