\section{Rechtliches}\label{sec:rechtliches}
Gerade bei Anfragen an andere Fachschaften herrscht oft große Rechtsunsicherheit. Folgendes können wir zu den häufigen Fragen sagen:
\begin{itemize}
    \item Eine Anfrage bei anderen Fachschaften zu der Lehrqualität eines Bewerbers ist i.\,d.\,R. rechtlich unbedenklich\footnote{Dabei sollte man vorsichtshalber nicht erwähnen, dass es sich bei der Person um einen Bewerber handelt (wobei sich das der anderen Fachschaft aus dem Kontext ergeben wird). Mittlerweile
              kann man im Internet auch Listen finden, auf denen aktuelle Berufungsverfahren inklusive Bewerber und Listenplatzierte aufgeführt werden.}. Auch das Einholen der Evaluationen eines Bewerbers ist hier möglich. Die Herausgabe muss aber unter Hinblick des Datenschutzes erfolgen und wird daher bei internen Evaluationen nur mit Zustimmung des Bewerbers möglich sein.

    \item Der Höflichkeit halber wäre es angebracht, den Bewerber zu fragen, ob er etwas dagegen habe, dir die Evaluationen zusenden zu lassen.

    \item Wenn du Informationen oder Evaluationsergebnisse einer anderen Fachschaft erhalten hast, so sind das zunächst einmal deine persönlichen Informationen. Wenn du diese Informationen in das Berufungsverfahren einbringen möchtest, so ist der rechtlich unbedenklichste Weg, wenn du dem Bewerber während des Interviews passende Fragen stellst.

    \item In die andere Richtung sind Informationen aus der Berufungskommission heraus dann unbedenklich, wenn sie während eines öffentlichen Teils bekannt geworden sind. So sind alle Informationen, welche die Berufungskommission während der öffentlichen Probevorlesungen/Fachvorträge gewinnt unproblematisch. Auch Informationen aus den öffentlichen Sitzungen der Berufungskommission sind hier nicht als bedenklich einzustufen. Nicht zulässig wäre aber eine Weitergabe aus nichtöffentlichen Teilen, insbesondere aus Interviews zur außerfachlichen Qualifikation, Zitaten aus Gutachten, etc.

    \item Generell darf aber jeder seinen persönlichen Eindruck von Bewerbern wiedergeben.
\end{itemize}

%%% Local Variables:
%%% mode: LaTeX
%%% TeX-master: "../bkhandbuch"
%%% End:
