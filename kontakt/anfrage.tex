\section{Du möchtest eine Anfrage schreiben}\label{sec:anfrage}
Eine Anfrage kannst du im deutschsprachigen Raum am einfachsten per E-Mail stellen. Ein entsprechendes Beispiel findest du im Anhang auf Seite~\pageref{sec:musteranfrage}. Im Regelfall solltest du die entsprechende Adresse ganz einfach mit der Internetsuchmaschine deiner Wahl finden. Falls dieses nicht möglich ist könntest du aber Folgendes versuchen:
\begin{itemize}
    \item Frag im Büro deiner Bundesfachschaftenkonferenz nach der Adresse. Wir von der KoMa haben beispielsweise ein Adressverzeichnis aller deutschsprachiger Mathematikfachschaften.
    \item Ansonsten kannst du auch im Adressreader des fzs nachschauen: \url{http://www.adressreader.de/}. Dieser wird vom \emph{freier zusammenschluss von studentInnenschaften e.\,V.} angeboten und ist online kostenlos verfügbar.
\end{itemize}
Sollte ein Kontakt per E-Mail nicht möglich sein kann man übrigens auch die Post-Methode versuchen. Diese kann in solchen Fällen gegebenenfalls zum Erfolg führen.


%%% Local Variables:
%%% mode: LaTeX
%%% TeX-master: "../bkhandbuch"
%%% End:
