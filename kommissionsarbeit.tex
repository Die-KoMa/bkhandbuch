\chapter{Kommissionsarbeit}\thispagestyle{fancy}
Anders als man erwarten könnte, beginnt die Arbeit für eine Berufungskommission bereits lange vor der ersten Sitzung. Denn schon vorher muss die Stelle genehmigt werden und auch in allen an der Ausschreibung beteiligten Statusgruppen wird schon im Vorhinein gemauschelt, wer sich denn an der Kommission beteiligen sollte. Dieses trifft natürlich auch für die Studierenden zu.

Dieses Kapitel führt im Folgenden chronologisch durch den Ablauf einer Berufungskommission und zeigt die einzelnen Etappen auf, an denen ihr besonders gefragt seid.

\section[Zusammensetzung und Ausschreibung]{Zusammensetzung der Kommission und Ausschreibung der Stelle}
\sectionmark{Zusammensetzung und Ausschreibung}
Die Ausschreibung und Beantragung bzw. Einrichtung einer neuen Stelle ist üblicherweise ein langwieriger Prozess. Wir gehen hier davon aus, dass bereits eine entsprechende Instanz deiner Hochschule die Ausschreibung einer Professur angeordnet bzw. genehmigt hat (z.\,B. Dekanat, Präsidium, Rektorat).

Für die Besetzung dieser Stelle werden an öffentlichen Hochschulen Kommissionen eingesetzt, welche jeweils eine Liste mit Vorschlägen und eine entsprechende Reihung erarbeitet. In Deutschland -- je nach Bundesland und geltendem Hochschulgesetz kann die Zusammensetzung der Kommission und Gestaltung der Arbeit innerhalb der Kommission recht unterschiedlich ausfallen -- befindet sich stets mindestens ein studentischer Vertreter in solch einer Kommission\footnote{Wir gehen davon aus, dass sich an dieser Tatsache auch in Zukunft nichts ändern wird, verweisen aber auf das Publikationsdatum dieses Handbuches.}. In diesem Abschnitt soll nun der Prozess der Kommissionszusammensetzung, aber auch der Auswahl eines geeigneten studentischen Mitgliedes genauer beleuchtet werden.

\subsection{Mitglieder der Kommission}
Die Zusammensetzung der Kommission wird üblicherweise im Fakultätsrat oder Fachbereichsrat vorgenommen. Dort nominieren die jeweiligen Statusgruppen ihre Vertreter -- also jeweils Professoren, Mitarbeiter und Studierende -- und wählen diese nach Statusgruppen getrennt. Die Anzahl der Vertreter der jeweiligen Statusgruppen regelt dabei eure \emph{Berufungsordnung}.

Neben den Vertretern der einzelnen Statusgruppen kann eure Berufungsordnung noch weitere Mitglieder vorschreiben. Eventuell wünscht sich aber auch der Fakultätsrat oder Fachbereichrats weitere externe oder beratende Mitglieder, welche sich an der Kommission beteiligen sollen. Solche Mitglieder können etwa folgende sein (die Bezeichnungen variieren von Hochschule zu Hochschule):

\begin{description}
    \item [Beratendes Mitglied]
          Ein beratendes Mitglied kann eine Person aus der eigenen Hochschule, also z.\,B. ein weiterer Professor oder aber auch ein weiterer Student sein.

    \item [Fakultätsübergreifende Mitglieder]
          Diese Mitglieder aus einer anderen Fakultät --- gele\-gent\-lich auch "`Wachhunde genannt"' --- sollen im Auftrag der Hochschulleitung dafür sorgen, dass das Berufungsverfahren ordnungskonform und zügig durchgeführt wird. Diese sind jeder nicht an jeder Hochschule zu finden.

    \item [Auswärtiger Professor]
          Dieses bezeichnet einen Professor einer anderen Hochschule. Gerade bei kleinen Fachbereichen wird gerne auf diese Möglichkeit zurückgegriffen, um eine hinreichende Fachkompetenz in der Kommission zu haben. Auswärtige Professoren haben dabei in der Regel Stimmrecht.

    \item [Auswärtiger Experte]
          Experten sind meist Personen außerhalb der Hochschullandschaft, welche mit oder ohne Stimmrecht in die Kommission aufgenommen werden. Beispielsweise bei Stiftungsprofessuren wird gerne ein Vertreter der entsprechenden Stiftung mit in die Kommission aufgenommen.

    \item [Frauenbeauftragte]
          Die Frauenbeauftragte (oder Gleichstellungsbeauftragte, je nach Hochschule) soll die Rechte der Frauen vertreten. Sie hat in der Regel kein Stimmrecht, darf aber ein Votum zur Frage der Berücksichtigung von
          Frauen abgeben. In vielen Bundesländern darf Sie auf Einladung von Frauen bestehen.

    \item [Schwerbehindertenbeauftragte]
          Der Schwerbehindertenbeauftragte soll die Rechte schwerbehinderter Bewerber vertreten. Er nimmt in der
          Regel nur dann an der Berufungskommission teil, wenn und solange schwerbehinderte Bewerber im Verfahren
          sind. Der Schwerbehindertenbeauftragte hat kein Stimmrecht.

\end{description}

Ob die o.\,g. Mitglieder Stimmrecht haben oder nicht, hängt von der jeweiligen Berufungsordnung ab. Bei der Zusammensetzung der Kommission gilt jedoch immer: die Hälfte der Mitglieder müssen Professoren sein. Daher kann sich durch das Hinzuziehen von weiteren Mitgliedern ggf. die Stimmengewichtung der professuralen Mitglieder innerhalb der Kommission ändern. Hier müssen wir aber auf deine jeweilige Berufungsordnung verweisen.

\subsection{Das studentische Mitglied}
Es ist nicht immer leicht sich für ein studentisches Mitglied für die Kommission zu entscheiden bzw. es überhaupt zu finden. Folgendes sollte bei der Auswahl eines studentischen Mitgliedes aber immer beachtet werden:
\begin{itemize}
    \item Eine Berufungskommission benötigt punktuell sehr viel Zeit. Es ist insbesondere nur schwer möglich, neben dem Studium sämtliche Kommissionssitzungen, Bewerbungsvorträge und Bewerbungsgespräche zu besuchen. Daher solltest du schauen, ob die Möglichkeit eines Stellvertreters in deiner Berufungsordnung gegeben ist. Falls dieses nicht möglich ist, so kann man oft ein weiteres beratendes studentisches Mitglied in die Kommission wählen lassen.

    \item Es ist ggf. möglich Vergünstigungen und Erleichterungen für die Arbeit in einer Berufungskommission zu erhalten. So kann man mit den Dozenten absprechen, dass man aufgrund der hohen Arbeitsbelastung einen Übungszettel nicht bearbeiten oder später abgeben darf. Insbesondere in der Woche, in der die Vorträge und Interviews stattfinden, kommt man kaum zum Studieren. Wenn man Kurse bei Professoren besucht, die selber in der Berufungskommission sitzen, hat man gute Chancen etwas Arbeitserleichterung zu bekommen. Sollte es in einigen deiner Kurse Anwesenheitspflicht geben, kannst du für Sitzungen der Berufungskommission davon befreit werden.

    \item Ihr solltet euch immer \emph{selbst} -- ohne Beeinflussung von außen -- überlegen, wen ihr in diese Kommission schicken möchtet. Diese Person sollte aber das Grundstudium, bzw. die erste Hälfte des Bachelors abgeschlossen haben. Dieses wird teilweise in den Berufungsordnungen auch gefordert. In jedem Falle ist dieses aber sinnvoll, da so sicher gestellt wird, dass bereits eine genügende Fachkompetenz und Erfahrung beim studentischen Mitglied vorhanden ist.

    \item Ein wichtiger Punkt ist die Frauenquote, welche stets eine Kardinalsfrage bei der Zusammenstellung der Kommission ist. Meist haben die Professoren oder Mitarbeiter nur einen sehr geringen Frauenanteil. Somit können Studierende allein über das Einsetzen von weiblichen Kommissionsmitgliedern ggf. ein besonderes Gewicht in dieser Kommission einnehmen.
\end{itemize}

\subsection{Ausschreibung der Stelle}
Die Stellenausschreibung wird üblicherweise in der ersten Kommissionssitzung gemeinsam mit einem \emph{Kriterienkatalog} verabschiedet. Bei beidem solltet ihr darauf achten, dass die pädagogische Eignung explizit genannt wird. Ebenso sollte der Lehrbedarf im Service-Bereich\footnote{Service-Bereich sind in diesem Sinne Veranstaltungen für andere Studiengänge, wie z.B. Mathe für Wirtschaftswissenschaftler.} auch als Aufgabe in der Ausschreibung stehen.

Neben der pädagogischen Eignung für die abzudeckenen Veranstaltungen solltet ihr zudem darauf achten, dass die Stelle breit genug ausgeschrieben wird. Ein ab und an auftretendes Problem ist es nämlich, dass teilweise schon sehr genaue Vorstellungen existieren, wer die Stelle annehmen könnte. Wenn nun auch die Ausschreibung explizit auf diese Person zugeschnitten ist verhindert das die Möglichkeit einer Auswahl.

\subsubsection*{Kriterienkatalog}\label{kriterienkatalog}
Der Kriterienkatalog ist die wesentliche Entscheidungsgrundlage für die Auswahl und Reihung der Listenkandidaten. Die Festlegung der Kriterien muss in der Regel in der ersten Kommissionssitzung -- insbesondere vor Beschluss der Ausschreibung -- erfolgen und dient zur Bewertung der Bewerber. Die darin enthaltenen Kriterien sind die alleinigen Bewertungskriterien, welche die Kommission im Weiteren anlegen darf. Kriterienkataloge gibt es sowohl mit als auch ohne Gewichtung und Reihung der einzelnen Punkte.

\begin{center}
    \fbox{
        \begin{minipage}{0.8\textwidth}
            \begin{itemize}
                \item Wissenschaftliche Qualifikation
                \item Passendes Arbeitsgebiet gemäß Ausschreibung
                \item {\bf Lehr- und Vortragserfahrung sowie pädagogische Eignung}
                \item Fachübergreifende Bezüge zu anderen an der Hochschule vertretenden Arbeitsgebieten (Potential zur Bildung von Kooperationen)
                \item Erfahrung in der Einwerbung von Drittmitteln
                \item Internationale Erfahrung
            \end{itemize}
        \end{minipage}
    }
    \begin{center}
        Ein Beispiel für einen Kriterienkatalog.
    \end{center}
\end{center}

%%% Local Variables:
%%% mode: LaTeX
%%% TeX-master: "../bkhandbuch"
%%% End:

\section{Bewerbungsunterlagen}\thispagestyle{fancy}
Nachdem die Stelle ausgeschrieben worden ist kommen ca. zwei Monate später die Bewerbungsunterlagen im entsprechenden Sekretariat an. Oft erstellt eine Schreibkraft direkt aus diesen Daten Übersichtstabellen, um Bewerber leichter vergleichen zu können. Dieses ist jedoch nicht immer der Fall und auch die Vergleichsinformationen zur Lehre der einzelnen Dozenten schwanken stark. Wir empfehlen euch daher aus den folgenden Punkten die für euch wichtigsten Vergleichspunkte zu wählen und eine eigene Vergleichstabelle zur Lehre der einzelnen Bewerber zu erstellen.

Folgende Merkmale können hierzu sinnvoll verwendet werden:
\begin{itemize}
    \item Bisherige Lehrerfahrung
    \begin{itemize}
        \item Anzahl der bereits gehaltenen Lehrveranstaltungen? (je Grund- und Hauptstudium)
        \item Größe der bisher größten Veranstaltung?
        \item Welche Vorlesungen hat der Bewerber bereits angeboten?
        \item Wie viele (fachspezifische) Seminare hat der Bewerber schon angeboten?
        \item Hat der Bewerber schon einen ganzen Vorlesungszyklus gelesen?
        \item Hat der Bewerber Erfahrung mit Lehramtsvorlesungen?
        \item Hat der Bewerber Erfahrung im Service-Bereich (Lehrexporte für andere Fächer)?    
    \end{itemize}
    \item Anzahl der bisher betreuten Abschlussarbeiten? (je Bachelor und Master)
    \item Haben Abschlussarbeiten bereits zu Publikationen geführt?
    \item Hat der Bewerber Lehrevaluationen angehängt? Sind diese aktuell und was steht darin? 
    \item Hat der Bewerber Lehrpreise erhalten oder Drittmittel für Lehre eingeworben?
    \item Beinhaltet die Bewerbung einen Abschnitt zur Lehre\footnote{Ein solches "Teaching Statement" kann man auch explizit bei der Bewerbung fordern.}? Ist dieser überzeugend?
\end{itemize}

Diese Vergleichstabelle sollte zur zweiten Sitzung der Kommission fertig sein. Ebenso ist eine Gruppierung der Bewerber anhand dieser Liste in Gruppe A, B und C (von gut über mittel nach schlecht) sinnvoll und erfolgt oft in der Kommission.
Deine Aufgabe ist es nun, die bestmöglichen Kandidaten zu finden, wobei dein Hauptaugenmerk auf der Lehre liegen sollte. 
In in der zweiten Sitzung wird zwar vor allem die Qualität und Quantität der bisherigen Publikationen der Bewerber bewertet, allerdings muss auch die Lehrbefähigung gegeben sein und Zweifel daran solltest du schon an dieser Stelle deutlich anbringen.
Ein Abschnitt zur Lehre kann zunehmend von den Bewerbern erwartet werden und sollte ebenfalls in die Bewertung einfließen.

Aus Gründen der Gleichstellung wird manchmal eingebracht, dass weitere Personen eingeladen werden sollten\footnote{Sollten etwaige Quoten mit der bisherigen Bewerbendenlage nicht erfüllbar sein, so deutet das auf einen Mangel der bisherigen Bewerbendensuche hin und eine Neueröffnung des Verfahrens könnte notwendig sein. Teilweise können auch gezielt geeignete Personen eingeladen werden, sich zu bewerben.}.
Ihr solltet euch dabei dafür einsetzten, dass nur Personen eingeladen werden, die auch tatsächlich eine Chance haben, d.h. für die Stelle geeignet sind.
Insgesamt muss die Qualität der Bewerbungen an erster Stelle stehen und schlechte bis mittelmäßige Bewerber sollten nicht in die zweite Runde gelangen.

Nach dieser Sitzung wird es dann (meistens) mindestens vier Wochen Pause geben, bis Termine mit allen Bewerbern vereinbart wurden, an denen sie sich persönlich bei euch vorstellen. Diese Zeit sollte mit dem Einholen von Informationen sinnvoll gefüllt werden.

%%% Local Variables:
%%% mode: LaTeX
%%% TeX-master: "../bkhandbuch"
%%% End:

\section{Informationen einholen}
Es ist üblich, dass die studentischen Mitglieder der Berufungskommission Informationen über die Lehrqualität der Bewerber an den jeweiligen Heimuniversitäten einholen.
Das sollte vor dem Bewerbungsgespräch geschehen, damit du auf Basis dieser Informationen Fragen stellen kannst.

Je nach Ergiebigkeit der einzelnen Quellen bietet es sich an mehrere der folgenden Möglichkeiten zu kombinieren:
\begin{description}
      \item [Bewerbungsunterlagen]
            Vor den Gesprächen solltest du die Bewerbungsunterlagen der Bewerber nocheinmal gründlich lesen.
      \item [Uniwebsiten]
            Viele Lehrende stellen ihre Lehrmaterialien im Internet zur Verfügung. Diese können eine gute Grundlage für die Beurteilung der Lehrqualität sein. Hierzu zählen Vorlesungsskripte, Übungsblätter, Klausuren und andere Materialien.
      \item [Youtube]
            Einige Lehrende stellen auch Vorlesungsaufzeichnungen und Videos bereit. Diese sind eine gute Möglichkeit, um einen Eindruck von der Lehrweise des Bewerbers zu bekommen.
      \item [Suchmaschinen]
            Dieses kann der erste Anlaufpunkt sein, um zunächst einmal ein grobes Bild über den Bewerber zu erhalten.
      \item [Forschungsportale]
            Profile auf Portalen wie Google Scholar können dir helfen, die wissenschaftliche Qualifikation des Bewerbers zu beurteilen, dabei wird besonders der $h$-Index oft herangezogen.
      \item [Webseite des Bewerbers]
            Sofern die Bewerbungsunterlagen nicht besonders ergiebig sind lassen sich hier meist die bisher gehaltenen Veranstaltungen finden. Auch findet sich oft ein Überblick über bisherige akademische Leistungen.
      \item [Fachschaft kontaktieren]
            Hier ist die E-Mail das Medium der Wahl. Vorteil bei solchen Nachfragen ist, dass man gleich Informationen über das Engagement in der universitären Selbstverwaltung erhalten kann. Eine Muster-E-Mail für eine solche Anfrage findest du im Abschnitt~\ref{sec:anfrage}.
      \item [Studentische Gutachten]
            Bei ausländischen Bewerbern kann man darum bitten, studentische Gutachten über die Lehrqualität mit einzureichen. Im angelsächsischen Raum ist dieses Vorgehen üblich. Jedoch ist hier zu beachten, dass der Bewerber meist selber den Studierenden aussucht, der das Gutachten schreibt.
\end{description}

Für alle auftretenden rechtlichen Fragen verweisen wir auf Abschnitt~\ref{sec:rechtliches}.

Im deutschsprachigen Raum findet man an fast jeder Hochschule eine Fachschaft, Studiengangsausschuss oder Fachschaftsverein. 
Am einfachsten sind diese meist per E-Mail zu erreichen. 
Sei dabei umsichtig, denn Bewerbungsverfahren sind in der Regel vertraulich.
Nicht alle Fachschaften sind bereit, Informationen zu geben.
Sollte deine Fachschaft in so einer Sache angesprochen werden, müsst ihr diese nicht beantworten, solltet aber unter keinen Umständen falsche Informationen geben.
Eine Auskunft von der Fachschaft kann dir Informationen zum Umgang mit Studierenden und über die Lehrqualität des Bewerbers geben, die in den Bewerbungsunterlagen nicht enthalten sind.
In der Regel darfst du solche Informationen im Berufungsverfahren selbst nicht verwenden, sie können aber helfen die Bewerbenden besser zu verstehen.

Daneben existieren aber noch einige weitere Möglichkeiten Informationen über die Bewerber zu erhalten, die nicht direkt in den Bewerbungsunterlagen stehen.
Beispielsweise kann man auch die Bewerbenden direkt fragen, allerdings sollte ein solches Vorgehen mindestens mit der Kommissionsleitung abgesprochen sein.


%%% Local Variables:
%%% mode: LaTeX
%%% TeX-master: "../bkhandbuch"
%%% End:

\section{Die Vorträge an sich}
An jeder Hochschule gibt es Fachvorträge, in denen die Bewerber\gstar{innen} ihre aktuellen Forschungsgebiete präsentieren, ihre Erfolge loben und auch Ausblicke für ihre nächsten Ziele geben. Neben diesem Fachvortrag müssen die Bewerber\gstar{innen} auch an fast allen Hochschulen eine Lehrprobe halten, welche dazu dient, ihre pädagogischen Fähigkeiten unter Beweis zu stellen. Leider existieren immer noch Hochschulen, an denen zu diesem Zweck nur ein Teil des Fachvortrages herhalten muss. Dieser Teil muss sich dann spezifisch an Studierende richten. In einem solchen Fall solltet ihr euch unbedingt für eine Lehrprobe einsetzten und darlegen, dass diese essenziell für eine informierte Bewertung der pädagogischen Eignung ist. Wir gehen im Folgenden auf beide Fälle getrennt ein.

Für jede Art dieser Vorträge ist zu sagen, dass es für euch immer hilfreich ist, wenn weitere Studierende während der Vorträge anwesend sind und euch bei der Bewertung dieser Vorträge unterstützen\footnote{Beispielsweise kann man Fragebögen (siehe~\ref{sec:fragebogen}) austeilen, um Notizen bitten oder einfach nach dem Vortrag in die Runde fragen.}. Daher solltest du dich auch bemühen, die Vorträge möglichst gut bei den Studierenden in deinem Fachbereich publik zu machen. Wunderbar eignen sich dafür etwa Mailverteiler, Aushänge oder Ankündigungen in Vorlesungen.

\subsection{Lehrprobe}
Eine Lehrprobe ist das Halten einer Vorlesung vor einer ausreichend großen Anzahl an Studierenden im Rahmen eines Berufungsverfahrens. Die Themen der Lehrprobe sollten aus dem Grundstudium kommen um es allen anwesenden Studierenden zu ermöglichen, sich auf die Qualität des Vortrags zu konzentrieren. Eine Lehrprobe kann etwa als Vertretung einer Analysis oder Lineare Algebra Vorlesung gehalten werden, aber auch als eigenständiger Lehrvortrag. Bei der Auswahl der Themen muss eine Abwägung getroffen werden. Spezifische Themen führen zu einer besseren Vergleichbarkeit. Auf der anderen Seite können die Bewerber\gstar{innen} durch weniger konkrete Anweisungen zeigen, dass sie in der Lage sind, eine sinnvolle Themenauswahl zu treffen. Sollte eine Vorlesung zu einem bestimmten Thema gefordert werden, müssen die Bewerber\gstar{innen} im Vorherein unbedingt durch den Vorsitzende\gstar{n} darüber informiert werden, welches Vorwissen vorausgesetzt werden kann, bzw. was für Themen vor der geforderten Vorlesung behandelt wurden. Dies ermöglicht es, herauszufinden, ob die Bewerber\gstar{innen} in der Lage sind, eine Vorlesungsreihe sinnvoll zu strukturieren.

Wir haben hier zwei Beispiele:

\subsubsection{Uni Flensburg}
\begin{itemize}
    \item Lehrprobe wird bei allen Bewerber\gstar{innen} gefordert
    \item findet nach dem Fachvortrag statt
    \item Bewerber\gstar{innen} dürfen sich selber das Thema aussuchen, es muss aber mit der Ausrichtung der Stelle verbunden sein
    \item Zielgruppe sind Bachelor-Student\gstar{innen} im Hauptstudium
    \item Dauer: 45\,min Vortrag zzgl. 15\,min für Fragen
    \item im Durchschnitt sind 30 - 40 Studierende anwesend
    \item Angekündigt wird von der Fachschaft in einzelnen Veranstaltungen und durch Aushang
\end{itemize}

\subsubsection{Uni Bremen}
\begin{itemize}
    \item pro Bewerber\gstar{in} wird sich ein halber Tag Zeit genommen
    \item Dauer: 45\,min Probevorlesung
    \item Inhalt des Vortrags (im laufe der Zeit)
          \begin{itemize}
              \item Es werden zwei Themen zur Auswahl angeboten: "`Satz über impl. Funktion"' oder "`Satz von Picard-Lindelöff"'
              \item Bewerber\gstar{innen} halten Ersatzvorlesung für Grundstudiumsvorlesung (Analysis 1/2 oder Lineare Algebra 1/2)
          \end{itemize}
    \item am Ende werden zur Bewertung Fragebögen an zuhörende Student\gstar{innen} verteilt
\end{itemize}


\subsection{Fachvortrag}
Sofern keine Lehrprobe gehalten wird, solltest du darauf bestehen, dass sich ein signifikanter Teil des Fachvortrages an Studierende richten muss. Hierbei ist ein \emph{signifikanter Teil} eines 45 Minuten Vortrages mindestens 15 Minuten lang\footnote{Und mit mindestens meinen wir wirklich, dass man keinesfalls darunter gehen darf.}. Zielgruppe sollten auch in diesem Bereich Studierende mit abgeschlossenem Grundstudium sein.

Du solltest außerdem darauf achten, dass der/die Vorsitzende der Kommission daran denkt, den Vortragenden diese Zeiteinteilung mitzuteilen. Dies wird ab und an schon mal vergessen.

\subsection{Bewertung der Vorträge}
Der Vortrag/die Lehrprobe sind üblicherweise hochschulöffentlich, jede Student\gstar{in} kann sich
diese also anhören. Um bewerten zu können, wie gut der Vortrag ankommt, kann man die zuhörenden Studierenden
im Anschluss befragen. Einen Musterfragebogen findest du in Anhang~\ref{sec:fragebogen}.

Grundsätzlich gilt, dass du dir immer Notizen machen solltest. Zwischen den Vorträgen und den späteren Treffen der Kommission, in denen ebenfalls argumentiert wird, können Monate vergehen.
Da es später notwendig ist, begründet über die Bewerber\gstar{innen} zu sprechen, sollten diese Notizen aber nicht den Inhalt wiedergeben, sondern deine Eindrücke über die Qualität des Vortrags. Dies ist etwas ungewohnt. Deshalb haben wir ein paar Punkte zusammengestellt, auf die geachtet werden kann:
\begin{itemize} % eventuell erweitern/überprüfen, dass es sinnvoll ist.
    \item Welche Medien verwendet die Bewerber\gstar{in} und setzt er/sie sie sinnvoll ein?
    \item Spricht die Bewerber\gstar{in} dem Publikum zugewandt?
    \item Übersichtlichkeit der Darstellung (z.\,B. Farben, Handschrift, Folien)
    \item Wie geht die Bewerber\gstar{in} auf Fragen ein?
    \item Wird abgelesen oder frei gesprochen?
    \item Wie ist die Motivation der Bewerber\gstar{in} und kann er/sie motivieren?
    \item Artikulationsvermögen (insb. bei nicht Deutsch-Muttersprachlern)
    \item Geschwindigkeit des Vortrages
    \item Ist ein roter Faden erkennbar?
    \item Ist der Vortrag gut strukturiert?
\end{itemize}
Erfahrungsgemäß übersieht man inhaltliche Fehler, wenn man auf alle genannten Punkte achtet. Deswegen ist es sinnvoll, sich die Arbeit aufzuteilen, wenn man mit mehreren Studierenden in der Kommission sitzt. Ansonsten werden inhaltliche Fehler aber in der Regel von den Professor\gstar{in} festgestellt.

\subsection{Fragen zum Vortrag}
Bereits während und auch direkt im Anschluss an die Fachvorträge werden üblicherweise Fragen gestellt. Auch ihr solltet diese Gelegenheit nutzen. Nützliche Fragen sind etwa:
\begin{itemize}
    \item Können Sie \dots\ noch einmal erklären. Vielleicht für einen Studierenden im 3. Semester?
    \item Was sollte ich von diesem Vortrag mitnehmen? In zwei Sätzen bitte.
    \item Wie ist die Relevanz des Vortrages in den entsprechenden Themenbereich einzuordnen?
    \item Ggf. könnt ihr eine Frage auf Englisch stellen um das Englisch der Bewerber\gstar{in} zu testen.
\end{itemize}

%%% Local Variables:
%%% mode: LaTeX
%%% TeX-master: "../bkhandbuch"
%%% End:

\section{Bewerbungsgespräch}
Im Anschluss an die Vorträge finden üblicherweise die Bewerbungsgespräche zwischen den Bewerbern und der Kommission statt.
Im Regelfall sind diese Gespräche vertraulich und hinter verschlossener Tür.
Ihr seid dann zur Verschwiegenheit verpflichtet und dürft nur mit (studentischen) Vertretern in höherrangigen Gremien über die Inhalte dieser Gespräche sprechen, welche auch Einblick in die internen Unterlagen der Berufungskommission haben.\footnote{%
    An einzelnen Hochschulen sind diese Gespräche dahingegen in die Hälften Forschung und Persönliches geteilt.
    Hierbei ist dann der erste Teil hochschulöffentlich.}

Der Vorsitzende der Kommission leitet üblicherweise das Gespräch und wechselt dabei auch zwischen den Themenbereichen.
Dies sorgt dafür, dass der Bewerber nicht 90\% der Zeit über sein Lieblingsgebiet sprechen kann.
Irgendwann wird der Vorsitzende dann zur Lehre überführen, spätestens hier solltet ihre eure Fragen stellen.
Zur besseren Vergleichbarkeit solltet ihr darauf achten, jedem Bewerber die gleichen Fragen zu stellen.

Sollte die Zeit knapp werden und wurde immer noch nicht über die Lehre gesprochen, solltest du die Initiative ergreifen und versuchen in diese Richtung zu lenken.

Wir haben im Folgenden einige Fragen gesammelt, aus welchen du gerne auswählen darfst und von denen du dich bei deinen eigenen Fragen inspirieren lassen kannst.
Da die Zeit meist knapp bemessen ist, solltet ihr bevorzugt Fragen stellen, aus deren Antworten ihr euch die beste Einsicht über den Bewerber erwartet.

\subsection{Fragenkatalog: allgemeine Fragen}
\begin{itemize}
    \item Was bedeutet für Sie gute Lehre?
    \item Haben Sie sich unsere Bachelor- und Masterprüfungsordnung angeschaut?
          Kennen Sie sich damit aus?\footnote{%
              Da Professoren erwarten, dass sich Bewerber mit den Forschungsschwerpunkten des Institutes beschäftigt haben, können wir dieses auch erwarten.%
          }
    \item Wie stehen Sie zu Studiengebühren?
    \item Wie stellen Sie sich ihren Übungs-/Vorlesungsbetrieb vor?
          (Präsenzübungen, Vorrechnen, Hausaufgaben, …)
    \item Was sollte ein Bachelor können, wenn er fertig ist?
          Was sollte ein Lehramtsstudent können, wenn er fertig ist?
    \item Welche Erfahrungen in der akademischen Selbstverwaltung haben Sie?
    \item Wie stellen Sie sich die Zusammenarbeit mit der Fachschaft vor?
    \item Differenzieren Sie zwischen Lehramts- und Vollfachstudenten?
          Wie gehen Sie auf die besonderen Bedürfnisse der Lehramtsstudenten ein?
    \item Haben Sie schon an hochschuldidaktischen Fortbildungen teilgenommen?
    \item In einer Veranstaltung fallen 80\% der Studenten durch die Klausur.
          Wie gehen Sie damit um?
    \item Wie können Studenten Sie mit Fragen erreichen?
          Wie viele Tage pro Woche sind Sie an der Hochschule?
          Können Studenten auch außerhalb der Sprechzeiten zu Ihnen kommen?
    \item Wie unterscheidet sich bei Ihnen eine Vorlesung mit 12 Studierenden von einer mit 400 Studierenden?
          Wie gehen Sie mit dem höheren Lärmpegel um?
    \item Nennen Sie spontan je ein Thema für eine Bachelor-, Master- und Staatsexamensarbeit.
    \item Welche Vorlesungen wollen/werden Sie anbieten?
          Was ist ihr Kanon?
          Werden Sie eine Spezialisierungssequenz anbieten?
    \item Wie bereiten Sie ihre Vorlesungen vor?
          Welche Materialien (z.\,B. Skript) gibt es zur Vorlesung?
    \item Können Sie interessante Themen für Vorlesungen anbieten, die sich in erster Linie an Lehrämtler richten?\footnote{%
              Dieses sollten interessante und lebensnahe mathematische Themen sein, mit denen man auch mal ein paar Schüler beeindrucken kann;
              jeder Matheprof sollte so etwas in petto haben}
    \item Welche Kompetenzen haben Sie im Bereich der digitalen Lehre?
    \item Wie gehen Sie mit den Ergebnissen der Lehrevaluation um?
          Wie werten Sie diese aus, welche Konsequenzen ziehen Sie daraus?
          Was halten Sie von derartigen Befragungen?
\end{itemize}

\subsection{Fragenkatalog: bei Bedarf}
\begin{itemize}
    \item Wie viele/welche Vorlesungen haben Sie schon gehalten?
          (sofern nicht in Unterlagen angegeben)
    \item Gab es bei Ihnen eine Lehrevaluation und können Sie uns diese zuschicken?
          Wie haben Sie dort abgeschnitten und was halten Sie von solchen Befragungen? (wenn nicht bereits offiziell vorhanden)
    \item Auf spezielle Tätigkeiten eingehen, wenn diese z.\,B. auf der Webseite des Bewerbers gefunden werden konnten (z.\,B. Prüfungsausschuss).
\end{itemize}

%%% Local Variables:
%%% mode: LaTeX
%%% TeX-master: "../bkhandbuch"
%%% End:

\section{Entscheidung vs. Gutachten}
Nach der Vorstellung der Bewerber entscheidet die Kommission, welche Bewerber weiter berücksichtigt werden sollen. Über diese werden bei verschiedenen namhaften Persönlichkeiten aus dem jeweiligen Bereich Gutachten über die Qualität der Bewerber eingeholt. Das Einholen dieser Gutachten dauert bis zu zwei Monate.

Sobald alle Gutachten eingetroffen sind, wird auf Grundlage dieser Gutachten eine Berufungsliste erstellt. Diese Liste hat in der Regel drei Plätze, wobei einzelne Plätze aber auch (wenn auch sehr selten) doppelt besetzt werden können. Sofern diese Liste durch die weiteren Gremien bestätigt wird, erteilt irgendwann der Präsidenten bzw. Rektor einen Ruf an den Erstplatzierten. Wenn dieser nicht annimmt, dann geht ein Ruf an den Zweitplatzierten usw.

Im Idealfall sollte diese Entscheidung aufgrund der Gutachten sehr einfach sein, indem optimalerweise alle Gutachter eine gleiche Reihung vorschlagen. In der Praxis erhält man aber teils widersprüchliche Aussagen, merkt dass Gutachter den Ausschreibungstext falsch verstanden haben oder kann klar aus dem Gutachten herauslesen, dass der Gutachter mit genau einem Kandidaten persönlich gut befreundet ist. Alles das muss in dieser Sitzung abgewogen werden. Die rechtlichen Rahmenbestimmungen werden bereits durch die Professoren sichergestellt. Anhand der Gutachten und deiner Eindrücke (die durch Notizen belegt sind) in den Probevorträgen bist du aber herzlich dazu eingeladen mitzudiskutieren wen du als passende Besetzung für die ausgeschriebene Stelle siehst.

Die fachliche Eignung der Kandidaten und die Passung in das Institut kannst Du als Student meist wenig bis kaum beurteilen. Daher solltest Du diese Kriterien den Professoren überlassen. Deine Stärke liegt in der Beurteilung der Lehrqualität.

Eines solltest du hierbei immer im Hinterkopf behalten: es kann/sollte/darf natürlich nur jemand berufen werden, der auch vernünftige Lehre macht. Dieses ist deine Hauptaufgabe und du bist gefragt um dieses sicherzustellen und ggf. dafür zu sorgen, dass die pädagogisch schwächeren Kandidaten auf die hinteren Listenplätze wandern oder ganz von der Liste fallen. Im Zweifelsfall ist aber oft noch ein Sperrvermerk herauszuhandeln. Du solltest übrigens auch darauf vorbereitet sein, dass du vielleicht schon zu Beginn der Sitzung gefragt wirst: "`Wie würden denn Sie aus studentischer Sicht die Bewerber gewichten?"'


\subsection{Begutachtung der Gutachten}
Dieser Abschnitt soll dir ein paar Hilfestellungen geben, wie man pro oder contra bestimmter Bewerber argumentieren kann.

Auch wenn in machen Gutachten steht, dass die Lehrqualität des Bewerbers hervorragend ist, so sind diese Gutachten immer mit Vorsicht zu genießen. Oft haben die Gutachter die einzelnen Bewerber nur auf Tagungen vortragen gesehen und kennen daher nicht die Qualität ihrer Vorlesungen.

\paragraph{Über die Gutachten}
\begin{itemize}
    \item Es gibt \emph{Einzelgutachten}, welche nur einen Bewerber begutachten. Diese fallen üblicherweise sehr gut bis hervorragend aus. Einzig ein neutrales bis negatives Einzelgutachten kann dir zu Entscheidungsfindung dienen.
    \item \emph{Vergleichende Gutachten} sind Gutachten, die mehrere der Bewerber vergleichen. Hierbei gibt der Gutachter meist auch seine persönliche Reihung an. Dieses sind mit die wichtigsten Entscheidungshilfen.
    \item Wenn ein Bewerber \emph{nicht habilitiert} ist muss im Gutachten stehen, dass er zu eigenständiger Lehre befähigt ist bzw. alle Voraussetzungen für eine Professur erfüllt.
    \item Die didaktische Eignung eines Bewerbers kann nur in den seltensten Fällen mit einem Standardgutachten begründet werden.
    \item Von den Bewerbern selbst vorgeschlagene Gutachter sollte mit Vorsicht genossen werden. Dieses sind meist Kandidaten für das Schreiben von Einzelgutachten. Einzelne Berufungsordnungen schließen von den Bewerbern vorgeschlagene Gutachter grundsätzlich aus.
    \item Jeder Bewerber sollte von wenigstens zwei Gutachtern begutachtet werden, davon mindestens bei einem Gutachter vergleichend.
\end{itemize}

\paragraph{Argumentation gegen einen Bewerber}
\begin{itemize}
    \item Anhand des Kriterienkatalogs (hier muss pädagogische Eignung drin stehen, vgl. Seite~\pageref{kriterienkatalog}).
    \item Ggf. hat der Bewerber keine Habilitation.
    \item Notizen aus Vortrag und Gespräch.
    \item Bewerber hat nur geringe Lehrerfahrung.
    \item Je nachdem was im Gutachten steht: Gutachten kann die Lehrqualität nicht beurteilen bzw. Gutachter bescheinigt sehr gute Lehre.
    \item Schlechter Eindruck bei Studierenden (z.\,B. durch Umfrage beim Vortrag).
    \item Evaluationsergebnisse.
    \item Auf formale Fehler achten.
    \item Mit anderen Kommissionsmitgliedern und Sonderbeauftragen (Gleichstellungsbeauftragte etc.) zusammenschließen.
\end{itemize}

\paragraph{Argumentation für einen Bewerber}
\begin{itemize}
    \item Kriterienkatalog: wenn eine gute Lehrerfahrung vorliegt.
    \item Ggf. haben andere Bewerber keine Habilitation.
    \item Notizen aus Vortrag und Gespräch.
    \item Gute Lehrerfahrung, insbesondere auch mit großen Vorlesungen.
    \item Eindruck von Studierenden (z.\,B. durch Umfrage bei Vortrag).
    \item Evaluationsergebnisse.
    \item Gegen die anderen Bewerber diskutieren.
    \item Je nachdem was im Gutachten steht: Gutachten kann die Lehrqualität nicht beurteilen bzw. Gutachter bescheinigt sehr gute Lehre.
\end{itemize}

Jedoch ist es so, dass auch ein hervorragender Hochschullehrer nur dann auf eine Stelle gesetzt werden kann, wenn er in das jeweilige Forschungsprofil passt. Wenn du merkst, dass ein Bewerber nicht auf die Stellenausschreibung passt, dann ist es nicht sinnvoll weiter für ihn zu kämpfen.

Weiter ist noch zu sagen, dass die Pausen hervorragend für Mauscheln und Kontakte knüpfen geeignet sind. Dort kann man manchmal in einem netten Plausch schneller zum Ziel kommen als direkt während der Sitzung. Je nachdem wie gut du mit den Professoren klar kommst, kannst du diese auch in die ein oder andere Richtung lotsen. Insbesondere kannst du in festgefahrenen Situation versuchen einen Kompromiss vorzuschlagen.

\subsection{Die Abstimmung}
Der Ablauf der Abstimmung ist üblicherweise sehr genau durch die Berufungsordnung geregelt. Beispielsweise wird meist ein doppeltes Quorum (Mehrheit der Professoren und Mehrheit der Mitglieder) gefordert.

Fast immer ist die endgültige Abstimmung über die Bewerber geheim. Dazu solltest du dich auch auf jeden Fall mit deiner Berufungsordnung auseinander setzen. Gerade wenn du überlegt gegen den von der Kommission favorisierten Kandidaten zu stimmen, solltest du dir das genau überlegen. Es ist meistens auch sinnvoll, schon während der Beratung deutlich die Vorbehalte vorzubringen. Wie du letztendlich über die Bewerber abstimmst ist aber natürlich eine Gewissensangelegenheit. Ob du dagegen, dafür oder mit Enthaltung stimmst, solltest du dir situationsabhängig überlegen.

Die Professoren möchten üblicherweise ein einstimmiges Ergebnis, um den weiteren Weg des Berufungsverfahrens durch die einzelnen Instanzen einfacher zu gestalten. Daher hast du als Student ein gewisses "`Druckmittel"'. Vermutlich wirst du den ersten Listenplatz damit nicht ändern können, aber weiter hinten auf der Liste kannst du so evtl. ein pari passu oder einen Sperrvermerk erreichen.


\subsection{Das Sondervotum}
Wenn du gegen einen Kandidaten stimmst, er aber trotzdem auf die Liste kommt, so hast du üblicherweise das Recht ein Sondervotum abzugeben. Das bedeutet, du kündigst (je nach Berufungsordnung) unverzüglich nach der Abstimmung an, dass du ein Sondervotum einreichen möchtest und reichst dieses in der gegebenen Frist ein.

Der Vorteil eines Sondervotums ist der, dass dieses durch alle weiteren Gremien wandert und du so die Möglichkeit hast, deine Bedenken begründet weiter zu geben. Ebenso solltest du in einem solchen Fall auch direkt die studentischen Vertreter in den nächst höheren Gremien (Fachbereichs-/Fakultätsrat, Senat) informieren, damit diese mit einem gewissen Hintergrundwissen diskutieren können.

Jedoch solltest du dir vor einem Sondervotum stets darüber klar sein, dass dieses eine sehr deutliche bzw. extreme Meinungsäußerung ist und bei einzelnen Mitgliedern der Kommission für Unmut sorgen könnte.

%%% Local Variables:
%%% mode: LaTeX
%%% TeX-master: "../bkhandbuch"
%%% End:

\section{Studentisches Votum}
In einigen Bundesländern hast du die Möglichkeit bzw. die Aufgabe ein \emph{Studentisches Votum} einzureichen. In diesem Votum sollen die Listenplätze aus studentischer Sicht und vor allem unter der Frage der Lehrqualität verglichen und diskutiert werden.

Wenn du ein solches Votum schreiben musst, sollte das Folgende darin vorkommen:
\begin{itemize}
    \item Diskussion der einzelnen Bewerber aus studentischer Sicht (Qualität des Vortrags).
    \item Inhalt sollte ein "`Votum über die Lehrleistung der Kandidaten"' sein.
    \item Jeder Bewerber sollte in ein paar Sätzen einzeln diskutiert werden.
    \item Abschließend sollte ein Vergleich aller Bewerber erfolgen.
\end{itemize}

Es ist sinnvoll zuerst zu beschreiben, was Grundlage deiner Bewertung ist. Wenn Du für den Vortrag/Probevorlesung Fragebögen verteilt hast, dann beschreibe kurz, wie dieser aussah. Erläutere auch, zu welchen Themen du Fragen in den Interviews gestellt hast. Bei der Bewertung der einzelnen Bewerber braucht man nicht in "`Arbeitszeugnisdeutsch"' zu verfallen. Statt "`Er hat sich bemüht eine gute Vorlesung zu halten."' kannst Du einfach schreiben, dass die Zuhörer die Vorlesung nicht gut fanden. Natürlich sollte kein Bewerber zerrissen werden, aber wenn jemand eine schlechte Vorlesung gehalten hat, solltest du das auch zum Ausdruck bringen.

Zum Abschluss des Votums solltest du dich dann dazu äußern, ob du mit der Reihenfolge der Berufungsliste
einverstanden bist, oder Vorbehalte gegen diese hast. Vorbehalte solltest du dann ausführlich begründen.

%%% Local Variables:
%%% mode: LaTeX
%%% TeX-master: "../bkhandbuch"
%%% End:

\section{Der weitere Gang}
Nach der Abschlusssitzung wandert die Vorschlagsliste weiter an die höhere Gremien. Deine Arbeit endet aber noch nicht ganz. Falls es in der Kommission etwa sehr kontroverse Abstimmungen gab oder gar ein Kandidat auf die Liste genommen wurde, welchen du für nicht listenfähig hältst, so solltest du dich als nächstes an deine studentischen Vertreter in den nächst höheren Gremien wenden. Nur durch deine Informationen können diese in den folgenden Diskussionen nämlich die Probleme korrekt wiedergeben, die du bei den Kandidaten gesehen hast.

Sollte die Liste dann irgendwann beschlossen sein und ein Kandidat tatsächlich den Ruf annehmen, dann ist die Kommissionsarbeit schließlich beendet. Es kann jedoch noch der Fall eintreten, dass eine \emph{pari passu} besetzte Stelle noch einmal zur Stellungnahme an die Kommission zurück gegeben wird.

Sobald nun aber ein Kandidat zugesagt hat oder die Liste erschöpft ist und kein Bewerber gefunden wurde, endet tatsächlich deine Arbeit in der Kommission. Zu guter Letzt wäre jetzt der Zeitpunkt gekommen, zu dem du deine Erfahrungen (soweit möglich) für die Fachschaft dokumentierst.

%%% Local Variables:
%%% mode: LaTeX
%%% TeX-master: "../bkhandbuch"
%%% End:


%%% Local Variables:
%%% mode: LaTeX
%%% TeX-master: "bkhandbuch"
%%% End:
